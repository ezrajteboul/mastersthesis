% Appendix A

\chapter{Interview Transcripts}\label{app:inttrans} % For referencing this appendix elsewhere, use \ref{AppendixA}

\lhead{Appendix A. \emph{Interview Transcripts}} % This is for the header on each page - perhaps a shortened title

\section{transcript for interview with Sang Wook Sunny Nam}

1. How did you first get interested in audio technology? What was its part in your learning process as a mastering engineer?

I went to mastering in 2000. There were no resources to learn mastering, but there were books and photos about studios in the US - I was in Korea - and there were tools that known as applicable to mastering. So I used these, mostly digital items like compressors. Plugins were just born by then, so we didn't use them. So we had digital hardware, like waves from swissland (?), and more things from Germany.

After I went to the west coast at the Mastering Lab, they had a completely different concept of the gear they used. I had to learn everything all over again. They're still quite different from regular studios, in those they use mostly off-the-shelf gear. But I had to learn the history of all the recording equipment, because a lot of our designs were based on work from the 60's. I learned how they worked, how they were designed, how to fix and improve them. They're in a smaller form factor now.

Do you hold the distinction between discrete transistor circuits and IC circuits as being important in this context?

Yes - discrete, tubes... transformers, resistors... every part is important. They've all changed, become smaller. The overall quality of the parts and the circuits they're in have been in some sense compromised. I learned how the new technologies, how those smaller form factors damaged the sound of the equipment, and how to keep away from them, all that sort of stuff. What parts I need, how to evaluate them, where to source them...

So you've listened to ceramic vs. tantalum capacitors...
Right. That's easy. But polyester vs polystyrene? What type of polyester? What type of structure? Even with the same structure, what company makes a better polysterene or polyester condenser?

Have you done double blind A-B test for all of those variants?

Yes.

When did you do those tests?

I didn't have any need to do those while working at the mastering lab because my mentor had already done all the listening tests. Wires, switches, volumes, everything. When I left the company, I did it all again myself, because he didn't tell me all of his results. I had seen a few, I knew some of them, but most of them aren't available in the market. In those cases, I had to find old stock from somewhere, and if I didn't find those, I had to find an alternative. So I go on Ebay to look for old parts - from american to russian military to european parts... I went through, bought samples, listened to everything...

I\'ve had to do similar work in the past, and it always seemed like a fun aspect of projects.

It's a fun part, if you're a student. But if you're spending money and time, it becomes more of a problem...

2. What do you think of your current setup? Is there still work to be done, or are you happy with it?

I'm fairly happy with it. I have some equipment in my mind I'd like to build, still... but my clients are very happy with the outcomes, and I have quite a bit of freedom to do what I need with it, so I'm pretty happy with it.

3. The fabrication of your custom equipment involves contractors and engineers - most of it isn't built by you. How much do you document this collaborative process?

Well they submit all the schematics and that sort of stuff to me, so I have the final result and a sketch of the design. I also keep all the email correspondence, some notes I take when I evaluate the products...

Do you share any of this documentation?

Oh no. It's a very time and money consuming process. Very good polystyrene condensers cost 200 piece. If I'm testing a stereo pair, I'm spending 400 just for those. Unfortunately, they didn't sound good - so I end up spending 5 to 7 grand just on capacitors. So anybody can do it, but, you know - it's my money and time, and it's really hard for me to share that for free.

4. Do any of the people you contract for this equipment end up sharing the work they've done through collaborating with you?

The designer and builder of my equipment (Josh Florian - JCF audio) is my close friend, and he's pretty secretive of the information he gets from me. He doesn't use it in his own work or other commissions. Also, its very expensive for mass product, those parts are hard to get...

5. You've mentioned that learning the equipment was a big part of your education. I'm guessing the other part was training your brain listen to what you want to do, what you need to do with a recording when it comes in your hands. For most projects, what's the technical vs. the mental?

Psychological understanding of what the gear does... will take weeks, months, years. These pieces of equipment I've been using are quite different from modern EQs, compressor... if you've learned how to use commercial standards, you'll have to learn to adapt to mine. I'm using shelf eqs, so I don't have any Q control, any peak curves.

Is it an active circuit?

No, it's all passive. There's only one amplifier at the end. With the parametric EQ most people are used to, it's really easy to go to the frequency you want and take it out, or add something... but you have to be very creative to make something peak- like with a shelf eq. Also the bands are very limited. You need 2 shelf bands to approximate the behavior of one band of parametric EQ, and I have four shelf bands. So I have a very specific strategy to play with the balance of sounds. Also, because those are very wide EQs, you have to get used to that. It's like using nothing but primary colors to paint an image. If you have 48 colors, you just pick up the right one and paint with it, but this is completely different. So I had to learn to adapt my process to these mechanisms, which are very simple but very hard to use for some complex behaviors that more standard equipment can do easily.

Does having a say in the way those items are designed probably helps make it easier for you to use them?

Yes, that's true. Also, certain EQs have some things that they can do better than others. There's two types of filters: LC and RC. LC means you have to use an inductor. If use that, there's a resonance, so you can get a little bit of a peak curve out of it. Also there's the specific sound of the inductor, which will color the output - it might sound a bit more aggressive, or euphonic... so you have to know what this EQ would sound like on a particular sound.
What order filter are you using for your mastering work?
Because it's all passive, I'm using all 6db/octave first order filters. If you want to do 2nd order, you get more parts: more distortion, more non-linearities, more noise. 6dB/oct is more than enough for most cases.

Most EQ slopes in DAWs can get much higher very fast, with 24 or 48 dB per octave being common.

Yes, 24, 48, butterworth... those are common. If I have to do anything that drastic, I can do it in the digital domain, or I can ask for it to be fixed at the mixing stage. If you want to do that at the mastering level, it's a fairly significant problem, so you don't want to do that... That's what's less harmful.

6. You've worked with Josh as your go to engineer?

He was assistant engineer when he started at the mastering lab. He found at that his interest was in electronics rather than recordings. He's also a great drummer. So he learned a lot from our resident tech back there, what he calls \"yestertech\"... tube electronics, discrete, power supply, grounding schemes, all sort of stuff. Then he developed all the new stuff out of that and became the owner of his company that makes great products...
Do you know any other hardware engineers you'd put on the same level?
One of the resident techs went independent thirteen years ago. He's not making any mass-marketed products, but he'll do commissions. His name is Steve Hazleton, in Tennessee. He's another guy that I can go to if Josh is not available. The other people... people who make mass products for consumers had to deal with all the safety issues and regulations that can be problematic for some people. They put a lot of stuff in their designs that ultimately compromise the quality of their work... so I wouldn't call them.

You made it seem like there was a back and forth between you and the designer to get these electronics together, starting with your query for a specific item or modification. Does this process go both ways, and do you listen to some of their recommendations for mastering equipment?

Yes. The equipment that I have... When Josh first listened to what I wanted, the equipment I needed and the topologies I wanted to use, we had to improvise a little bit, which I wanted to do. There's definitely an interplay. He developed a couple of things I didn't think of, and sometimes I'll suggest something he had not considered. That's always going on.

7. Is there one specific item that illustrates this process well in your studio?

The monitor panel is all custom made. I wanted to listen to what's coming in, and what's coming out. What's coming from the DAW, and compare everything. I told him I wanted those things, so he'll develop a schematic, and explain how each volume is controlled, how to implement mono... how to make those three inputs independent, how to make the panel that has the less contacts. The more connections, the more contacts. Even though we're using very good switches, you'll lose some details. That sort of thing can be more specific points developed by him... ultimately, we decided on three contacts and he built it.

The compressor is another example. I wanted a low pass filter on the side chain so it doesn't see the big bottom end when it compresses it. Since we were using 4 channel switchers, he said we could make it a variable filter, so I decided on four values. 100 Hz, 200 Hz, something else, 400 Hz... That sort of process.

You use only switches and resistor networks, no potentiometers, no faders?

Yes. No faders. Only switches. You can also use relays, or cheaper switches, or any VCA type OPM for switches...

Do you do that?

No, but it's a possibility. But all these alternatives make you lose a lot of low-level detail, so I avoid it. Also the materials of the contact, and their structure... the one I'm using has two contacts. It's military grade silver contacts - gold has a specific sound that's not very useful. So does the copper... Everything has a different sound, and silver is my favorite. It's an expensive switch. It's double sided, so a one pole, twelve out... just one wafer is 60-70... and you have to order thousands of them. The price is just...

Those types of switches were used solely on my console. Volume controls are also only resistive networks.
Do you try to minimize the amount of wiring in the overall studio?
Yes. All hand wired, short. There's still a lot...

Is it all point to point soldering?

Yeah.

So no ceramic boards or circuit boards?

Well, some of the equipment, like EQs... is built with circuit boards. But the console is entirely point to point.

Do you ever take a look at the inside?

Yeah, every three months I'll open it and clean all the contacts.
Do you ever appreciate that wiring as a work of art in itself?
Yes. Fortunately I bought the remnants of A\&M mastering... the founders of that studio were working at the mastering lab and built the console for A\&M studios... So they had all the parts I wanted. And A\&M's philosophy was really similar to the mastering lab's so I was very happy to get that. I had all the good switches, and an already made point to point console. We had to adapt a couple of things, but most of the work was already done. Josh had to do a lot of wiring, but still...

Wiring all those switches... I can't do it. Somehow Josh can do really fast.

Would you ever take out a soldering iron and fix something yourself if it needed it?

No. I'm not good at the smaller things.

8. How do you know if/when a particular piece of gear is finished? (25min52sec)

We'll discuss the topology. For amplifiers, for example, I usually avoid op-amp designs. Even with discrete op-amp based circuits like the 2520 or 990, those use a lot of feedback, and I don't like the sound of either. So we were talking about discrete amps, like a quad or two push-pulls topology? The shelf EQ has a loss of around 21 dB, so will those circuits have enough gain? He'll build a prototype and we can try it and listen to it. He'll make a test board that has a 21dB loss to test it with, I'll listen to it for a few days, send it back with comments...

After he is done the designs, I listen to for a week or two for anything active. Sometimes he makes a mistake and process stretches a bit, but usually it's just one trip and... done.

At one point, in the very first stage, he sent around 4 different amplifiers with variations for me to try, so I picked my favorite and refined that design. I don't want to overuse one amplifier, because even though they're very transparent, it still has its own sound. I don't want to add the same topology multiple times, otherwise those characteristics compound. So if we need another amplifier, say in the compressor, I'll use a different topology - maybe tubes, or solid state topology.

9. To what extent do you participate in online audio-enthusiast communities and what role do they play in selecting your equipment, if any?

Human ears can get off-road really quickly. Ears work as comparators. You'll listen to one thing, then another thing, and always compare. If you hear a very bright tune, then a well balanced tune, the balanced one will seem dull. It's very easy to loose objectivity in listening. I had to find a few people that know how to listen to low-levels, to electronics... This is what I usually say: listen to the quality, instead of the quantity. One dB of EQ is the same if you look at the quantity, but every unit has a different sound. People have to pay attention to that, and I haven't found many that do, so I always ask them for a second opinion if I need one. One of them is Bill Schnay, in L.A., and he has a very high resolution recording studio. I listened to capacitors there to pick the ones that sounded the best: we tracked drums in the live room straight into the switch with a very high resolution microphone, and I'll listen to a capacitor with the lowest capacitance to get a sense of what it'll do. He was sitting next to me and helping me, making sure we remained objective. Having a second opinion is really important.

But, online... no. I go there, to find out what the next product is and what people think about it. But the ultimate decision is always with your ear. The people who are lurking... active on those websites... if you're busy, you don't have time to do that. What they're saying is so not true...

10. Do you measure the performance of your gear electrically, and if yes, what importance do those measurements have?

I do. Good sounding gear has to measure well, but the opposite is not true. Like the 990, it measures well but I don't like the sound. When Jensen first made it, we tried it at the mastering lab, we had a very big live sound at MGM - now it's Sony, back then it was MGM - we put fast instruments like tambourine or any percussion 40 ft. away from the microphone, and rung the bell, or the tambourine. The tails of the sounds decrease, the microphones record that, then we listen to that through high resolution amplifiers that we know with the same gain and it what it does is \"ding\" then the decay is going on. We can still hear decay all the way down on a good amplifier. The 990 goes \"wiuuuu- ss!...\" because of the internal feedback. For measurements' sake, you cut out the low level detail, and they think it's noise, and the feedback system cuts it out. But there's still information in that. So measurements isn't a guarantee, but you still need very low distortion measurements to sound good.

Do you look mostly at THD (total harmonic distortion) measurements for this?

The problem with THD and some of those measurements is that they're done using sine waves. What's the easiest thing to measure? Sine waves. What do sine waves have to do with music? Not much. That's another problem of measurement. This can be a good measure of what the amplifier does, but it can't be all of it.
You have favorite topologies of discrete transistor circuits and tube circuits. Do you want to talk about that a bit more?
There's not too much... I like less parts. If I can do with two push pull transistors or tube, I'll use that. If I need more gain, I'll need a stronger design... Tube can be very euphonic, with low distortion, but mostly so at low wattage. Their problem is when they need more power. But if you think about the music, everything is a few watts at most. Drums, or fortissimo would draw a lot of current and require more wattage. But most of the time you're drawing 1, 2, 3, 4 watts. So tubes are great for listening, but not necessarily when you're mixing or mastering those strong transients.
Have you ever considered an hybrid, adaptive amplifier that distributes the load between a single ended vacuum tube amp and a push-pull transistor amp based on the dynamic range of a piece?
I'm not a real designer, so I don't know what that involves. You'd probably need a lot of interfacing between the two sides? I don't know... I have used a tube push-pull amp, and it was really good.

11. What is most important in your work process?

At the end all that matters is the sound you're getting. It's hard to be objective if you know certain things about your equipment. Forget what it is, listen to what it does and what it can do, where the limits are. It's very bad to have all this technical knowledge without having the listening abilities. I can see a lot of people with that problem, especially on websites. They understand how things work, what's new, but they don't listen.

A funny thing: one guy asked what the best DAW was, and a thousand replies were added on that thread. \"This DAW has this function\", \"that one has another\", but over those thousand replies, no one talked about their sound. That's the current situation of the engineers, and it's really bad. They read the articles and they know the technology... but no listening.

Dither - lots of guys talk about dither. Lots of dithers available. But in my ear, most times dither doesn't work well. For example, if you're using four plugins on one stream, all of those are going to add dithers. If they're using high frequency-boosted dither, even if it's inaudible in one plugin, if you have for or five you can hear that sound and it affects the music. So I tell them to turn of the dither... and they say \"oh no, truncation of errors... no...\". All this information about technology is influencing the way people hear music, and the new technology is always better... In some areas, that might be true, but in music it's not always true. In my opinion, and the history of music says that too, 50-50 is very generous numbers for new technologies. I'd say 20-80. I'd wait 2-3 years after something came out, and trying it in my studio before having an opinion on a particular item. OS, or Pro Tools updates... all sorts of stuff.

In the end, what makes your studio special? You've spent a lot of time listening to everything, and that's why people hire you?
Mastering is special, because in a recording studio you have 48 or so channels to control your quality. If there's a bit of loss in one channel, the consequences aren't always serious. In my studio, my hands are tied to two channels. If I don't use those two channels better than other mastering engineers, I have a big problem to start with.

After that, even if you have a great equalizer, if you play the music wrong, you're doomed. The first thing I can say than my studio can do better than any other studio is play two channels. My DA converter uses a very specific topology different from any other mastering business. It doesn't have any digital filters. You need digital filters to do oversampling - I don't do oversampling in my deck. I don't have digital filters. I have a special process that fixes oversampling and digital filtering that was done during the recording.

So that's the most important part for me. How to play two channels right. In a musical way.

Do you know anyone who has built their own mastering studio from scratch?

All this? Yes, Bernie Grunman. A\&M... which is gone now... the Mastering Lab. So Bernie, Mastering Lab and me would be the three... All the others... I don't know anybody else.

No one who wasn't a professional in the first place is doing this from scratch.

Right. Also, you need to have a tech to build all this, understand the topologies and how to use them.

That's too much for one person?

And too expensive. All the young engineers who build the gear don't have deeper knowledge of yestertech. Everything I see today has remnants of a 2520 or a 990. Some people work with the quad-type amp. But those three represent most of the designs. So it's a pretty limited knowledge base. Now if you have the resources to look for more original things, you have a lot more to work with. The encyclopedia of audio - the first edition, from the 50's. That's the real treasure.

\section{transcript for interview with Tristan Shone of Author and Punisher}

TS: I've gone full circle: engineering, punk bands, metal bands, back to art school, away from engineering, did some rough and tumble sculpture that was not very satisfying, then I came back into music because it's much more natural for me and I've become a musician at heart. Back to doing engineering to make the money and balance off the touring. Everything's just kind of a big clusterfuck right now, trying to manage a musical endeavor, with trying to get back to really doing instrument design.So much of my life is taken up by booking and promotion, running a business as a band but also balance new design and art grants... there's a lot of logistical crap that seems to take most of my time. Which is maybe not as interesting for your thesis. 

ET: can you develop your background a little bit more?

TS: I don't really have any composition experience, other than I'm a trained piano player, and I learned to play guitar, so I could play in metal bands. That's where I came out of college, interested in Robotics and control systems. That's what I did as an undergrad at RPI in Troy NY. I was playing in thrash metal and apocalyptic doom bands from the mid / late 90s, neurosis, melvins, godflesh... also some drum and bass, electronics... while being in my classes, and working on stuff like electric cars, at RPI, I was helping with that, and assisting some professors and learning mechatronics... I think I realized I didn't want to go to grad school for engineering because... I really liked the gadgets and I liked mechanical engineering, and I loved theory of control systems and robotics but it was a bit too much. Not as interesting as physically making things work. I liked machining and fabrication, so I went and got a job in a clean room. This was during the telecom boom, around 2000. You could make a good salary out of school. So I said screw it and went to design these automated systems for testing MEMS - micro electro mechanical systems. Semiconductor based machines. Other people would etch these out of silicon, and craft these little machine that would basically switch fiber optic lights. I would make these setups with x-y-z mechanisms that would test them... It was a bit dry. Being in the clean room after being in a college band and touring... College life to working in a clean room with really high level scientists was not really my thing. I did a few years of it, switching around to a few different companies, and playing in a band in town that never really went anywhere but also... I found some companies I liked to work for. There was also a professor at RPI I worked for, an art professor called Chris Chicksemihay. He's a media artist, he was at MIT's media lab, and I sort of helped him while he was there designin some parts for his installations, travelling with him a couple of times. I went to Finland, met a lot of people in the media art world, kind of introduced me to what you could do with sculpture and mechanical sculpture, microcontrollers before the arduino... They're easy to use now, you don't have to compile 70 files and set environmental variables on your computer... we were using those... devices that he had developed while at MIT. That's most of my engineering before grad school. Going back was as much a decision for my carreer, wanting to work with art, as it was wanting to leave corporate america and having nothing to do with it. 

ET: where did you go for your art program? 

TS: I went to UCSD. Which is also where that guy Chris Chicksemihay went, one of the guys in the Yes Men was there. Who else was there... Barbara Kruger, Jean Pierre Gorin... Lev Manovich... Good professors, and a very tech oriented school. That's where I work today, in the neuroscience department. We do imaging microscopy, and I work on all the automation, with a bunch of biologists and physicists who come up with ways of doing things. 

ET: were you familiar with all of RPI's experimental music aspect? 

TS: Yeah, I was somewhat involved in that. That professor I worked for was in that... I think that the music department was pretty small at that time. Then they built that giant building, EMPAC. I'm in touch with some people there, we tried to get a performance there, and it never happened... 

ET: Can you talk a little about the connection between metalworking and metal music? There's something very physical about both (12:25) 

\section{transcript for interview with Nicolas Collins}

(conversation informally starts on a discussion of headphones, earmuff padding, and the suggestion that the two recordings of the chat will be used with one out of phase to create a composition): 

NC: ... In some concert halls, they have devices to jam cellphones. This is known. I've seen a handful of references to circuits that do this. this isn't like building a nuclear centrifuge, anyone should be able to do it. The only tricky thing is working at such high frequencies that cellphones operate at. It's not as easy as building a fuzztone - the level at which I work.So I see a couple of versions of this, small ones not powerful enough to fill a whole hall, but I love the idea of carrying something the size of an iphone that creates a black hole around, wherever you go, so that in a 3 meter circumference everyone's cellphone's stop working. 

ET: Have you ever read the Pirate's Dilemna? It opens with a discussion of the ethical implications of using those ipod radio-emitters for your car. They could jam a frequency not only in your car, but in a 2-3 vehicle radius around you.

NC: There's so much interest in what's called \"silence studies" in the last ten years. This is cyclical. There's a Cage / Rauschenberg moment in the 60's, then it came back with a vengeance... It took so long for Cage's ideas... not to be accepted, but rather internalized. For example, people from many aesthetics could view silence as a positive element, rather than as the absence of something. It hit something, at the turn of the millenia, when all these people realized you could carve things out of all those negative spaces... and in some fields this had been accepted for a long time. Graphic design... It's one of those things that follows a zeitgeist or a pattern or a cycle... That idea comes up a lot in hacking. That idea of injecting something or removing something...

ET: I'd like to try and go through a few questions, hopefully we can develop that as we go along those. I wanted to start with something basic. I'm interested in contemporary practices in electronic music hardware design, trying to link the first electronic music instruments with diy and tudorian electronic music and today's open source movements. You're in a great position, where you've had a chance to work with Tudor but also see taken advantage of contemporary technologies...

NC: I'm old, yes. But I'm still alive. 

ET: and you wrote the book! 

NC: fine... I did do that. 

ET: So what's the current place of hardware in what you do today? 

NC: ... (pause) you know... because of my age... the arc of my material resources are a little different... from yours, from those of my mentors. I'm from a particular generation. My first work was based on hardware because when I was 17 and wanted to do this there was no computer. It was the epoch of mainframe computers. You would not get access to those as a young person, and even then they were an offline, non real time thing, and from the very beginning I was interested in real time music production and performance. So the alternative was synthesis. This was the era of synthesis. However, synthesizers were also impossible to afford. None of these technologies were what you'd call personal. Ownable. 

But it was at that moment that integrated circuits went from being extremely functional building blocks, transistors out of which you'd design a whole circuit... to... more modular things, that could do more things out of the box. The most critical chip was a signetics 566, which was an oscillator on a chip. 8 pins, you hook up a very small number of components and you'd get a couple of waveforms. It was designed for touchtone telephones, which is the only reason it existed, because that had a huge market. And across the board, you'll fine people from my generation and a bit older for whom that was the first thing they've ever worked with. Because who doesn't want an oscillator? 

So... my entry to electronic sound was hardware. But by the end of the 70's, microcomputers emerged. It was pre-apple, industrial computers. It was sort of like large Arduinos, right? And it was a lot like working with arduinos. So composers from my generation, who are now say between 55 and 70, really dove into the computer stuff early on, and by the mid 80's it was sort of a no brainer. You could get so much done. The biggest drawback was that you couldn't really do internal audio processing on a personal computer until the 90's. But there was so much available in terms of MIDI controllers and everything, and I basically since 1979 kept two parallel tracks of doing these circuit based things and computer based things, because the fact of the matter was I wasn't terribly interested in the sounds of synthesizers... so I had to find other roots. 

ET: and a summary of that is in your paper about early microcomputers in your practice? 

NC: Yes - that's the paper on semiconducting. It's the idea that certain technologies have natural strengths. from my standpoint. And most composers think of orchestration as a decision, rather, than say writing for violas the rest of your life. So for me, switching back and forth between technologies was no different from switching back and forth between instruments. 

But I have to say, as the decades passed, it was obvious that the computer was becoming the more powerful, more versatile tool, and if I wasn't willing to spend the time being a brilliant analog engineer - I was always self taught - there was much more possibility and much more openness and much more of a community for a sort of open source in the software domain, rather than the hardware domain. But I kept a hand in the hardware all throughout this time. If you look at the few records I've done over time, there's all these oddball instruments. Hybrids of electric and mechanical things. Sometimes maybe guitars, live sampling systems... It was all a mishmosh. What happened, what changed for me was that the end of the 90's, I started teaching in art school. It was this moment where you may be able to identify more clearly... I call it the digital hangover. The computer had become so powerful that people were just knocking back shots without thinking of the consequences. You couldn't really do anything. My mantra's always been control-x / control-v. It's the world's most powerful tool! You can cut a term paper, you can cut audio, you can cut video, you can design a website. It's the world's most amazing pencil. But as I discovered, from the art school context, art students are peculiar in the sense that every single one of them, even if now they do exclusively digital, they all started drawing. There is not an artist in the world that didn't scribble, even if now they use a mouse. And that seems to be really ingrained in visual artists, this desire to do things with their hands. We think of that as a musician's thing - musicians are about the tactile. But I think that musicians play their isntruments 24 hours a day. They have a nice life+work separation. Artists are always fiddling with something. 

It was those students who pushed me to do the class, and it was this generation of hungover... from digital overindulgence... that led to the rise of circuit bending. Because the circuit bending movement went back to the early 90's, when he started writing articles for the experimental musical instruments journal. And there was always a little cult of this stop. Always this buzz in the air about the speak and spell. I had a speak and spell in 1979 that I hacked. this is pretty basic stuff. But he took off at the end of the 90's, with this sort of anti-computer backlash. For a while people were waking up one morning and saying oh \"I'm never programming again". And for a while it was like that, a real split between the circuit bending people and the computer music people, and they basically had nothing to do with each other. Circuit bending people were militant about their anti-computer stance. Porta-studios came back with a vengeance, the casette was a real format... It was almost like a luddism. But then... a few things happened. The most important one was the sort of parallel growth of limits in the open source community and the arduino. Those two ... people had been making arduino type things since the 80's... STEIM made this beautiful sensorlab thing- but it was \$3000! Completely insane. So the combination of the affordability of the arduino and the open source nature of doing programs on it and the fact that they had provided this glue between the physical world and your laptop meant that it was like the peace accord in Belfast. Suddenly catholics and protestants could talk to each other - over the top, but I think a lot of orthodoxy broke down at that point. 

ET: people realized the speak and spell used microcontrollers. 

NC: That's what I tried to tell these guys. Every single toy they use is a sample-playback computer. I did a workshop with the other Nick Collins, in Mexico, some years back. There was so much confusion about the two of us. He's a supercollider maven, and the organizers could not figure out these two dudes were two different people, so they built these multiple workshops with Nic(k) Collins, with no indication of which was which - there were 2 or 3 of them. And of course everyone came to all of them and they didn't know what they were going to get. We decided we would simply do the workshop together and every hour we'd switch between software and hardware. It worked! It was clearly the threshold point. Everyone was equally comfortable working in the two modes, which was a big change. Where does that gets us?

It gets us where we are today. Coming into teaching late I'm much better at making the distinction between my life and my job than I was when I was a grad student. But I can't deny that teaching, not only in chicago but also these countless workshops have fed back into my own practice. I got interested in one very specific thing at the beginning which is that when I would do a workshop I would have 25 kids sitting around a table with little amps and speakers working on kind of similar projects or technologies at the same time. Everyone would be working with contact mics, or making their first oscillator. But it was this great orchestral electronic sound, that wasn't mixed down to a p.a.... it's also in the same general region, but uncoordinated. Now for a guy who's background is in deep minimalism... I started opening up to a chaos... the things you can get with a large number of human beings that you can't get with a line of code - unless you're really really clever - and I'm not, I write relatively simplistic code. So I got interested in the group dynamics of hardware based stuff, where you don't control things as accurately as... god forbid, a guitar, in your hand. 25 electric guitars in a room, it'd be a very different experience. I got interested in the noise world. The sound world of... disreputable electronics. Electronics that you weren't sure were working correctly, or that you knew was damaged but still interested in the sounds it could make... so I did a piece called \"Salvage" - it's on youtube - where you try to revive a disfunctional or broken circuit by essentially injecting voltages into an unpowered board and basically using it as timing components for oscillators. So you get a very complex oscillator with a high degree of chaos in it. And it goes through a set of complex evolutions as more people start joining. There's a very simple instruction set. The idea is that it sarts out relatively cause and effect-y, because there's only one person doing this, but by the time you get up to 6 you get this sort of density of decision making that's very difficult to think about being done with a computer. 

That being said, you know George Lewis made these really beautiful softwares that improvise. George has been working litterally since about 1980 on program that improvise. And because he is such a great improviser, he's someone you should pay attention to. The basic idea's always been that the computer listens to the player, and responds as if it is a player. The reason I mention this is because instead of creating a standard algorithm for what its improvisation should be... to the best of my knowledge, and you'll have to confirm this with him, what he's actually done is that he's written different routines that embody different improvisers that he knows. So that in his computer he has multiple different personalities that behave differently in response to the same data. 

Now if I was smarter I would try to do something like that. Computer program that instead of having 6 people doing something I have one person do it, and then five \"people" to play along. At the same time, I always have a lot of warm bodies in this workshop and this one way to harness the energy. I've spawned a couple of solo pieces of from that. What I'm trying to do is harness the apparent chaos and comformability - seemingly incompatible - of some analog circuits - but use software as a way to get rid of the the sort of monophonic property that most circuit performances have. To create some sort of complimentary behavior. 

ET: The other person I'm interviewing tomorrow is Dan Snazelle. Are you familiar with his Ardcore module? 

NC: Yup, that's right. Do you want a précis? A short version? I've carried parallel practices in hardware and software for years and years and years. They've always worked together but what I would say is at the moment, it's the chaotic aspects, the instability of circuits that are coming to a full forward in the stuff I'm doing. 

ET: So has there been any one device or project that has created a noticeable shift in your work? 

NC: No, I think there's been multiple ones. David Tudor used to talk about how he never understood tubes, and then Gordon Mumma tried to teach him how tubes worked, and they tried to build a tube amplifier, and tried three times, finally giving up. It wasn't until the transistor came around that he was comfortable making circuits. 

For others of us it was the integrate circuits. I'm lousy with transistors, but ICs are a piece of cake. The more complete building blocks are great. My whole book is predicated on this CMOS logic circuitry (26:18) from the 70's that lent itself beautifully to running on batteries. That was a critical technological bridge. 

For most people, the advent of midi and pc with reasonable userbase, so that software could be made by people other than yourself. In the 80's, the conflation between the music industry and the computer industry was critical for a lot of people. It didn't matter so much for me, a lot of my stuff had backed off from the computer, but for the community at large... 

Then when computers actually got fast enough to do real time audio processing. So when Max went from being a MIDI generator language to control synthesizers to having an MSP component that allowed you to do direct sound manipulation, that was a big deal. 

And I think... I don't do a lot of stuff with Arduino at the moment, but I know that that has been the next big step, because it's solved the problem of connecting the computer world to the physical world. Foundations like STEIM had been working since the mid 80's to make that work, spending billions of dollars on artists residencies and research. And suddenly, this Olivetti guy shows up and \$25 later, you got it all worked out. So you know, open source and Arduino would have been the next big step. And I suspect that this is going to be very important. My guys who started the first laptop orchestra at Princeton are now doing iPhone orchestras at Stanford. Ge Wang and Perry Cook... they had a very conspicuous laptop orchestra, and when Ge gets out to Stanford, he ups the ante and starts a phone orchestra. 

ET: Dan Iglesias made a nice wrapper for LibPD called mobmuplat...

NC: Oh right, that's why the name is familiar. It makes a lot of sense, there's still a lot of people who don't want to use their computer on stage. They just like the idea of wrapping it up in a smaller package. A point was made to me that people are developping apps much faster than people are developing full feature software for larger platforms. For every major rev of Ableton or Max you have a million new apps that allow you to test all these areas of work.

ET: Is there anything you're curious to see implemented? 

NC: I would be interested in - and I think some people have done this - but I'm very interested in sort of the electromagnetic spectrum that we have around us. Kristina Kubisch does these really beautiful EM sound walks, and I do all these things with coils in my workshops where you pick up the sound of your iphone... but I've always been curious what the wifi traffic sounds like. Make a really simple receiver in that bandwidth, with a frequency shift to bring it all down - not to steal the information, I couldn't care less what people are doing - but to hear if there's any rhythmic quality to the community that's working in that spirit. 

ET: there's somewhat of a visual equivalent that's been done, with some code sniffing all the image content being downloaded for people's webpages and attempting to recreate an approximate mosaic of the overall network's image consumption. 

NC: and I've seen demonstrations of some of these slightly suspect softwares that allow you to look at wifi traffic on a network. clearly it can be done. It's just that there's some difference between extracting the data, and my desire is so much simpler - what is the sound of all those things going back? You do have to do a little of stuff, because even when there's no data there's still a constant carrier, so you have to get rid of the droniness... 

It's an interesting point. There's this piece called something like \"just because you can, should you", and it's a reaction against the diy community. There's just a sense that its creating so many things? do you need to be making all these things? The downside of this world you're looking at is it leads to a preponderance of things. It has environmental impacts. Recycling software is much better than throwing out a circuit with a battery in it. It's a question of resources. Then there's the moral aspect, the psychological aspect of hoarding, with being object-oriented. If you've talked to people in bending communities, very often the instrument remain in the forefront of musical practices. There's something pretty dreary about concerts a bending festivals. It often seems like the music might be an afterthought. There's nothing wrong with being a luthier. There are people whose tradition is building great instruments. It can be Stradivarius, it can be Trimpin, it can be the engineers that are behind the cracklebox, it can be bob moog, but they're not necessarily the people to who you want to listen to records by. So I think that there is a need to be clear about that. From Tudor's generation down, there's an air of tension. Am I going to be taken seriously as a composer, if I make this thing? Am I going to be taken for an artisan? 

ET: That's one of the things that's fascinating to me about Tudor. He comes from this very respected musical standpoint, but embraces the experimental electronics, live electronics practice, and he's taken very seriously. But that seems to be mostly because of where he's coming from. Not necessarily because people objectlively thought his electronics were producing compelling compositions... 

NC: It's complicated. He had this reputation as a virtuso pianist, and then the artistry was elevated in his role as the interpreter, the realizer of these Cage pieces, whose scores had to essentially be translated for performance. That act of conversion, he elevated to this high art, which very few people have reached since. After that came the creation of these electronic instruments in service of the cage scores, and after that came David Tudor as composer. It was too many talents that leeched stuff along the line... there was a smaller vocabulary left to describe it, so to speak. I'm very conscious - I've known the guy from the early rainforest period - I'm very conscious of the fact that it's only after his death that the composer aspect of him began to be treated seriously, in terms of the the written stuff... There's that issue of the Leonardo Music Journal called composers inside electronics, which coincides with the getty papers. That's where you'll see a nice overview of the different periods... 

ET:Do you know about the Little Bits kits? 

NC: I was at Moogfest, the guy from Little Bits gave a talk... 

ET: Korg has a series of synthesizer based ones... 

NC: I've seen a number of those things developed over the years... Radioshack even tried a few times to make some of those lego-y things to teach electronics. Since my kids were really into lego, I tried to show them that, the mindstorm things too... But my kids never latched on to that, and I never invested much time into using these for artistic experimentation... I think it's all quite good - here's my take, getting back to this idea of because you can do it, should you - we all tend to loathe ourselves and the group we represent, so I'm always very conscious about promoting ourselves and the ideas... I have this weird reputation as the hardware guy. If you'd read about me 15 years, I'd have this weird reputation as the computer guy. These things change. But am I weary of people setting camps. \"I'm not going to use unless I make it myself" or \"I'm not going to use it unless it's linux" or that kind of statement... but I do think that one of the great virtues of learning to program or learning to work with hardware is that we get a better understanding of the technologies that our lives are ruled by, across all domains. My father's generation was one that tinkered. He was a college professor, for christ's sake. He'd build a book shelf, not go to ikea - we didn't have ikea back then. If the car fucked up, he'd try to fix, whether or not he actually could. It was assumed you would open the hood and check the oil, and make sure the cables weren't frayed, and you' try to second guess your mechanic. 

I tell people how the first time I was in Europe in the mid 70's, I was in Germany and I saw that all the driving schools has these models of cars, cutaway models of car with this cross-section of the transmission an everything! Like the visible fish. I learned later that to get a driver's license in Germany at that time, you had to answer questions on the written test about how a car works. Not just what this sign means, but also explain how a carburator works... 

ET: which is what HAM radio tests are today... 

NC: True. Trevor Pinch edited a really nice companion to sound studies. There's an Oxford one and a Cambridge one, he did one of those. There's a really beautiful thing on German... in germany in the 30's, there was this emphasis on diagnostic listening. You would be taught to listen to the engine of the car to pinpoint defects. My father's generation, they'd be taught to replace the tap washer when it would drip. The idea was that the technology was open. Even if you didn't understand what was happening, people would open the hood. People do not open the hood anymore. One of the things that happens in my workshops that is ultimately the best takeaway, is that there's always someone that comes up and mentions that dreadful word, \"empowering". They may never touch a circuit again, they may have done this because they thought it'd be fun, or because their boyfriend was doing it or something like that, but they say it was the first time ever opening a radio or this or that. It's the first time they'd ever touch their part. 

ET: Do you feel like there's more than intuitive connections between this long-standing practice of opening things up for music and the more recent open-source movement? 

NC: I think it's very unlikely that an obscure music fringe had an influence... I do think that there are certain social trends or zeitgeist that have a long nose, rather than a long tail. A long nose, where you sort or see these signs of a build up. Take something like the arduino, which has a very strong presence in the diy community now - it's a very good universal tool. As I say, you can look at proto-arduinos that have been produced since the late 80s, but it had to hit a certain price point. Just like circuit bending took off because there was a shift in cultural consciousness, there was a broader acceptance that you didn't need to know what you're doing. My generation, even though we were terrible engineers, we really tried to understand what we were doing. The only reason we would do something interesting is because we didn't [understand what we were doing], but we tried really hard. When the benders came around, the whole idea was \"don't tell me how this work". I have this quote in my book, one of the first things that happened: I was setting up on a little table for one of the workshops and this mountain man comes up and asks \"are these bent or hacked?" So I ask him what the difference is and he says \"oh, um, bending means you have no idea what you're doing when you open it up, and hacking means you have a little bit of an idea." Then I thought, from an ecclesiastical standpoint, that's kind of interesting. So we had that ground shift, and it was the same thing with programming. I remember when we first started with microcontrollers it was like \"oh boy this is going to be hard, we have to learn how to do this, we have to learn how to do that". And my students discovered that no, all you have to say is I want to control the speed of a motor, all you have to do is search motor speed control arduino, you get a chunk of code, you cut and paste. So that is amazing. 

I think there is still an issue, which could potentially be called a problem, which is what we might call the preset idea. When the DX7 came out, it had such an amazing timbral palette, compared to most other things. 98\% of the users never got beyond the presets on the front panel. There were many, and they were very rich. Except they were finite in number, and after a while you could pick them out in pop songs. The algorithm for FM synthesis has a certain sound to it, but some people did remarkable stuff to it, really differentiated it from the presets on the front panel. The problem with bending is that in a way it's a bit like presets. In other words, we are now on the speak and spell preset or the casio preset - and you can identify them. With the cut and paste approach to code, it can lead to something similar, which is that module of code, which you didn't end up tweaking very much because it did a good job... now something like motor speed control is pretty utilitarian, but there are other aspects. But you look at other languages, Supercollider, Max MSP - those are very open as to what they can do. But both languages, Max in particular, come with all these modules, these objects, that are very powerful but also quite recognizable. There was a period of several years where I could identify a max patch just by hearing it. It mostly had to with the sample playback stuff that Max did very easily. It was another boom to people who were starting out, but it was a sort of presetting. 

So this is potentially a danger with the app market. It takes a very powerful programming environment, and it generates one patch, so to speak. If you develop it yourself, you'll spend some time tweaking this and that, then you can use it in 3 or 4 different pieces and it's adapted. When its an app, it sort of sits there begging you to use it and have it be your instrument. 

ET: For me there's sort of two origins for presets: the community at large, the programmers - those are the tutorials, example patches that are built in and the cycling 74 website - then there's the patch your friend made next to you. I think the distinction is important and interesting, for how those two communities work together. How do both communities influence your work? How important have other people been in the development of your work. 

NC: It's a mix. Statistically, the students of the workshops have been more influential than known individuals. With one or two exceptions. In the early days of the workshop, I had this vocabulary of techniques that were chsen because they were relatively easy to do, inexpensive, and most importantly they did things that computers couldn't do easily. I wasn't trying to do stuff... I was trying to match a market need. It wasn't a Moog synthesizer and it wasn't a computer. Along the way, the assortment of project and the tweaking and tuning of them was very much influenced by the feedback I'd get from people. The other thing is, people suggested stuff. There was this guy, John Bowers. A computer science professor, and he's also done really interesting low end electronic stuff. He was the one who showed me this business of the making a speaker into an oscillator with just a battery. He called it the victorian synthesizer. He brought it up when we were doing a workshop on loudspeakers and all the things you could do with them. Now this is a standard part of the workshop I do - it teaches you so many things, you can get it going instantly... But I look around and... I sort of see what people are doing. It feeds back in. But my general instinct is that I get more from general feedback from the participants. 

ET: so this is the side of the community that you feel is more influential than your peers? 

NC: yeah... maybe just because there's so many more of them... maybe it's because they're younger, and they have a keener insight into what's changing. I'm always looking for the next thing. Starting about 5 years ago, I saw this interesting, incredibly low level electronics. I see this sort of arc, which is best represented in Korea. There's an awesome scene in Seoul. It's Doto Lim and Ballon \& needle project. Otomo Yoshihide comes to Seoul, and it's like this catalyst for this sort of noise. And you see this evolution: lets start a band, then lets add the effects, then it gets noisier and noisier, and then they say lets disconnect the instruments and use only the effects. You go from Otomo to Japan Noise... then you get to the point where they say lets open up the effects, lets see what's inside, lets do a piece with just the one transistor we pulled out (56:37) from the pedal... let's just do something with dirty contacts. It's this funny kind of arc that's represented very well in the Korean scene. I've seen this post- effect pedal stuff happen. It's really interesting. 

ET: How do approach limitations in your work? Have proprietary tools and designs or planned obsolescence affected you? 

NC: The notion that if you pick up an object, whether its a violin or a chip, it has certain limitations built in to it, that would impose a method for using them? 

ET: Throwing back to the notion of presets we were discussing earlier. 

NC: That might be why, as I say... I for one try to avoid defining myself by a medium... like computer music or hardware hacking or chamber music or imprivisation. I'm sure there are people who are happy being in such a niche. I like string quartets or I like piano music or I write for Jazz bands... But my personal interest is to seek out different resources and work within the confine of those. If you look at my background, there he was with Lucier, sort of experimental music, electronic scene in the 70's, then in NY in the 80's, working with improvisers and downtown bands, then in Europe in the 90's, working with chamber ensembles, now in Chicago, in the boondocks, teaching at an arts school, and he's created this whole cult of workshop based hardware practice. Each one of those has provided its own benefits and limitations. But I think that's something that's ubiquitous to art practice. I don't think it has o do with a time... it's always been the case. I'm only saying this because in art school I'm very conscious of the fact that these days you have very few students who define themselves in terms of medium. Few people say \"I'm a video artist" \" I'm a sculptor"... they say \"I'm an artist". And then \"oh yeah I also do some video and some other stuff and I draw and I've done a print edition". The only people that define themselves by medium these days are painters. Painters still do that. And not all of them, but that's where you get the highest concentration of self-identified students within a medium. Next question! Let's try to get through all of them. 

ET: Is personalization of electronic music instruments just a set of decisions concerning which limitations are acceptable? 

NC: Yeah... I think people sometimes have different ways of defining what's an instrument. But from the classical era to the the rock and roll era people said \"I'm a violinist" or \" I'm a pianist". The conductor was the oddball, but people who made music defined themselves in terms of their instruments. What happens is now, post-electric guitar, the instrument has expanded. Is an electric guitar just a string and a neck? or does it have an amp ? How do you relate to the amp? Oh, you use a pedal. Is it a fuzz, an overdrive? Suddenly you're performing in this network of technologies. Then as I said with my Korean friends you disconnect the guitar, you only play the pedals... there's the transition. I think that it's more difficult now to localize yourself as an instrumentalist. Then you fall into this thing of realizing that an electric guitar is an incredibly versatile instrument. It's been used in so many styles of music. Nobody ever says \"oh my god not another electric guitar"... well they do, but not in the same way they say \"oh, a wahwah pedal...?" or \"a vocoder?". If you wall yourself a wah-wah pedal-er, it seems... much more limited. And maybe it's that preset thing again. It's got such a limited range, it doesn't cross that threshold of expressiveness. That being said you have people that have made a point of working within that. People like Toshi Nakamura and the no input mixing scene. Some of that is incredibly limited... or Sachiko M's early stuff with samplers, where she only played with sine waves, the test tone in her sampler. 

ET: Hannah Perner Wilson's MIT master's thesis discussed making basic components from scratch. She defines the advantages of such a practice as the opportunity for personalization, a better chance for transparency, and the importance of skill transfer. Any thoughts on this? 

NC: I'm doing an advance hacking course this year, so I'm in that right now. We're doing sophisticated designs, but also stuff like baking our own piezoelectric cristals, so we're going in both directions. We're doing stuff by making various parts with kitchen chemicals. It's very good in terms of understanding the material however usually what you make is not as good as what you can get commercially. So it's much more of a learning experience than something that makes sense from a product standpoint. If you haven't, you should read a book called the toaster project, by english design students who decided to build a toaster from scratch to see how the industry actually worked. Hammer pennies to draw wire and everything like that. It's an exercice in how the economy of scale works these days. You can do beautiful performance things where you draw and use the graphite as part of a circuit... I've had students who've done this, I have students who've done etching and used the scrapings of the burring on the metal as a conductor for a sound performance. In some cases it makes for a very beautiful performance medium, but I think with very exception, to build these as substitute for commercial ones doesn't make much sense. Build a wearable circuit because you want to interact with it, not because you want to put a wool resistor inside your Moog ladder filter. 

ET: Going back to the community aspect a little bit, who do you feel like Handmade Electronic Music has influenced the most: artists or engineers, academics or sef-learners? 

NC: Fortunately for my publisher, all of the above. I thought they were crazy to put the book out when they took it. But... because I didn't think any academic would buy it. and it was an academic press. But I knew that there was nothing on the market for this sort of grassroots community, and all the people who were asking me to do workshops could buy it. And then I could stop doing workshops - well, its fanned the flame of workshops, but I think it was very well timed because there was always a need for a practical guide for the community of builders. But the viral history of experimental music that's in it and all those sidebars - there's 150 artists referenced in that book. That made it incredibly attractive for academics. It was being used in music schools, in art schools, it seems to have had a really widespread impact than I thought it would. More power to the book. 

ET: Who is this book, and more generally devices like arduino and tinkering practices empowering the most? 

NC: I think the biggest change is for non-academics. That's the impact of the web. There's a very large base of people who do not need the base of the academic environment for their education. The web is for a lot of things bad, and its use as a formal educational tool is I think deeply flawed, but as an informal tool, its amazaing. When I was learning circuitry, you'd get a xerox of a xerox of xerox of a circuit that got from Tudor to your hands through 5 other people. Some stuff might have as well come from the soviet union. Now, even before people spoke of open source, the early web was about people giving away information for free. I think that was critical. I'd like to see statistics - I'm guessing more Arduinos are sold to freelance artists and tinkerers than to universities for the arts and technology... 

ET: Where do you see the successors of your book? 

NC: I've really been waiting for another book to come out. I'm working on my own other book. 

ET: you might be your own successor. 

NC: I hope not, I'm really dragging my heels on this one. There was I think the exploratorium came out with a book last year. I haven't seen it yet but it's supposed to be quite nice. Sort of stuff with conductive ink, pushing new materials. Make magazine comes in and out of the periphery of my percetion, but i don't know if they've gotten into anything major in terms of that market. They pay a lot of attention toward Arduino, Raspberry Pi, Beagleboard... That's where the main area of attention for books is right now. It's a more book-able subject, and it's one that does tie in a strong academic community. 

ET: so do you see this beeing more a book than online resources? 

NC: I think it's going to take both forms. The textbook market is still strong, whether its paper or an ebook. It's going to be book-ish. My next thing is going to be a cookbook. I wanted to flip the tables and contribute designs in multiple categories. Instead of an oscillator circuit, it'll be an oscillator circui \"by Ezra". Then you'll give reasons why you did it the way you did it, instead of just giving out stock designs. That'll be couple with analytical essays about the diy movement from a sociological standpoint, and interviews with or essays by major figures in the scene, to give it a little weight. 

ET: In that sense, how has a knowledge of avant-garde traditions oriented your hardware work? 

NC: In the sense that I could never get a job with a legitimate designer- All my stuff is directed towards these bizzare applications that I had, and my test... the buddy I'm staying with, the guitarist Robert Goss from Band of Susans... In exchange for staying at his house, I always bring him some circuit or I repair something. He was the one who steered me in more practical directions - so I've been doing this basically since we were in college. I'm hyper-specialized in all of my skills. My composition skills, my playing skills, my hardware skills, my computer skills... I'm not a generalist in any way. But I've been doing these things for long enough that I can squeeze them in a direction that serves a more general public. I will say that one of things that was really important for me in the book was making it ideologically neutral. I wanted to be able to have a techno producer and a circuit bender and a sculptor sitting side by side not feeling like anyone of them was being doctrined. Now, as you go through the book, and the workshop, the level of the sort of the notion of experimentalism as a neutral topic becomes very high. It's not a bout rogue procedure, it is abut finding your own path through it. the point is although you can see that as an avant garde trait, most people realize it is applicable in whatever field they are in. The techno producer needs to get a better drum sample that separates him out from the other people and maybe this contact mic is the solution. The sculptor knows nothing about music, but realizes that this malfunctioning circuit that buzzes works well with whatever she's working on at the moment. It's a non-threatening form of experimentalism. When you listen to the book's audio tracks, clearly they come from an experimental vein, not a piece of pop music on there. I'm sure most people don't listen to those. The videos are different, because they're so goofy, they're like having you're own youtube channel. Even if you don't like cats, you'll look at the little funny kitten video. Even if you don't like Davd Tudor, you'll watch a 40 second recording of Rainforest by a three year old with a camera in his hand. 

ET: Is there an engineer's music? Do engineers come to your talks? 

NC: I once had an engineer come to a workshop I did. He was the least competent person in the workshop, I was flabbergasted. When I knew he was an engineer, he said, \"but this is the first time I've ever touched an electronic component. I got a BSEE using CAD systems, I've designed a digital signal processor, it's the first time I've soldered". There used to be an engineer's music. In the day of synthesizers... Most of this is stuff you'd never hear. It was litteraly made by the engineers. In computer music in particular there are people that are much more technicians than composers, but it's a fine, fuzzier distinction. I think in a way you can look at the music that comes out of circuit bending as, if not engineer's music, luthier's music. If Bob Moog made a record, would that be an engineer's music or a luthier's music. 

ET: Do you think creating instruments and creating music are converging practices? 

NC: Well, yes! That's the thing... there's not so much written on it... but back in the 70s as I was saying, there's this real nervousness on the part of post-cagean composers about being treated seriously as composers. Most of them very much disliked the word improvisation. They used things like \"open music" or open form score.

ET: In the same sense that Cage disliked the term experimental? 

NC: Not quite, he disliked improvisation for other reasons. Those musicians didnt mind improvisation, but they wanted to be thought of as composers, not improvisers. My generation comes around and says \"who cares, we can do all this stuff. It's not so critical". But I was very aware that it was replaced by the question of are you an instrument maker or a composer, when you make a circuit? Tudor's thing was composing inside electronics. You build the circuit that is the piece. He was very adamant about that. But those of us who worked with him, we had our doubts. Did I build an instrument, or have I built a piece. That was critical. 

The same exists when you write a software patch. You can do things in max and supercollider that you think are a composition, then you eventually realize it's more like an instrument, I could give this to somebody else and they'd make a new composition with it. 

I have this one piece, Devil's Music. It existed as an array of cheap hardware samplers in the 80s... live radio sampling, very successful piece. I made a software version at the end of the 90s, and started distributing and having performances of the piece. With DJs doing it on their laptops, sometime with me sometimes without. Although it was really an instrument, it was so limited in terms of what it did that no matter who ran the piece, no matter what their aesthetic was when they decided to play it and pick their samples, it always sounded like Devil's music. You could always tell it was that composition. It was like wow, there I succeeded... but that was the exception to the rule. 

That idea of are you an instrument maker, or a composer, or a performer... And you didn't even the whole can of worms that is sound art. Let's not go there. 

Western Culture has these sort of strange distinctions: composer, musicians, interpreters, improvisers... there's a lot of culture when honestly it doesn't matter. You build the flute with the reed at the bank of the river because you're bored watching the sheep, and you play something based on what your grandmother used to sing. When I was at Wesleyan, there was a guy studying Gamelan, and he said \"you can listen to a piece of music from 1500 and a piece of gamelan music from 1950, and you can't tell the difference between the two of them." This was during the heyday of Cagieanisms. There's no idea of innovation there like we have. that's a different western thing. 

\section{transcript for interview with Dan Snazelle of Snazzy FX}
\section{email exchange transcript for John Bellona}

1 - your approach to hardware/software combinations seems really practical, but my sense is you still explore the capabilities of each framework or medium. Do you look up to any artist or engineer that might have done that particularly well as a template for your own work? 

When I first got into electronics, guitar pedals, digital music I loved engineer/producers. Dave Fridmann. Nigel Godrich. Tom Dowd.
  Later on, I was exposed to acousmatic composer Scott Wyatt, who applies a lot of engineering concepts to music that just make his music sounds really crisp and his trajectories are very clear. The sound is very important. This fact is essential for me to remember especially when having to consider gesture and technology in the same sphere. Its easy to undercut sound because you run out of time working on the other components, but sound is why I'm doing all the other things in the first place. So why undercut the reason I'm here?  While I am continually exposed to new works that I love (John Cage Four2 (SATB), David Behrman On The Other Ocean), I attempt to come back to the sound as helping to define the work. I will make decisions to support an idea or concept, but the resultant sound should always work with my ears. I hope this makes sense?

2 - one of the interesting challenges of teaching electronic music to me is the variety of interests and experiences within a given student body. How do you feel hardware and software facilitate that catering to individual personalities? Is one (hard or software) favored over the other at the academic level? 

Software is a pedagogical conundrum. Teaching concepts is essential to cutting across technologies, but then there is just the ability to work with software in general that requires the technical.  I've taught both ways, and with both, each cater.
Here are a two examples of success and failure (rambling)
 I've taught Photoshop as an in-depth technical course. (http://deecerecords.com/artd360/) I've seen previous instructors teach art concepts through the software, which didn't quite satisfy the entire student body (students from product design, architecture, painting, journalism, printmaking, sculpture...). I think it was important, for me, to ask why are students taking this course? And how is the course offered in the catalog?  By teaching the nitty gritty of how-to inside Photoshop, I witnessed non-technical students able to understand how to work with a digital technology and make work they were happy with, discovering solutions on their own. I also witnessed more tech-savvy student jump over the differences in Illustrator or other graphic/image software since they learned to understand what they were looking for. I was happy with the results in this, even though I was getting really into the how-to of a specific software.
	On the flip, I've taught Processing as an in-depth technical course (this listing is when I taught the lab that still included Flash (\url{http://deecerecords.com/artd252/#processing}), and I completely failed at it. While programming can be for everyone, the way it is taught will definitely cater to particular personalities. The lab I taught Processing in used to be Flash animation, but we slowly switched over to Processing only since the application of programming became more important than the industry need for Flash savvy designers (and Processing is free). For anyone who hasn't learned a programming language, and isn't into the idea of it, the challenge can be, at times, overwhelming. You don't need a lot of programming to do interesting work, but perhaps that was my fallacy in teaching the course that first time.  I thought people had to know array manipulation and Objects, but one doesn't.  This one should have been more exploratory and project-based.
	The software battles rage on...
	pure Data vs. Max/MSP.  
	Processing vs. Flash vs. openFramework vs. Arduino.  
	RTcmix vs. cSound vs SuperCollider.
	Logic vs. GarageBand.  ProTools vs. Logic.  GarageBand vs. Reaper.
	
	Do you have students buy a software? Or do you use open source?
	Do you teach how to use the software? Or do you ask larger questions that use the software to answer them?
	Do you use software that is for Mac or Windows? (I consider anyone using Linux as being independent and technical)

	Certain software hide concepts that are necessary to understanding both the analog and digital domains, and for this type of reason I stay away.  GarageBand is a huge no-no for me because the software doesn't allow students to understand how signal flow works.  Dynamics and Effects processing are applied in the same way in this software, but that's not the case in 95\% of other digital DAWs and 100\% of the analog world.  FX processing requires busses, and dynamics processing is placed in line. Is the software free? yes.  Is it already installed on every Mac and on every lab machine? yes.  Should we use it?  No... 
	People will also teach and use what they know. I've seen undergrads taking a RTcmix course, and I've heard gripes about learning an outdated software. 
	I think any software is fine, so long as you either address the needs of the students or are extremely clear about the course will be about (conceptual vs. technical, etc.)

3 - Do you feel like academia / institutionalized art is representative of the overall community (or even just your community) that operates between art and engineering? 

Do you feel like academia / institutionalized art is representative of the overall community (or even just your community) that operates between art and engineering? 
	I've found that collaboration is a necessity if we want to elevate our capacity to work with multiple disciplines (art, music, engineering, architecture, dance, to name a few) and to create exciting work. We almost no longer can do it all on our own because of the knowledge required.  If we can prepare students to ask questions, reach out to others, and establish collaborative relationships, everyone will benefit.  This type of learning also extends outside of the classroom into real-world problem solving, and helps situate the necessity of art in the eyes of an administration who may perceive music as only being an acoustic instrument one plays to become cultured in Classical/Art. (yes, capitals here) Electronic music has similarities with today's community because of the artistic and interdisciplinary process that is involved even if the the end results may sound much different than a listener's expectation from a piece. 

4 - what's your community like? What's the place of sharing at the local / international level there? 

I've really enjoyed the Kyma community, which is a small niche in the larger world of electronic music. I enjoy the community because it's comprised of academics, sound designers (film and video games), musicians, composers, video artists, and more. Everyone is very humble and open to exploration, collaboration, and dialogue.  While I completely understand that the hardware locks out many practitioners, the community contains many individuals with whom I want to have artistic conversations with.

With regard to academia, I see my work and process paralleling that of art and digital music departments than traditional composition. This probably has to do more with the ideas of exploration, collaboration, questioning, and openness. I've also learned quite a bit from working with Harmonic Laboratory, the art collective I am part of. Working between disciplines informs how I approach a project, the connections I see and want to explore, the type of vocabulary I use, and the work I create. Collaborating with Harmonic Lab has allowed a work to take shape rather than to oppressively mold the container.

\section{email exchange transcript for Martin Howse}

Could you start by describing a little bit your approach to installation / performance hardware? How did it come to be? What's the place of physical items in your work today?  
Is your interdisciplinary practice (art, technology, teaching) means to achieving a unique goal, or different ways to express thoughts and interests?  
What level of complexity do you look for in your electronic instruments? 
how do you approach the use of new devices in your setup? 
how have proprietary designs, tools, and planned obsolescence affected your work? 
What have the responses to the detektor and dark interpreter devices been? 
they are both hardware/software combinations. is that a practical decision? 
how important are other practitioners in the development of that work? 
how has teaching affected your creative interests? 
in your experience, who does open source / affordable hardware empower the most in the arts? 
Is there any recent technical development you're particularly excited about? 
what is your current professional community? Do you feel close to it? 
How close do you feel to avant-garde or experimental music traditions? 
How has your awareness of those traditions influenced your hardware or text-based work?

\section{email exchange transcript for Jessica Rylan}

could you start with a brief description of your hardware and music backgrounds, what you are doing today and the important steps in getting to that point? 
what is the current place of hardware design in your practice? 
what parallels do you draw between technical processes and other disciplines or hobbies? 
In what way does scientific research trickle down to influencing your electronics and the music you make with them? Would you say there are parallels between the processes? 
Are you familiar with Hannah Perner Wilson’s kit of no parts? As part of her Master’s Thesis at MIT, she devised this set of objects that could be built using some homemade elements - conductive ink to turn seashells into speakers, conductive thread to make electronics in wearables, etc. Would you consider using these methods for electronic music? 
The personal synth seems to be a good example of what can happen when the artist/engineer personas merge: unique \"machinstruments\". You've described it as offering unexpected possibilities - a device with which you have an emotional relationship. To what extent do you feel like other electronic musician make a setup their own? 
What would electronic music be life if every one of them had their version of the personal synth? 
When designing audio hardware, how do you approach challenges and limitations?
How have proprietary designs / tools and planned obsolescence affected your practice?
You've questioned the scientific approach to sound in favor of embracing a more chaotic, unpredictable method. How do you reconcile that with the limitations of analog electronics, and how does your current music setup reflect that? 
How important have other people been in the technical development of your practice? 
You've described your interest in electronics as coming from a very personal place, from your grandfather and popular electronics found in your old house. How has that personal connection evolved and influenced your design work? Has anyone taken the place of those sources of inspiration? 
What is your current professional community like ? 
To what extent do you identify with a tradition of use, re-use, misuse or subversion of technology for the arts? 
Do you feel compelled to help people with the same design questions you had when you started? 
How has musical training influenced your instruments and musical practice? 
Do you perform with anything other than your own instruments? 
How fluid is the transition from circuit design work to music?  

\section{email exchange transcript for Bonnie Jones}

Could you start by describing a little bit your delay / mic setup for performance / composition? How did it come to be, and how is it important in your work today? 
what parallels do you draw between technical processes (like assembling that setup) and other disciplines or hobbies? 
What level of complexity and indeterminacy do you look for in your electronic instruments? 
how do you approach the use of new devices in your setup? 
how have proprietary designs, tools, and planned obsolescence affected your work? 
how important are other practitioners in the development of that work? 
how has teaching audio electronics affected your creative interests? 
in your experience, who does open source / affordable hardware empower the most in the arts? 
Is there any recent technical development you're particularly excited about? 
what is your current professional community? Do you feel close to it? 
do you identify with hacking / bending / re-use / misuse of technology for the arts? 
How close do you feel to avant-garde or experimental music traditions? 
How has your awareness of those traditions influenced your hardware or text-based work? 

\section{email exchange transcript for Louise and Ben Hinz of Dwarfcraft and Devi Ever FX}

what are your musical interests and training? 
do you see any parallels between designing a circuit and playing guitar?
both of the companies you're involved with are popular in the pedal market. What made you go from basement experimenters to a business, and how did that influence your circuit design work?
the Pitch Grinder is your first commercially available digital product (unless I missed something in the Devi website / history). How was that foray into software / hardware combinations, and what prompted you to do that? 
how traditionalist is your customer base? Does that conflict with your musical interests? 
Dan Snazelle was mentioning that Eurorack has become more popular than when he started Snazzy FX. How did you come to start selling modules, and was it a significant shift from pedals? 
modules also put you one step closer to experimental / avant garde / contemporary classical music. Any opinions? 
how do you approach limitations in hardware design? 
have proprietary tools / designs and planned obsolescence influenced your work? 
Conversely, have you shared any of your designs / tricks / magic? 
how important have other people been in the technical aspect of your work? 
do you have any opinions or thoughts on your professional community? 
have you helped beginners with design question like Devi did for you? 
are there any philosophies or beliefs that seem important to you as you do this work? 
