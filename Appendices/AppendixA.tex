% Appendix A

\chapter{Interview Transcripts}\label{app:inttrans} % For referencing this appendix elsewhere, use \ref{AppendixA}

\lhead{Appendix A. \emph{Interview Transcripts}} % This is for the header on each page - perhaps a shortened title

These are the full transcripts of the interviews undertaken as part of this thesis.

\section{transcript for interview with Sang Wook Sunny Nam}

\textbf{Ezra Teboul}: How did you first get interested in audio technology? What was its part in your learning process as a mastering engineer?

\textbf{Sangwook Nam}: I went to mastering in 2000. There were no resources to learn mastering, but there were books and photos about studios in the US - I was in Korea - and there were tools that known as applicable to mastering. So I used these, mostly digital items like compressors. Plugins were just born by then, so we didn't use them. So we had digital hardware, like waves from swissland (?), and more things from Germany.

After I went to the west coast at the Mastering Lab, they had a completely different concept of the gear they used. I had to learn everything all over again. They're still quite different from regular studios, in those they use mostly off-the-shelf gear. But I had to learn the history of all the recording equipment, because a lot of our designs were based on work from the 60's. I learned how they worked, how they were designed, how to fix and improve them. They're in a smaller form factor now.

\textbf{Ezra Teboul}: Do you hold the distinction between discrete transistor circuits and IC circuits as being important in this context?

\textbf{Sangwook Nam}: Yes - discrete, tubes... transformers, resistors... every part is important. They've all changed, become smaller. The overall quality of the parts and the circuits they're in have been in some sense compromised. I learned how the new technologies, how those smaller form factors damaged the sound of the equipment, and how to keep away from them, all that sort of stuff. What parts I need, how to evaluate them, where to source them...

\textbf{Ezra Teboul}: So you've listened to ceramic vs. tantalum capacitors...

\textbf{Sangwook Nam}: Right. That's easy. But polyester vs polystyrene? What type of polyester? What type of structure? Even with the same structure, what company makes a better polysterene or polyester condenser?

\textbf{Ezra Teboul}: Have you done double blind A-B test for all of those variants?

\textbf{Sangwook Nam}: Yes.

\textbf{Ezra Teboul}: When did you do those tests?

I didn't have any need to do those while working at the mastering lab because my mentor had already done all the listening tests. Wires, switches, volumes, everything. When I left the company, I did it all again myself, because he didn't tell me all of his results. I had seen a few, I knew some of them, but most of them aren't available in the market. In those cases, I had to find old stock from somewhere, and if I didn't find those, I had to find an alternative. So I go on Ebay to look for old parts - from american to russian military to european parts... I went through, bought samples, listened to everything...

\textbf{Ezra Teboul}: I\'ve had to do similar work in the past, and it always seemed like a fun aspect of projects.

\textbf{Sangwook Nam}: It's a fun part, if you're a student. But if you're spending money and time, it becomes more of a problem...

\textbf{Ezra Teboul}: What do you think of your current setup? Is there still work to be done, or are you happy with it?

\textbf{Sangwook Nam}: I'm fairly happy with it. I have some equipment in my mind I'd like to build, still... but my clients are very happy with the outcomes, and I have quite a bit of freedom to do what I need with it, so I'm pretty happy with it.

\textbf{Ezra Teboul}: The fabrication of your custom equipment involves contractors and engineers - most of it isn't built by you. How much do you document this collaborative process?

\textbf{Sangwook Nam}: Well they submit all the schematics and that sort of stuff to me, so I have the final result and a sketch of the design. I also keep all the email correspondence, some notes I take when I evaluate the products...

\textbf{Ezra Teboul}: Do you share any of this documentation?

\textbf{Sangwook Nam}: Oh no. It's a very time and money consuming process. Very good polystyrene condensers cost 200 piece. If I'm testing a stereo pair, I'm spending 400 just for those. Unfortunately, they didn't sound good - so I end up spending 5 to 7 grand just on capacitors. So anybody can do it, but, you know - it's my money and time, and it's really hard for me to share that for free.

\textbf{Ezra Teboul}: Do any of the people you contract for this equipment end up sharing the work they've done through collaborating with you?

\textbf{Sangwook Nam}: The designer and builder of my equipment (Josh Florian - JCF audio) is my close friend, and he's pretty secretive of the information he gets from me. He doesn't use it in his own work or other commissions. Also, its very expensive for mass product, those parts are hard to get...

\textbf{Ezra Teboul}: You've mentioned that learning the equipment was a big part of your education. I'm guessing the other part was training your brain listen to what you want to do, what you need to do with a recording when it comes in your hands. For most projects, what's the technical vs. the mental?

\textbf{Sangwook Nam}: Psychological understanding of what the gear does... will take weeks, months, years. These pieces of equipment I've been using are quite different from modern EQs, compressor... if you've learned how to use commercial standards, you'll have to learn to adapt to mine. I'm using shelf eqs, so I don't have any Q control, any peak curves.

\textbf{Ezra Teboul}: Is it an active circuit?

\textbf{Sangwook Nam}: No, it's all passive. There's only one amplifier at the end. With the parametric EQ most people are used to, it's really easy to go to the frequency you want and take it out, or add something... but you have to be very creative to make something peak- like with a shelf eq. Also the bands are very limited. You need 2 shelf bands to approximate the behavior of one band of parametric EQ, and I have four shelf bands. So I have a very specific strategy to play with the balance of sounds. Also, because those are very wide EQs, you have to get used to that. It's like using nothing but primary colors to paint an image. If you have 48 colors, you just pick up the right one and paint with it, but this is completely different. So I had to learn to adapt my process to these mechanisms, which are very simple but very hard to use for some complex behaviors that more standard equipment can do easily.

\textbf{Ezra Teboul}: Does having a say in the way those items are designed probably helps make it easier for you to use them?

\textbf{Sangwook Nam}: Yes, that's true. Also, certain EQs have some things that they can do better than others. There's two types of filters: LC and RC. LC means you have to use an inductor. If use that, there's a resonance, so you can get a little bit of a peak curve out of it. Also there's the specific sound of the inductor, which will color the output - it might sound a bit more aggressive, or euphonic... so you have to know what this EQ would sound like on a particular sound.

\textbf{Ezra Teboul}: What order filter are you using for your mastering work?

\textbf{Sangwook Nam}: Because it's all passive, I'm using all 6db/octave first order filters. If you want to do 2nd order, you get more parts: more distortion, more non-linearities, more noise. 6dB/oct is more than enough for most cases.

Most EQ slopes in DAWs can get much higher very fast, with 24 or 48 dB per octave being common.

Yes, 24, 48, butterworth... those are common. If I have to do anything that drastic, I can do it in the digital domain, or I can ask for it to be fixed at the mixing stage. If you want to do that at the mastering level, it's a fairly significant problem, so you don't want to do that... That's what's less harmful.

\textbf{Ezra Teboul}: You've worked with Josh as your go to engineer?

\textbf{Sangwook Nam}: He was assistant engineer when he started at the mastering lab. He found at that his interest was in electronics rather than recordings. He's also a great drummer. So he learned a lot from our resident tech back there, what he calls ``yestertech''... tube electronics, discrete, power supply, grounding schemes, all sort of stuff. Then he developed all the new stuff out of that and became the owner of his company that makes great products...

\textbf{Ezra Teboul}: Do you know any other hardware engineers you'd put on the same level?

\textbf{Sangwook Nam}: One of the resident techs went independent thirteen years ago. He's not making any mass-marketed products, but he'll do commissions. His name is Steve Hazleton, in Tennessee. He's another guy that I can go to if Josh is not available. The other people... people who make mass products for consumers had to deal with all the safety issues and regulations that can be problematic for some people. They put a lot of stuff in their designs that ultimately compromise the quality of their work... so I wouldn't call them.

\textbf{Ezra Teboul}: You made it seem like there was a back and forth between you and the designer to get these electronics together, starting with your query for a specific item or modification. Does this process go both ways, and do you listen to some of their recommendations for mastering equipment?

\textbf{Sangwook Nam}: Yes. The equipment that I have... When Josh first listened to what I wanted, the equipment I needed and the topologies I wanted to use, we had to improvise a little bit, which I wanted to do. There's definitely an interplay. He developed a couple of things I didn't think of, and sometimes I'll suggest something he had not considered. That's always going on.

\textbf{Ezra Teboul}: Is there one specific item that illustrates this process well in your studio?

\textbf{Sangwook Nam}: The monitor panel is all custom made. I wanted to listen to what's coming in, and what's coming out. What's coming from the DAW, and compare everything. I told him I wanted those things, so he'll develop a schematic, and explain how each volume is controlled, how to implement mono... how to make those three inputs independent, how to make the panel that has the less contacts. The more connections, the more contacts. Even though we're using very good switches, you'll lose some details. That sort of thing can be more specific points developed by him... ultimately, we decided on three contacts and he built it.

\textbf{Sangwook Nam}: The compressor is another example. I wanted a low pass filter on the side chain so it doesn't see the big bottom end when it compresses it. Since we were using 4 channel switchers, he said we could make it a variable filter, so I decided on four values. 100 Hz, 200 Hz, something else, 400 Hz... That sort of process.

\textbf{Ezra Teboul}: You use only switches and resistor networks, no potentiometers, no faders?

\textbf{Sangwook Nam}: Yes. No faders. Only switches. You can also use relays, or cheaper switches, or any VCA type OPM for switches...

\textbf{Ezra Teboul}: Do you do that?

\textbf{Sangwook Nam}: No, but it's a possibility. But all these alternatives make you lose a lot of low-level detail, so I avoid it. Also the materials of the contact, and their structure... the one I'm using has two contacts. It's military grade silver contacts - gold has a specific sound that's not very useful. So does the copper... Everything has a different sound, and silver is my favorite. It's an expensive switch. It's double sided, so a one pole, twelve out... just one wafer is 60-70... and you have to order thousands of them. The price is just...

\textbf{Sangwook Nam}: Those types of switches were used solely on my console. Volume controls are also only resistive networks.

\textbf{Ezra Teboul}: Do you try to minimize the amount of wiring in the overall studio?

\textbf{Sangwook Nam}: Yes. All hand wired, short. There's still a lot...

\textbf{Ezra Teboul}: Is it all point to point soldering?

\textbf{Sangwook Nam}: Yeah.

\textbf{Ezra Teboul}: So no ceramic boards or circuit boards?

\textbf{Sangwook Nam}: Well, some of the equipment, like EQs... is built with circuit boards. But the console is entirely point to point.

\textbf{Ezra Teboul}: Do you ever take a look at the inside?

\textbf{Sangwook Nam}: Yeah, every three months I'll open it and clean all the contacts.

\textbf{Ezra Teboul}: Do you ever appreciate that wiring as a work of art in itself?

\textbf{Sangwook Nam}: Yes. Fortunately I bought the remnants of A\&M mastering... the founders of that studio were working at the mastering lab and built the console for A\&M studios... So they had all the parts I wanted. And A\&M's philosophy was really similar to the mastering lab's so I was very happy to get that. I had all the good switches, and an already made point to point console. We had to adapt a couple of things, but most of the work was already done. Josh had to do a lot of wiring, but still...

Wiring all those switches... I can't do it. Somehow Josh can do really fast.

\textbf{Ezra Teboul}: Would you ever take out a soldering iron and fix something yourself if it needed it?

\textbf{Sangwook Nam}: No. I'm not good at the smaller things.

\textbf{Ezra Teboul}: How do you know if/when a particular piece of gear is finished? (25min52sec)

\textbf{Sangwook Nam}: We'll discuss the topology. For amplifiers, for example, I usually avoid op-amp designs. Even with discrete op-amp based circuits like the 2520 or 990, those use a lot of feedback, and I don't like the sound of either. So we were talking about discrete amps, like a quad or two push-pulls topology? The shelf EQ has a loss of around 21 dB, so will those circuits have enough gain? He'll build a prototype and we can try it and listen to it. He'll make a test board that has a 21dB loss to test it with, I'll listen to it for a few days, send it back with comments...

After he is done the designs, I listen to for a week or two for anything active. Sometimes he makes a mistake and process stretches a bit, but usually it's just one trip and... done.

At one point, in the very first stage, he sent around 4 different amplifiers with variations for me to try, so I picked my favorite and refined that design. I don't want to overuse one amplifier, because even though they're very transparent, it still has its own sound. I don't want to add the same topology multiple times, otherwise those characteristics compound. So if we need another amplifier, say in the compressor, I'll use a different topology - maybe tubes, or solid state topology.

\textbf{Ezra Teboul}: To what extent do you participate in online audio-enthusiast communities and what role do they play in selecting your equipment, if any?

\textbf{Sangwook Nam}: Human ears can get off-road really quickly. Ears work as comparators. You'll listen to one thing, then another thing, and always compare. If you hear a very bright tune, then a well balanced tune, the balanced one will seem dull. It's very easy to loose objectivity in listening. I had to find a few people that know how to listen to low-levels, to electronics... This is what I usually say: listen to the quality, instead of the quantity. One dB of EQ is the same if you look at the quantity, but every unit has a different sound. People have to pay attention to that, and I haven't found many that do, so I always ask them for a second opinion if I need one. One of them is Bill Schnay, in L.A., and he has a very high resolution recording studio. I listened to capacitors there to pick the ones that sounded the best: we tracked drums in the live room straight into the switch with a very high resolution microphone, and I'll listen to a capacitor with the lowest capacitance to get a sense of what it'll do. He was sitting next to me and helping me, making sure we remained objective. Having a second opinion is really important.

But, online... no. I go there, to find out what the next product is and what people think about it. But the ultimate decision is always with your ear. The people who are lurking... active on those websites... if you're busy, you don't have time to do that. What they're saying is so not true...

\textbf{Ezra Teboul}: Do you measure the performance of your gear electrically, and if yes, what importance do those measurements have?

\textbf{Sangwook Nam}: I do. Good sounding gear has to measure well, but the opposite is not true. Like the 990, it measures well but I don't like the sound. When Jensen first made it, we tried it at the mastering lab, we had a very big live sound at MGM - now it's Sony, back then it was MGM - we put fast instruments like tambourine or any percussion 40 ft. away from the microphone, and rung the bell, or the tambourine. The tails of the sounds decrease, the microphones record that, then we listen to that through high resolution amplifiers that we know with the same gain and it what it does is \"ding\" then the decay is going on. We can still hear decay all the way down on a good amplifier. The 990 goes \"wiuuuu- ss!...\" because of the internal feedback. For measurements' sake, you cut out the low level detail, and they think it's noise, and the feedback system cuts it out. But there's still information in that. So measurements isn't a guarantee, but you still need very low distortion measurements to sound good.

\textbf{Ezra Teboul}: Do you look mostly at THD (total harmonic distortion) measurements for this?

\textbf{Sangwook Nam}: The problem with THD and some of those measurements is that they're done using sine waves. What's the easiest thing to measure? Sine waves. What do sine waves have to do with music? Not much. That's another problem of measurement. This can be a good measure of what the amplifier does, but it can't be all of it.
You have favorite topologies of discrete transistor circuits and tube circuits. Do you want to talk about that a bit more?
There's not too much... I like less parts. If I can do with two push pull transistors or tube, I'll use that. If I need more gain, I'll need a stronger design... Tube can be very euphonic, with low distortion, but mostly so at low wattage. Their problem is when they need more power. But if you think about the music, everything is a few watts at most. Drums, or fortissimo would draw a lot of current and require more wattage. But most of the time you're drawing 1, 2, 3, 4 watts. So tubes are great for listening, but not necessarily when you're mixing or mastering those strong transients.
Have you ever considered an hybrid, adaptive amplifier that distributes the load between a single ended vacuum tube amp and a push-pull transistor amp based on the dynamic range of a piece?
I'm not a real designer, so I don't know what that involves. You'd probably need a lot of interfacing between the two sides? I don't know... I have used a tube push-pull amp, and it was really good.

\textbf{Ezra Teboul}: What is most important in your work process?

\textbf{Sangwook Nam}: At the end all that matters is the sound you're getting. It's hard to be objective if you know certain things about your equipment. Forget what it is, listen to what it does and what it can do, where the limits are. It's very bad to have all this technical knowledge without having the listening abilities. I can see a lot of people with that problem, especially on websites. They understand how things work, what's new, but they don't listen.

A funny thing: one guy asked what the best DAW was, and a thousand replies were added on that thread. \"This DAW has this function\", \"that one has another\", but over those thousand replies, no one talked about their sound. That's the current situation of the engineers, and it's really bad. They read the articles and they know the technology... but no listening.

Dither - lots of guys talk about dither. Lots of dithers available. But in my ear, most times dither doesn't work well. For example, if you're using four plugins on one stream, all of those are going to add dithers. If they're using high frequency-boosted dither, even if it's inaudible in one plugin, if you have for or five you can hear that sound and it affects the music. So I tell them to turn of the dither... and they say \"oh no, truncation of errors... no...\". All this information about technology is influencing the way people hear music, and the new technology is always better... In some areas, that might be true, but in music it's not always true. In my opinion, and the history of music says that too, 50-50 is very generous numbers for new technologies. I'd say 20-80. I'd wait 2-3 years after something came out, and trying it in my studio before having an opinion on a particular item. OS, or Pro Tools updates... all sorts of stuff.

\textbf{Ezra Teboul}: In the end, what makes your studio special? You've spent a lot of time listening to everything, and that's why people hire you?

\textbf{Sangwook Nam}: Mastering is special, because in a recording studio you have 48 or so channels to control your quality. If there's a bit of loss in one channel, the consequences aren't always serious. In my studio, my hands are tied to two channels. If I don't use those two channels better than other mastering engineers, I have a big problem to start with.

After that, even if you have a great equalizer, if you play the music wrong, you're doomed. The first thing I can say than my studio can do better than any other studio is play two channels. My DA converter uses a very specific topology different from any other mastering business. It doesn't have any digital filters. You need digital filters to do oversampling - I don't do oversampling in my deck. I don't have digital filters. I have a special process that fixes oversampling and digital filtering that was done during the recording.

So that's the most important part for me. How to play two channels right. In a musical way.

\textbf{Ezra Teboul}: Do you know anyone who has built their own mastering studio from scratch?

\textbf{Sangwook Nam}: All this? Yes, Bernie Grunman. A\&M... which is gone now... the Mastering Lab. So Bernie, Mastering Lab and me would be the three... All the others... I don't know anybody else.

\textbf{Ezra Teboul}: No one who wasn't a professional in the first place is doing this from scratch.

\textbf{Sangwook Nam}: Right. Also, you need to have a tech to build all this, understand the topologies and how to use them.

\textbf{Ezra Teboul}: That's too much for one person?

\textbf{Sangwook Nam}: And too expensive. All the young engineers who build the gear don't have deeper knowledge of yestertech. Everything I see today has remnants of a 2520 or a 990. Some people work with the quad-type amp. But those three represent most of the designs. So it's a pretty limited knowledge base. Now if you have the resources to look for more original things, you have a lot more to work with. The encyclopedia of audio - the first edition, from the 50's. That's the real treasure.

\section{transcript for interview with Tristan Shone of Author and Punisher}

TS: I've gone full circle: engineering, punk bands, metal bands, back to art school, away from engineering, did some rough and tumble sculpture that was not very satisfying, then I came back into music because it's much more natural for me and I've become a musician at heart. Back to doing engineering to make the money and balance off the touring. Everything's just kind of a big clusterfuck right now, trying to manage a musical endeavor, with trying to get back to really doing instrument design.So much of my life is taken up by booking and promotion, running a business as a band but also balance new design and art grants... there's a lot of logistical crap that seems to take most of my time. Which is maybe not as interesting for your thesis. 

ET: can you develop your background a little bit more?

TS: I don't really have any composition experience, other than I'm a trained piano player, and I learned to play guitar, so I could play in metal bands. That's where I came out of college, interested in Robotics and control systems. That's what I did as an undergrad at RPI in Troy NY. I was playing in thrash metal and apocalyptic doom bands from the mid / late 90s, neurosis, melvins, godflesh... also some drum and bass, electronics... while being in my classes, and working on stuff like electric cars, at RPI, I was helping with that, and assisting some professors and learning mechatronics... I think I realized I didn't want to go to grad school for engineering because... I really liked the gadgets and I liked mechanical engineering, and I loved theory of control systems and robotics but it was a bit too much. Not as interesting as physically making things work. I liked machining and fabrication, so I went and got a job in a clean room. This was during the telecom boom, around 2000. You could make a good salary out of school. So I said screw it and went to design these automated systems for testing MEMS - micro electro mechanical systems. Semiconductor based machines. Other people would etch these out of silicon, and craft these little machine that would basically switch fiber optic lights. I would make these setups with x-y-z mechanisms that would test them... It was a bit dry. Being in the clean room after being in a college band and touring... College life to working in a clean room with really high level scientists was not really my thing. I did a few years of it, switching around to a few different companies, and playing in a band in town that never really went anywhere but also... I found some companies I liked to work for. There was also a professor at RPI I worked for, an art professor called Chris Csikszentmihályi. He's a media artist, he was at MIT's media lab, and I sort of helped him while he was there designin some parts for his installations, travelling with him a couple of times. I went to Finland, met a lot of people in the media art world, kind of introduced me to what you could do with sculpture and mechanical sculpture, microcontrollers before the arduino... They're easy to use now, you don't have to compile 70 files and set environmental variables on your computer... we were using those... devices that he had developed while at MIT. That's most of my engineering before grad school. Going back was as much a decision for my carreer, wanting to work with art, as it was wanting to leave corporate america and having nothing to do with it. 

ET: where did you go for your art program? 

TS: I went to UCSD. Which is also where that guy Chris Csikszentmihályi went, one of the guys in the Yes Men was there. Who else was there... Barbara Kruger, Jean Pierre Gorin... Lev Manovich... Good professors, and a very tech oriented school. That's where I work today, in the neuroscience department. We do imaging microscopy, and I work on all the automation, with a bunch of biologists and physicists who come up with ways of doing things. 

ET: were you familiar with all of RPI's experimental music aspect? 

TS: Yeah, I was somewhat involved in that. That professor I worked for was in that... I think that the music department was pretty small at that time. Then they built that giant building, EMPAC. I'm in touch with some people there, we tried to get a performance there, and it never happened... 

ET: Can you talk a little about the connection between metalworking and metal music? There's something very physical about both processes...

TS: At the beginning, when I was building... there's different angles. I started building speakers in grad school with a friend of mine who was coming from rave culture. I was playing metal and buying these kind of guitar center amps, heads, speakers... you buy this keyboard, then this stupid little piece of shit... clamp-on thing for your keyboard stand... thin wall metal that's been welded, and it breaks... I just got to this point where I realized that all... anything you buy at guitar center or musician's friend is a piece of shit. Any pedal you'd buy was a piece of junk. I started to replace... just simple things, like if I needed a sustain pedal, I would go to an army navy store and just buy one that actually had gears in it you could oil, instead of having a plastic rack and pinion gear. That was the main component, that was the weakes link. Everything like that, I decided that I would just make myself. I had three year of time where I was just going to be in the welding shop. So I got rid of my Mesa Boogie Dual Rectifier head, cause its got that tread plate face on, but to me the aluminum tread plate was just fake-strong. That's the way I looked at it. It was basically just a way for metal dudes, who drive big trucks that are actually plastic to look tough. I said oh, I can actually apply the research lab mentality and make stuff out of real materials. It just so happens that I'm into metal and we'd listen to drum and bass because the guy I was working with was a big dubstep and drum and bass guy, in 2004. There was just something about making things... none of that steampunk mentality, where you would make something with the appearance that it was something that it wasn't, and also not adding... A lot of people when they look at my machines they say ooooh, you just did that for looks. No, you can ask me about any component on what I did and why I did it. It all has function. That, to me, getting back to metal and heavy music actually really is. I also think that there's an issue with some of this stuff... you still have to write songs, and write music, and there's a point in time where you just have to... I started getting ahead of myself, building too much stuff and not actually learning how to play them properly and really having a connection with the materials afterwards. That's where I'm at now, just trying to compose. 

ET: An interesting thing you've mentioned is the variety of technologies you've delt with: vacuum tube amps, solid state microcontrollers, your laptop as main synthesis engine... Is it about being practical for you? 

TS: that's a point of contention. In the art world, the tendency is towards analog electronics, because people want to make something that makes sound, the thing that you're moving is rubbing on something else... you plug it in, and it doesn't need a computer. For a lot of people, that's purity. I think that's why people like modular synthesizers. And for me, I agree. But at the same time, the most important thing for me, when I was conceptualizing these instruments, was that I was making something that felt right, that was the one to one connection with the sound. What does this bass sound sound like, how does it feel in your hand. With the small experiments with analog circuitry I did, making some of those circuits myself, I just wasn't getting what I wanted. I really wanted something... the sounds coming off of my laptop, I can sample something, tweak it...It's the difference for me between some of the metal guitar distortions that you can buy or the electronic bass tones that you can get that I was much more interested in. They hit you in the chest much harder. That was something I could do on the laptop, and I could make one interface that could control any sound I wanted. So I gave up on the analog purity pretty early on. 

ET: The result is what matters. 

TS: Yeah. I have seen... most of the time when I see people do stuff with modular synths now, it's a lot of bleep blops. It doesn't necessarily have the... there's a lot of experimentation while you're coming up with sounds, and that fine, it is what it is. 

ET: Last week I interviewed Dan Snazelle from Snazzy FX, which is a company that sells eurorack modules. One of their products is the ardcore, which is a multipurpose module with an arduino in it. He worked with Darwin Grosse from Cycling '74 to design the best way to use a microcontroller platform in a module, with a USB connector on the front panel for people to upload their sketches. They have a small online community to share the various uses and implementations people come up with. That seems to be where the two niches meet? 

TS: Yeah... 

ET: For a genre as irreverent as noise and metal, there seems to be not that much experimentation with the medium. 

TS: Being involved in this kind of industrial world... although in the last few years I have been getting into more of a music industry than I have in the past, working with musically inclined people at theaters and clubs, you have to trim down your setup, be on stage the second out of four bands. My desire is to have everything giant, welded, setup that's as strong as it could possibly be. I've had to make sacrifices to be able to fold something up. So I have to buy some of that crap every once in a while, and that pisses me off. But for something like industrial, the genre is a total farce. I've played some of these festivals. it's dudes in leather, that isn't even real leather, and they're playing instruments that are just the most plastic thing you've ever seen and it's all premade beats, and they're mostly pressing play. There's nothing industrial about that at all. There's nothing primal about it. And maybe we just need to redefine a new genre for something that actually takes effort and is composed live and has bring some emotion back in it. And I do see some people doing this now in the noise genre... but anyway. I'm sorry, I only talk shit about the industrial genre because it's so new to me, I was never a fan of it, now to be a little bit involved in it, it's  a very frustrating world. It's so fashion based... so much drama, just seems so far from what down to earth music is for me. Being involved in the music industry is very frustrating. With respect to the machines: I'm really interested in textures, and materials. I have so many ideas for new instruments, but they're so expensive and I'm recording and traveling... hopefully this summer I'll be able to work on a few. But really in my mind there's just very simple things: a shaft of really shiny hardened bearing steel, maybe a piece of brass self lubricating as it slides on there... I want to be able to feel the interaction between those two materials. Being able to make that connection between those two things is all I want from an instrument. But then you know, of course, I've got to encode the motion, and fit it inside a case, and make it light enough to take overseas. So you start with this interaction of materials for a motion and it expands out into whats possible. That can be a big thing: traveling, having plugs... 

One of the biggest problems for me recently is reliability. USB cables are a big problem. Wireless connections, or using industrial connectors helps make sure they withstand the test of time. 

ET: Have you considered some wireless arduino platforms? 

TS: I've done some wireless shield, I guess I haven't done bluetooth...

ET: One thing that I thought about when you were talking about industrial music is noise music communities. One example is the community described in Japanoise, perhaps it complements what you saw in that community? They take all cheaper pieces of gear, blow it out, misuse it... There's a little bit of the original spirit of the noise and industrial genres in that... do you feel closer to them than to the festival industrial crowd? 

TS: You mean hackers ? 

ET: by extension, I think. Hackers, circuit benders. People who try to go beyond what is bought. 

TS: I definitely don't feel like a hacker. I definitely appreciate the risk that they put into what they do... not everyone's going to design their own shit. I can't expect everyone to fabricate their own stuff. That's when hacking comes around, you don't necessarily have to know electronics, you just need to be creative. I really appreciate that. There's also so much risk! You're making a one of kind object that's going to be up on stage as part of a festival, and you don't have a backup. No nice rack in the back where your tech can plug the duplicate in if the other one fail. 

There's a lot of risk involved, I don't feel like a hacker. I think the scene I relato to the most is this metal/stoner/doom, jut on a note of that, there's a real sense of respect of quality, craft amps made with baltic birch, better speaker  drivers, discussion on how to get the best tube amp, and they understand how it works. I just know people in Portland designing their own pedals, really crafting this fine. There's this simplicity in the design (27:34). I wouldn't say I want to go back into that world of guitar, but everything's played live, there's improvisation... That's where I'm coming from. The controllerism, sequencing, grids in Ableton... I have a few sequences I play every once in a while but that world, there's not a lot of room for failure. It's all synced up, you play combinations... doesn't have the motion or the power some of the live rock stuff has. 

When I think about what I'm doing, people think oh it's robotic, futuristic. To me its not futuristic, it's very organic. I'm basically trying to be a one man band that plays live music. I'm trying to play electronic music efficiently, live, without sequences. That's the whole reason I built this stuff. It wasn't because I wanted to be robocop on stage... Maybe I should've made it less flashy. I like stainless steel and aluminum. You'd be surprised the number of people who want me to make this stuff more visually ridiculous. I don't want to. ``oh you should have this spider thing that comes around''. And it's tempting, I could be more popular, find some stupid tv show... but in the end I want someone to stand in the back of the room and not even see what I'm doing and say oh this is good music. 

You see the gimmicks that people do. I already feel too gimmicky as it is. 

ET: can you talk a bit about the development of your drone machines, then the dub machines? 

TS: the dub machines, and now I'm making these masks. The drone machines were the first ones. They were about simple rotations; force feedback. Making sounds that would be couple with rotational or linear motion that were heavy sounds that followed the profile of the weight versus... momentum? If you spin the disc that I have with a rotary encoder, and you slow it down with your hand you have this natural decceleration. If you plotted the speed over time you could see, because of the weight, you have a sound profile that's very natural. If it were plastic you'd have more of a linear pattern. If speed is pitch, plastic would be ``eeeeeooooo'' versus the heavy steel is ``heeeeeewooooooooooooo''. Or I can keep the speed, since is has a nice bearing in there, at a constant velocity, to keep the pitch going... I call them drone machines because they made much slower, drony music. The album was one thing, but when I would do it live at festivals, it'd be much more droney and more improvised. 

But the big problem with those is that they're really heavy. I couldn't take them anywhere. I couldn't go overseas. I also wanted to play faster music, so I came up with this setup where each machine had to be 50 pounds, with their case, for flights, so that changed things from being steel to aluminum, and then I didn't use ball bearings, I'd use teflon coated linear shafts, which aren't quite as nice but work well and are super light. Things like that. Spending a lot of time on which case to buy. That was a much more compact design, and I could just sit at a chair with everything around me and move my limbs. Then the masks... I found I used my voice as my main music making device because it always available to me. My hands are doing things. The head mic is just constanly developing into different things. I can use it as more of a midi controller, blow into different mics and trigger, but lately I've just been using it as if I scream in the middle of it different parts combine. I'm basically just using this one big mic with different effects. Then the new machines, which are these masks, are still in development honestly. Much more meant for acoustic settings, in a gallery with no amplification. They seal my face and I can do things with my voice using them. A tremolo, opening and closing the valves... I have this throat mic, essentially a world war 2 communication they'd use in tanks - you can speak very quietly and you'd hear it. It works great! I use it all the time. That thing is on my neck the whole show. I have effects with it and I control the bass tones... 

ET: Is there a clear divide between using the mic as a controller versus a source of sounds? 

TS: Actual audio vs midi? If you couple them together, for example I run my throat audio with some effects, then I can pick up the pitch of that and control a synth with it. So I can control a bass synth by just rumbling my throat. It's amazing, you can control the subwoofers just with your throat. I never really choose. I mute and unmute those channels quite often as things develop. I think it's nice when it gets beyond formulaic. Sometimes you don't even know exactly how a sound happened. That's why I like music and art a bit more than engineering sometimes, I could be doing something terrible to the sound system but having it sound great - now I need to figure out why its terrible and how to fix that without losing the sound... 

ET: I did want to develop the composition aspect of your work. How does the engineering and music processes work together? Is it an exchange between both? 

TS: There's a sound, in my head. For example this mask I'm working on, I've been composing these songs that I am using them on, and I had all of them in my head when I went to write them. But then when I went to record them, they weren't as palatable as I'd hoped. So the people who helped produce the album wouldn't include them. It was just too harsh. I thought it would combine better, but it doesn't. It's much more organic. So I'm working through the limitations... The way my body's working with the solenoids, the valves, the air pistons... they're slamming into my face, which is something I didn't expect... It's really harsh and I can't play the pieces I wanted to, so I'm re-writing those. And that's a nice accident, because this whole process is taking longer but it'll have a better effect. 

The limitations of what you do, design, what you think it's going to feel like and what actually happens when you play it, those are dfferent realities. 

There's different rhythms. Tonal ideas I'll have as I'm walking around designing things in the machine shop... when I go to work, then I'm at home, constantly thinking about that and what it should sound like. I don't really write the songs until the machines are done. 

ET: Do composing and designing instrument ever become part of one more unified process? 

TS: I definitely think I'm more a musician and an artist. There's something exact about what I do, but what I like the most is the accidents and the free form nature of what I'm doing. I use the engineering to achieve the goals I want for my music. I'm not making products, I'm not trying to think about having other people play this. It's come up before, the idea of me making products and things like that, I tried... thought about it... it takes the fun out of it for me, to go to conferences and show people. Market stuff, take that weird quality that you put in your instrument that you only understand and make that adaptable for someone else's interest. It doesn't seem interesting to me. I had one instrument, the idea was that I was grabbing onto someone else's ears and screaming into their face. So if I grabbed on to the microphone, that's it meant to me, to use that instrument. How do you make that marketable? A product? There's a weird connection to your instruments that makes you feel powerful, for yourself. I think it's much more art at this point, not product. Engineering is product. 

ET: Buchla described himself as a designer of instruments for a niche market, rather than a builder of tools. 

TS: Also, there's just no money in musical instruments. My time is better spent making stuff for me to play. If someone steals it and does a shitty plastic product, that's fine... 

ET: How important is the engineering side of your community? I've seen videos where you work with CAD-CAM tools, which are infamous for their proprietary nature. In parallel, the diy analog synth world is very much about sharing ideas to make good sounds. Do you feel like you're part of any larger effort? 

TS: I don't feel like I'm in a community. I have had people help me, anything I need to do in arduino, ableton, max for live... I couldn't do it without forums and calarts students, or friends I have who do similar stuff. So I need that world. But at the same time I use solidworks, I use it at work, I'm not going to go use some open source equivalent just because it's free, or because I want to build that community. I just don't have... I've tried teaching, I've given workshops on how to do this stuff, i'm not too good at it. It's just not one of my goals. It's hard enough using this weird combination of tools and getting a grip on it. Getting access to CNC machines is extremely hard, I can't be picky. I bought a cnc router to make speakers, redid the controller on it, the motors, built a vacuum table... good experience, but I sold it for about what it cost me, I'd rather have a cnc mill at this point. I value that community, because we wouldn't be able to do what we do without it. 

I know there are free cad tools, but I'm good at solidworks and I get it for free from my work. There's also some elements of software for live that I'm very weary of, and I do spend money on ableton, not really using pd... some of the people that tour a lot, like myself, have to be able to rely on their equipment. Even Max for Live, I've completely removed from my setup because it crashes. A lot of people use it, its great for development. I have some very useful things I've written on it, but it just doesn't work live for me. The amount its crashed... The people who use pd for live... they say how can you use ableton, etc. You go play 25 shows in a row, with your pd setup. I'd like to see it work. 

ET: It's a big obstacle in making live electronic music cheaper and easier for everyone. The free things isn't always reliable.

TS: I'm just trying to make music. I have no purist aspect, other than playing everything live. Anything I have to do for that is ok. If I have to get funding from the North Korean dictatorship, I'd do it. I'd take corporate sponsorship, too. I've done it with Lenovo. 

ET: You've mentioned you've done some teaching... do you also help people with projects if they're similar with your work? 

TS: I taught a class at UCSD, on music theory and production, more on the software side. That was fun, teaching people how to make beats. I've done workshops on 3D fabrication... Anyone sending me an email asking what do you use, as long as I'm not giving away secrets on how I get my sounds, I'll help people get the arduino setup going, the HIDUINO stuff, most of my stuff comes from other people. 

ET: Hiduino? 

TS: it's an alternate firmware for the Arduino, it turns an UNO into a USB class compliant piece of hardware that's recognized by Ableton. I don't need USB to serial conversion. 

There's definitely people doing much more complicated technical things. I don't do any sequencing, I don't use any drum machines. I just hit the drums, live. It's basically like there is a button for each drum. Sometime I'll loop something, that I've played live, play over it for a few bars... 

It's simple stuff. The key of what I find valuable in what i do is just making physical devices that feel good, that are strong, and that just have good interface designs in my opinion. 

ET: Do you document this process for yourself? 

TS: I comment my code and save that... The code isn't long or complicated, and it's not necessarily that good, so I don't want people to give me shit and read it and use it... But I have put up some stuff through Make Magazine and Wire that's available, for some things... People could email. I haven't had that happen much, probably because there are the people out there that I got it from, so readily available - making a midi controller with your arduino, it's simple. There are other devices like the raspberry Pi, the beaglebone... I'm interested in using them, I just haven't had the need yet. One element of what we do now in the live setup is I'm working with a video artist who uses VDMX, which is a semi open source software for programming lights and video, we take all my audio and midi and we send it out to him in the back at the front of house. He has my sounds directly control the sounds and the video. That's nice, it's very live, he has a hand in it as well... 

ET: what's the name of the person you collaborate with? 

TS: Will Michaelson - he goes under the alias cutmod. You'll see it on youtube. He's way more on the psychedelic side, for his stuff. But when we work together... you'll see in the videos. 

ET: does it complement your performative elements? Or are you trying to obscure your instruments, even if they're sculptural? 

TS: We started by having cameras on my stuff, projecting that. The instruments are interesting to people, but for me they're instruments. Like I'm playing a guitar, or the synthesizer. Showing off the instruments is ok, but it only works for so long. Then it becomes the whole thing, the gimmicky instruments. I wanted to go past that, just good music. The ideas and the motions, the abstract behind what I'm playing are images of life. Film, ideas... they don't necessarily have anything to do with tech. How long do I want to geek out on the tech? Right now that's probably the most successfull thing I can do, shoving the tech down their mouths, but there's so much more to it for me. None of my songs are about anything technica, or mech. e. or robotic, or futuristic. Doom, apocalypse, death of the planet... 

ET: it sounds great. 

TS: I know that's what a lot of people just like. they come out for one show, then they say they were hoping for more electro `` I listen to Skrillex''... 

\section{transcript for interview with Nicolas Collins}

(conversation informally starts on a discussion of headphones, earmuff padding, and the suggestion that the two recordings of the chat will be used with one out of phase to create a composition): 

NC: ... In some concert halls, they have devices to jam cellphones. This is known. I've seen a handful of references to circuits that do this. this isn't like building a nuclear centrifuge, anyone should be able to do it. The only tricky thing is working at such high frequencies that cellphones operate at. It's not as easy as building a fuzztone - the level at which I work.So I see a couple of versions of this, small ones not powerful enough to fill a whole hall, but I love the idea of carrying something the size of an iphone that creates a black hole around, wherever you go, so that in a 3 meter circumference everyone's cellphone's stop working. 

ET: Have you ever read the Pirate's Dilemna? It opens with a discussion of the ethical implications of using those ipod radio-emitters for your car. They could jam a frequency not only in your car, but in a 2-3 vehicle radius around you.

NC: There's so much interest in what's called \"silence studies" in the last ten years. This is cyclical. There's a Cage / Rauschenberg moment in the 60's, then it came back with a vengeance... It took so long for Cage's ideas... not to be accepted, but rather internalized. For example, people from many aesthetics could view silence as a positive element, rather than as the absence of something. It hit something, at the turn of the millenia, when all these people realized you could carve things out of all those negative spaces... and in some fields this had been accepted for a long time. Graphic design... It's one of those things that follows a zeitgeist or a pattern or a cycle... That idea comes up a lot in hacking. That idea of injecting something or removing something...

ET: I'd like to try and go through a few questions, hopefully we can develop that as we go along those. I wanted to start with something basic. I'm interested in contemporary practices in electronic music hardware design, trying to link the first electronic music instruments with diy and tudorian electronic music and today's open source movements. You're in a great position, where you've had a chance to work with Tudor but also see taken advantage of contemporary technologies...

NC: I'm old, yes. But I'm still alive. 

ET: and you wrote the book! 

NC: fine... I did do that. 

ET: So what's the current place of hardware in what you do today? 

NC: ... (pause) you know... because of my age... the arc of my material resources are a little different... from yours, from those of my mentors. I'm from a particular generation. My first work was based on hardware because when I was 17 and wanted to do this there was no computer. It was the epoch of mainframe computers. You would not get access to those as a young person, and even then they were an offline, non real time thing, and from the very beginning I was interested in real time music production and performance. So the alternative was synthesis. This was the era of synthesis. However, synthesizers were also impossible to afford. None of these technologies were what you'd call personal. Ownable. 

But it was at that moment that integrated circuits went from being extremely functional building blocks, transistors out of which you'd design a whole circuit... to... more modular things, that could do more things out of the box. The most critical chip was a signetics 566, which was an oscillator on a chip. 8 pins, you hook up a very small number of components and you'd get a couple of waveforms. It was designed for touchtone telephones, which is the only reason it existed, because that had a huge market. And across the board, you'll fine people from my generation and a bit older for whom that was the first thing they've ever worked with. Because who doesn't want an oscillator? 

So... my entry to electronic sound was hardware. But by the end of the 70's, microcomputers emerged. It was pre-apple, industrial computers. It was sort of like large Arduinos, right? And it was a lot like working with arduinos. So composers from my generation, who are now say between 55 and 70, really dove into the computer stuff early on, and by the mid 80's it was sort of a no brainer. You could get so much done. The biggest drawback was that you couldn't really do internal audio processing on a personal computer until the 90's. But there was so much available in terms of MIDI controllers and everything, and I basically since 1979 kept two parallel tracks of doing these circuit based things and computer based things, because the fact of the matter was I wasn't terribly interested in the sounds of synthesizers... so I had to find other roots. 

ET: and a summary of that is in your paper about early microcomputers in your practice? 

NC: Yes - that's the paper on semiconducting. It's the idea that certain technologies have natural strengths. from my standpoint. And most composers think of orchestration as a decision, rather, than say writing for violas the rest of your life. So for me, switching back and forth between technologies was no different from switching back and forth between instruments. 

But I have to say, as the decades passed, it was obvious that the computer was becoming the more powerful, more versatile tool, and if I wasn't willing to spend the time being a brilliant analog engineer - I was always self taught - there was much more possibility and much more openness and much more of a community for a sort of open source in the software domain, rather than the hardware domain. But I kept a hand in the hardware all throughout this time. If you look at the few records I've done over time, there's all these oddball instruments. Hybrids of electric and mechanical things. Sometimes maybe guitars, live sampling systems... It was all a mishmosh. What happened, what changed for me was that the end of the 90's, I started teaching in art school. It was this moment where you may be able to identify more clearly... I call it the digital hangover. The computer had become so powerful that people were just knocking back shots without thinking of the consequences. You couldn't really do anything. My mantra's always been control-x / control-v. It's the world's most powerful tool! You can cut a term paper, you can cut audio, you can cut video, you can design a website. It's the world's most amazing pencil. But as I discovered, from the art school context, art students are peculiar in the sense that every single one of them, even if now they do exclusively digital, they all started drawing. There is not an artist in the world that didn't scribble, even if now they use a mouse. And that seems to be really ingrained in visual artists, this desire to do things with their hands. We think of that as a musician's thing - musicians are about the tactile. But I think that musicians play their isntruments 24 hours a day. They have a nice life+work separation. Artists are always fiddling with something. 

It was those students who pushed me to do the class, and it was this generation of hungover... from digital overindulgence... that led to the rise of circuit bending. Because the circuit bending movement went back to the early 90's, when he started writing articles for the experimental musical instruments journal. And there was always a little cult of this stop. Always this buzz in the air about the speak and spell. I had a speak and spell in 1979 that I hacked. this is pretty basic stuff. But he took off at the end of the 90's, with this sort of anti-computer backlash. For a while people were waking up one morning and saying oh \"I'm never programming again". And for a while it was like that, a real split between the circuit bending people and the computer music people, and they basically had nothing to do with each other. Circuit bending people were militant about their anti-computer stance. Porta-studios came back with a vengeance, the casette was a real format... It was almost like a luddism. But then... a few things happened. The most important one was the sort of parallel growth of limits in the open source community and the arduino. Those two ... people had been making arduino type things since the 80's... STEIM made this beautiful sensorlab thing- but it was \$3000! Completely insane. So the combination of the affordability of the arduino and the open source nature of doing programs on it and the fact that they had provided this glue between the physical world and your laptop meant that it was like the peace accord in Belfast. Suddenly catholics and protestants could talk to each other - over the top, but I think a lot of orthodoxy broke down at that point. 

ET: people realized the speak and spell used microcontrollers. 

NC: That's what I tried to tell these guys. Every single toy they use is a sample-playback computer. I did a workshop with the other Nick Collins, in Mexico, some years back. There was so much confusion about the two of us. He's a supercollider maven, and the organizers could not figure out these two dudes were two different people, so they built these multiple workshops with Nic(k) Collins, with no indication of which was which - there were 2 or 3 of them. And of course everyone came to all of them and they didn't know what they were going to get. We decided we would simply do the workshop together and every hour we'd switch between software and hardware. It worked! It was clearly the threshold point. Everyone was equally comfortable working in the two modes, which was a big change. Where does that gets us?

It gets us where we are today. Coming into teaching late I'm much better at making the distinction between my life and my job than I was when I was a grad student. But I can't deny that teaching, not only in chicago but also these countless workshops have fed back into my own practice. I got interested in one very specific thing at the beginning which is that when I would do a workshop I would have 25 kids sitting around a table with little amps and speakers working on kind of similar projects or technologies at the same time. Everyone would be working with contact mics, or making their first oscillator. But it was this great orchestral electronic sound, that wasn't mixed down to a p.a.... it's also in the same general region, but uncoordinated. Now for a guy who's background is in deep minimalism... I started opening up to a chaos... the things you can get with a large number of human beings that you can't get with a line of code - unless you're really really clever - and I'm not, I write relatively simplistic code. So I got interested in the group dynamics of hardware based stuff, where you don't control things as accurately as... god forbid, a guitar, in your hand. 25 electric guitars in a room, it'd be a very different experience. I got interested in the noise world. The sound world of... disreputable electronics. Electronics that you weren't sure were working correctly, or that you knew was damaged but still interested in the sounds it could make... so I did a piece called \"Salvage" - it's on youtube - where you try to revive a disfunctional or broken circuit by essentially injecting voltages into an unpowered board and basically using it as timing components for oscillators. So you get a very complex oscillator with a high degree of chaos in it. And it goes through a set of complex evolutions as more people start joining. There's a very simple instruction set. The idea is that it sarts out relatively cause and effect-y, because there's only one person doing this, but by the time you get up to 6 you get this sort of density of decision making that's very difficult to think about being done with a computer. 

That being said, you know George Lewis made these really beautiful softwares that improvise. George has been working litterally since about 1980 on program that improvise. And because he is such a great improviser, he's someone you should pay attention to. The basic idea's always been that the computer listens to the player, and responds as if it is a player. The reason I mention this is because instead of creating a standard algorithm for what its improvisation should be... to the best of my knowledge, and you'll have to confirm this with him, what he's actually done is that he's written different routines that embody different improvisers that he knows. So that in his computer he has multiple different personalities that behave differently in response to the same data. 

Now if I was smarter I would try to do something like that. Computer program that instead of having 6 people doing something I have one person do it, and then five \"people" to play along. At the same time, I always have a lot of warm bodies in this workshop and this one way to harness the energy. I've spawned a couple of solo pieces of from that. What I'm trying to do is harness the apparent chaos and comformability - seemingly incompatible - of some analog circuits - but use software as a way to get rid of the the sort of monophonic property that most circuit performances have. To create some sort of complimentary behavior. 

ET: The other person I'm interviewing tomorrow is Dan Snazelle. Are you familiar with his Ardcore module? 

NC: Yup, that's right. Do you want a précis? A short version? I've carried parallel practices in hardware and software for years and years and years. They've always worked together but what I would say is at the moment, it's the chaotic aspects, the instability of circuits that are coming to a full forward in the stuff I'm doing. 

ET: So has there been any one device or project that has created a noticeable shift in your work? 

NC: No, I think there's been multiple ones. David Tudor used to talk about how he never understood tubes, and then Gordon Mumma tried to teach him how tubes worked, and they tried to build a tube amplifier, and tried three times, finally giving up. It wasn't until the transistor came around that he was comfortable making circuits. 

For others of us it was the integrate circuits. I'm lousy with transistors, but ICs are a piece of cake. The more complete building blocks are great. My whole book is predicated on this CMOS logic circuitry (26:18) from the 70's that lent itself beautifully to running on batteries. That was a critical technological bridge. 

For most people, the advent of midi and pc with reasonable userbase, so that software could be made by people other than yourself. In the 80's, the conflation between the music industry and the computer industry was critical for a lot of people. It didn't matter so much for me, a lot of my stuff had backed off from the computer, but for the community at large... 

Then when computers actually got fast enough to do real time audio processing. So when Max went from being a MIDI generator language to control synthesizers to having an MSP component that allowed you to do direct sound manipulation, that was a big deal. 

And I think... I don't do a lot of stuff with Arduino at the moment, but I know that that has been the next big step, because it's solved the problem of connecting the computer world to the physical world. Foundations like STEIM had been working since the mid 80's to make that work, spending billions of dollars on artists residencies and research. And suddenly, this Olivetti guy shows up and \$25 later, you got it all worked out. So you know, open source and Arduino would have been the next big step. And I suspect that this is going to be very important. My guys who started the first laptop orchestra at Princeton are now doing iPhone orchestras at Stanford. Ge Wang and Perry Cook... they had a very conspicuous laptop orchestra, and when Ge gets out to Stanford, he ups the ante and starts a phone orchestra. 

ET: Dan Iglesias made a nice wrapper for LibPD called mobmuplat...

NC: Oh right, that's why the name is familiar. It makes a lot of sense, there's still a lot of people who don't want to use their computer on stage. They just like the idea of wrapping it up in a smaller package. A point was made to me that people are developping apps much faster than people are developing full feature software for larger platforms. For every major rev of Ableton or Max you have a million new apps that allow you to test all these areas of work.

ET: Is there anything you're curious to see implemented? 

NC: I would be interested in - and I think some people have done this - but I'm very interested in sort of the electromagnetic spectrum that we have around us. Kristina Kubisch does these really beautiful EM sound walks, and I do all these things with coils in my workshops where you pick up the sound of your iphone... but I've always been curious what the wifi traffic sounds like. Make a really simple receiver in that bandwidth, with a frequency shift to bring it all down - not to steal the information, I couldn't care less what people are doing - but to hear if there's any rhythmic quality to the community that's working in that spirit. 

ET: there's somewhat of a visual equivalent that's been done, with some code sniffing all the image content being downloaded for people's webpages and attempting to recreate an approximate mosaic of the overall network's image consumption. 

NC: and I've seen demonstrations of some of these slightly suspect softwares that allow you to look at wifi traffic on a network. clearly it can be done. It's just that there's some difference between extracting the data, and my desire is so much simpler - what is the sound of all those things going back? You do have to do a little of stuff, because even when there's no data there's still a constant carrier, so you have to get rid of the droniness... 

It's an interesting point. There's this piece called something like \"just because you can, should you", and it's a reaction against the diy community. There's just a sense that its creating so many things? do you need to be making all these things? The downside of this world you're looking at is it leads to a preponderance of things. It has environmental impacts. Recycling software is much better than throwing out a circuit with a battery in it. It's a question of resources. Then there's the moral aspect, the psychological aspect of hoarding, with being object-oriented. If you've talked to people in bending communities, very often the instrument remain in the forefront of musical practices. There's something pretty dreary about concerts a bending festivals. It often seems like the music might be an afterthought. There's nothing wrong with being a luthier. There are people whose tradition is building great instruments. It can be Stradivarius, it can be Trimpin, it can be the engineers that are behind the cracklebox, it can be bob moog, but they're not necessarily the people to who you want to listen to records by. So I think that there is a need to be clear about that. From Tudor's generation down, there's an air of tension. Am I going to be taken seriously as a composer, if I make this thing? Am I going to be taken for an artisan? 

ET: That's one of the things that's fascinating to me about Tudor. He comes from this very respected musical standpoint, but embraces the experimental electronics, live electronics practice, and he's taken very seriously. But that seems to be mostly because of where he's coming from. Not necessarily because people objectlively thought his electronics were producing compelling compositions... 

NC: It's complicated. He had this reputation as a virtuso pianist, and then the artistry was elevated in his role as the interpreter, the realizer of these Cage pieces, whose scores had to essentially be translated for performance. That act of conversion, he elevated to this high art, which very few people have reached since. After that came the creation of these electronic instruments in service of the cage scores, and after that came David Tudor as composer. It was too many talents that leeched stuff along the line... there was a smaller vocabulary left to describe it, so to speak. I'm very conscious - I've known the guy from the early rainforest period - I'm very conscious of the fact that it's only after his death that the composer aspect of him began to be treated seriously, in terms of the the written stuff... There's that issue of the Leonardo Music Journal called composers inside electronics, which coincides with the getty papers. That's where you'll see a nice overview of the different periods... 

ET:Do you know about the Little Bits kits? 

NC: I was at Moogfest, the guy from Little Bits gave a talk... 

ET: Korg has a series of synthesizer based ones... 

NC: I've seen a number of those things developed over the years... Radioshack even tried a few times to make some of those lego-y things to teach electronics. Since my kids were really into lego, I tried to show them that, the mindstorm things too... But my kids never latched on to that, and I never invested much time into using these for artistic experimentation... I think it's all quite good - here's my take, getting back to this idea of because you can do it, should you - we all tend to loathe ourselves and the group we represent, so I'm always very conscious about promoting ourselves and the ideas... I have this weird reputation as the hardware guy. If you'd read about me 15 years, I'd have this weird reputation as the computer guy. These things change. But am I weary of people setting camps. \"I'm not going to use unless I make it myself" or \"I'm not going to use it unless it's linux" or that kind of statement... but I do think that one of the great virtues of learning to program or learning to work with hardware is that we get a better understanding of the technologies that our lives are ruled by, across all domains. My father's generation was one that tinkered. He was a college professor, for christ's sake. He'd build a book shelf, not go to ikea - we didn't have ikea back then. If the car fucked up, he'd try to fix, whether or not he actually could. It was assumed you would open the hood and check the oil, and make sure the cables weren't frayed, and you' try to second guess your mechanic. 

I tell people how the first time I was in Europe in the mid 70's, I was in Germany and I saw that all the driving schools has these models of cars, cutaway models of car with this cross-section of the transmission an everything! Like the visible fish. I learned later that to get a driver's license in Germany at that time, you had to answer questions on the written test about how a car works. Not just what this sign means, but also explain how a carburator works... 

ET: which is what HAM radio tests are today... 

NC: True. Trevor Pinch edited a really nice companion to sound studies. There's an Oxford one and a Cambridge one, he did one of those. There's a really beautiful thing on German... in germany in the 30's, there was this emphasis on diagnostic listening. You would be taught to listen to the engine of the car to pinpoint defects. My father's generation, they'd be taught to replace the tap washer when it would drip. The idea was that the technology was open. Even if you didn't understand what was happening, people would open the hood. People do not open the hood anymore. One of the things that happens in my workshops that is ultimately the best takeaway, is that there's always someone that comes up and mentions that dreadful word, \"empowering". They may never touch a circuit again, they may have done this because they thought it'd be fun, or because their boyfriend was doing it or something like that, but they say it was the first time ever opening a radio or this or that. It's the first time they'd ever touch their part. 

ET: Do you feel like there's more than intuitive connections between this long-standing practice of opening things up for music and the more recent open-source movement? 

NC: I think it's very unlikely that an obscure music fringe had an influence... I do think that there are certain social trends or zeitgeist that have a long nose, rather than a long tail. A long nose, where you sort or see these signs of a build up. Take something like the arduino, which has a very strong presence in the diy community now - it's a very good universal tool. As I say, you can look at proto-arduinos that have been produced since the late 80s, but it had to hit a certain price point. Just like circuit bending took off because there was a shift in cultural consciousness, there was a broader acceptance that you didn't need to know what you're doing. My generation, even though we were terrible engineers, we really tried to understand what we were doing. The only reason we would do something interesting is because we didn't [understand what we were doing], but we tried really hard. When the benders came around, the whole idea was \"don't tell me how this work". I have this quote in my book, one of the first things that happened: I was setting up on a little table for one of the workshops and this mountain man comes up and asks \"are these bent or hacked?" So I ask him what the difference is and he says \"oh, um, bending means you have no idea what you're doing when you open it up, and hacking means you have a little bit of an idea." Then I thought, from an ecclesiastical standpoint, that's kind of interesting. So we had that ground shift, and it was the same thing with programming. I remember when we first started with microcontrollers it was like \"oh boy this is going to be hard, we have to learn how to do this, we have to learn how to do that". And my students discovered that no, all you have to say is I want to control the speed of a motor, all you have to do is search motor speed control arduino, you get a chunk of code, you cut and paste. So that is amazing. 

I think there is still an issue, which could potentially be called a problem, which is what we might call the preset idea. When the DX7 came out, it had such an amazing timbral palette, compared to most other things. 98\% of the users never got beyond the presets on the front panel. There were many, and they were very rich. Except they were finite in number, and after a while you could pick them out in pop songs. The algorithm for FM synthesis has a certain sound to it, but some people did remarkable stuff to it, really differentiated it from the presets on the front panel. The problem with bending is that in a way it's a bit like presets. In other words, we are now on the speak and spell preset or the casio preset - and you can identify them. With the cut and paste approach to code, it can lead to something similar, which is that module of code, which you didn't end up tweaking very much because it did a good job... now something like motor speed control is pretty utilitarian, but there are other aspects. But you look at other languages, Supercollider, Max MSP - those are very open as to what they can do. But both languages, Max in particular, come with all these modules, these objects, that are very powerful but also quite recognizable. There was a period of several years where I could identify a max patch just by hearing it. It mostly had to with the sample playback stuff that Max did very easily. It was another boom to people who were starting out, but it was a sort of presetting. 

So this is potentially a danger with the app market. It takes a very powerful programming environment, and it generates one patch, so to speak. If you develop it yourself, you'll spend some time tweaking this and that, then you can use it in 3 or 4 different pieces and it's adapted. When its an app, it sort of sits there begging you to use it and have it be your instrument. 

ET: For me there's sort of two origins for presets: the community at large, the programmers - those are the tutorials, example patches that are built in and the cycling 74 website - then there's the patch your friend made next to you. I think the distinction is important and interesting, for how those two communities work together. How do both communities influence your work? How important have other people been in the development of your work. 

NC: It's a mix. Statistically, the students of the workshops have been more influential than known individuals. With one or two exceptions. In the early days of the workshop, I had this vocabulary of techniques that were chsen because they were relatively easy to do, inexpensive, and most importantly they did things that computers couldn't do easily. I wasn't trying to do stuff... I was trying to match a market need. It wasn't a Moog synthesizer and it wasn't a computer. Along the way, the assortment of project and the tweaking and tuning of them was very much influenced by the feedback I'd get from people. The other thing is, people suggested stuff. There was this guy, John Bowers. A computer science professor, and he's also done really interesting low end electronic stuff. He was the one who showed me this business of the making a speaker into an oscillator with just a battery. He called it the victorian synthesizer. He brought it up when we were doing a workshop on loudspeakers and all the things you could do with them. Now this is a standard part of the workshop I do - it teaches you so many things, you can get it going instantly... But I look around and... I sort of see what people are doing. It feeds back in. But my general instinct is that I get more from general feedback from the participants. 

ET: so this is the side of the community that you feel is more influential than your peers? 

NC: yeah... maybe just because there's so many more of them... maybe it's because they're younger, and they have a keener insight into what's changing. I'm always looking for the next thing. Starting about 5 years ago, I saw this interesting, incredibly low level electronics. I see this sort of arc, which is best represented in Korea. There's an awesome scene in Seoul. It's Doto Lim and Ballon \& needle project. Otomo Yoshihide comes to Seoul, and it's like this catalyst for this sort of noise. And you see this evolution: lets start a band, then lets add the effects, then it gets noisier and noisier, and then they say lets disconnect the instruments and use only the effects. You go from Otomo to Japan Noise... then you get to the point where they say lets open up the effects, lets see what's inside, lets do a piece with just the one transistor we pulled out (56:37) from the pedal... let's just do something with dirty contacts. It's this funny kind of arc that's represented very well in the Korean scene. I've seen this post- effect pedal stuff happen. It's really interesting. 

ET: How do approach limitations in your work? Have proprietary tools and designs or planned obsolescence affected you? 

NC: The notion that if you pick up an object, whether its a violin or a chip, it has certain limitations built in to it, that would impose a method for using them? 

ET: Throwing back to the notion of presets we were discussing earlier. 

NC: That might be why, as I say... I for one try to avoid defining myself by a medium... like computer music or hardware hacking or chamber music or imprivisation. I'm sure there are people who are happy being in such a niche. I like string quartets or I like piano music or I write for Jazz bands... But my personal interest is to seek out different resources and work within the confine of those. If you look at my background, there he was with Lucier, sort of experimental music, electronic scene in the 70's, then in NY in the 80's, working with improvisers and downtown bands, then in Europe in the 90's, working with chamber ensembles, now in Chicago, in the boondocks, teaching at an arts school, and he's created this whole cult of workshop based hardware practice. Each one of those has provided its own benefits and limitations. But I think that's something that's ubiquitous to art practice. I don't think it has o do with a time... it's always been the case. I'm only saying this because in art school I'm very conscious of the fact that these days you have very few students who define themselves in terms of medium. Few people say \"I'm a video artist" \" I'm a sculptor"... they say \"I'm an artist". And then \"oh yeah I also do some video and some other stuff and I draw and I've done a print edition". The only people that define themselves by medium these days are painters. Painters still do that. And not all of them, but that's where you get the highest concentration of self-identified students within a medium. Next question! Let's try to get through all of them. 

ET: Is personalization of electronic music instruments just a set of decisions concerning which limitations are acceptable? 

NC: Yeah... I think people sometimes have different ways of defining what's an instrument. But from the classical era to the the rock and roll era people said \"I'm a violinist" or \" I'm a pianist". The conductor was the oddball, but people who made music defined themselves in terms of their instruments. What happens is now, post-electric guitar, the instrument has expanded. Is an electric guitar just a string and a neck? or does it have an amp ? How do you relate to the amp? Oh, you use a pedal. Is it a fuzz, an overdrive? Suddenly you're performing in this network of technologies. Then as I said with my Korean friends you disconnect the guitar, you only play the pedals... there's the transition. I think that it's more difficult now to localize yourself as an instrumentalist. Then you fall into this thing of realizing that an electric guitar is an incredibly versatile instrument. It's been used in so many styles of music. Nobody ever says \"oh my god not another electric guitar"... well they do, but not in the same way they say \"oh, a wahwah pedal...?" or \"a vocoder?". If you wall yourself a wah-wah pedal-er, it seems... much more limited. And maybe it's that preset thing again. It's got such a limited range, it doesn't cross that threshold of expressiveness. That being said you have people that have made a point of working within that. People like Toshi Nakamura and the no input mixing scene. Some of that is incredibly limited... or Sachiko M's early stuff with samplers, where she only played with sine waves, the test tone in her sampler. 

ET: Hannah Perner Wilson's MIT master's thesis discussed making basic components from scratch. She defines the advantages of such a practice as the opportunity for personalization, a better chance for transparency, and the importance of skill transfer. Any thoughts on this? 

NC: I'm doing an advance hacking course this year, so I'm in that right now. We're doing sophisticated designs, but also stuff like baking our own piezoelectric cristals, so we're going in both directions. We're doing stuff by making various parts with kitchen chemicals. It's very good in terms of understanding the material however usually what you make is not as good as what you can get commercially. So it's much more of a learning experience than something that makes sense from a product standpoint. If you haven't, you should read a book called the toaster project, by english design students who decided to build a toaster from scratch to see how the industry actually worked. Hammer pennies to draw wire and everything like that. It's an exercice in how the economy of scale works these days. You can do beautiful performance things where you draw and use the graphite as part of a circuit... I've had students who've done this, I have students who've done etching and used the scrapings of the burring on the metal as a conductor for a sound performance. In some cases it makes for a very beautiful performance medium, but I think with very exception, to build these as substitute for commercial ones doesn't make much sense. Build a wearable circuit because you want to interact with it, not because you want to put a wool resistor inside your Moog ladder filter. 

ET: Going back to the community aspect a little bit, who do you feel like Handmade Electronic Music has influenced the most: artists or engineers, academics or sef-learners? 

NC: Fortunately for my publisher, all of the above. I thought they were crazy to put the book out when they took it. But... because I didn't think any academic would buy it. and it was an academic press. But I knew that there was nothing on the market for this sort of grassroots community, and all the people who were asking me to do workshops could buy it. And then I could stop doing workshops - well, its fanned the flame of workshops, but I think it was very well timed because there was always a need for a practical guide for the community of builders. But the viral history of experimental music that's in it and all those sidebars - there's 150 artists referenced in that book. That made it incredibly attractive for academics. It was being used in music schools, in art schools, it seems to have had a really widespread impact than I thought it would. More power to the book. 

ET: Who is this book, and more generally devices like arduino and tinkering practices empowering the most? 

NC: I think the biggest change is for non-academics. That's the impact of the web. There's a very large base of people who do not need the base of the academic environment for their education. The web is for a lot of things bad, and its use as a formal educational tool is I think deeply flawed, but as an informal tool, its amazaing. When I was learning circuitry, you'd get a xerox of a xerox of xerox of a circuit that got from Tudor to your hands through 5 other people. Some stuff might have as well come from the soviet union. Now, even before people spoke of open source, the early web was about people giving away information for free. I think that was critical. I'd like to see statistics - I'm guessing more Arduinos are sold to freelance artists and tinkerers than to universities for the arts and technology... 

ET: Where do you see the successors of your book? 

NC: I've really been waiting for another book to come out. I'm working on my own other book. 

ET: you might be your own successor. 

NC: I hope not, I'm really dragging my heels on this one. There was I think the exploratorium came out with a book last year. I haven't seen it yet but it's supposed to be quite nice. Sort of stuff with conductive ink, pushing new materials. Make magazine comes in and out of the periphery of my percetion, but i don't know if they've gotten into anything major in terms of that market. They pay a lot of attention toward Arduino, Raspberry Pi, Beagleboard... That's where the main area of attention for books is right now. It's a more book-able subject, and it's one that does tie in a strong academic community. 

ET: so do you see this beeing more a book than online resources? 

NC: I think it's going to take both forms. The textbook market is still strong, whether its paper or an ebook. It's going to be book-ish. My next thing is going to be a cookbook. I wanted to flip the tables and contribute designs in multiple categories. Instead of an oscillator circuit, it'll be an oscillator circui \"by Ezra". Then you'll give reasons why you did it the way you did it, instead of just giving out stock designs. That'll be couple with analytical essays about the diy movement from a sociological standpoint, and interviews with or essays by major figures in the scene, to give it a little weight. 

ET: In that sense, how has a knowledge of avant-garde traditions oriented your hardware work? 

NC: In the sense that I could never get a job with a legitimate designer- All my stuff is directed towards these bizzare applications that I had, and my test... the buddy I'm staying with, the guitarist Robert Goss from Band of Susans... In exchange for staying at his house, I always bring him some circuit or I repair something. He was the one who steered me in more practical directions - so I've been doing this basically since we were in college. I'm hyper-specialized in all of my skills. My composition skills, my playing skills, my hardware skills, my computer skills... I'm not a generalist in any way. But I've been doing these things for long enough that I can squeeze them in a direction that serves a more general public. I will say that one of things that was really important for me in the book was making it ideologically neutral. I wanted to be able to have a techno producer and a circuit bender and a sculptor sitting side by side not feeling like anyone of them was being doctrined. Now, as you go through the book, and the workshop, the level of the sort of the notion of experimentalism as a neutral topic becomes very high. It's not a bout rogue procedure, it is abut finding your own path through it. the point is although you can see that as an avant garde trait, most people realize it is applicable in whatever field they are in. The techno producer needs to get a better drum sample that separates him out from the other people and maybe this contact mic is the solution. The sculptor knows nothing about music, but realizes that this malfunctioning circuit that buzzes works well with whatever she's working on at the moment. It's a non-threatening form of experimentalism. When you listen to the book's audio tracks, clearly they come from an experimental vein, not a piece of pop music on there. I'm sure most people don't listen to those. The videos are different, because they're so goofy, they're like having you're own youtube channel. Even if you don't like cats, you'll look at the little funny kitten video. Even if you don't like Davd Tudor, you'll watch a 40 second recording of Rainforest by a three year old with a camera in his hand. 

ET: Is there an engineer's music? Do engineers come to your talks? 

NC: I once had an engineer come to a workshop I did. He was the least competent person in the workshop, I was flabbergasted. When I knew he was an engineer, he said, \"but this is the first time I've ever touched an electronic component. I got a BSEE using CAD systems, I've designed a digital signal processor, it's the first time I've soldered". There used to be an engineer's music. In the day of synthesizers... Most of this is stuff you'd never hear. It was litteraly made by the engineers. In computer music in particular there are people that are much more technicians than composers, but it's a fine, fuzzier distinction. I think in a way you can look at the music that comes out of circuit bending as, if not engineer's music, luthier's music. If Bob Moog made a record, would that be an engineer's music or a luthier's music. 

ET: Do you think creating instruments and creating music are converging practices? 

NC: Well, yes! That's the thing... there's not so much written on it... but back in the 70s as I was saying, there's this real nervousness on the part of post-cagean composers about being treated seriously as composers. Most of them very much disliked the word improvisation. They used things like \"open music" or open form score.

ET: In the same sense that Cage disliked the term experimental? 

NC: Not quite, he disliked improvisation for other reasons. Those musicians didnt mind improvisation, but they wanted to be thought of as composers, not improvisers. My generation comes around and says \"who cares, we can do all this stuff. It's not so critical". But I was very aware that it was replaced by the question of are you an instrument maker or a composer, when you make a circuit? Tudor's thing was composing inside electronics. You build the circuit that is the piece. He was very adamant about that. But those of us who worked with him, we had our doubts. Did I build an instrument, or have I built a piece. That was critical. 

The same exists when you write a software patch. You can do things in max and supercollider that you think are a composition, then you eventually realize it's more like an instrument, I could give this to somebody else and they'd make a new composition with it. 

I have this one piece, Devil's Music. It existed as an array of cheap hardware samplers in the 80s... live radio sampling, very successful piece. I made a software version at the end of the 90s, and started distributing and having performances of the piece. With DJs doing it on their laptops, sometime with me sometimes without. Although it was really an instrument, it was so limited in terms of what it did that no matter who ran the piece, no matter what their aesthetic was when they decided to play it and pick their samples, it always sounded like Devil's music. You could always tell it was that composition. It was like wow, there I succeeded... but that was the exception to the rule. 

That idea of are you an instrument maker, or a composer, or a performer... And you didn't even the whole can of worms that is sound art. Let's not go there. 

Western Culture has these sort of strange distinctions: composer, musicians, interpreters, improvisers... there's a lot of culture when honestly it doesn't matter. You build the flute with the reed at the bank of the river because you're bored watching the sheep, and you play something based on what your grandmother used to sing. When I was at Wesleyan, there was a guy studying Gamelan, and he said \"you can listen to a piece of music from 1500 and a piece of gamelan music from 1950, and you can't tell the difference between the two of them." This was during the heyday of Cagieanisms. There's no idea of innovation there like we have. that's a different western thing. 

\section{transcript for interview with Dan Snazelle of Snazzy FX}

Would you like to introduce yourself? 

I'm Dan Snazelle, I run Snazzy FX.

What's your design background? 

I'm self taught, with the aid of forums, lots of books. You can see I collect those. The typical starting point for a lot of people is Electronotes. I bought the whole package early on. I was an audio engineer, working freelance all over manhattan and the burroughs. I would just put up an ad, once a week, and there was a list 2 pages long of all the things I could do. Paid the bills for a few years, but it was really hard work because you'd get two hours in one burrough, then 2 hours in Staten Island, and everyone wants to do it right after work. Long story short, I rebought the Prophet 600 i used when I was a kid, and I wanted a nice poly - this was 2007 - but I realized that with my current job, and having kids, I would never be able to save up for something more than that. and that really upset me. 

My family had gone out of town for a week, so I went to an electronics store and got \$200 worth of parts. Found a schematic online, learned to read it, soldered it up, it worked. It's actually this thing (pulls out cookie tin with collection of controls and wires spewing out), it's got all these lovely pots on. It worked. You can even see the lovely inside (typical first project wire insanity). 

I just kept building little synth things. over time, I made this (points at older modular synth rack under the desk). It's not really any format... closer to frac? I just started building my own stuff. This is really ugly (still pointing at the modular rack, missing a few modules and dusty with hand painted labels) but it still works to this day. 

When did the shift to making synths for other people happen? 

The Audio Arc. There's videos of it online. It's huge. It's 19 inches accross by maybe 6 units tall. 60 controls on it. All because of this guy Charles Lindsey, who later helped me out with Snazzy FX. I had played a show in Brooklyn, and he'd seen the things I built - in plastic tubs or whatever. One of my friends knew him and told me that he wanted to commission me for a custom thing. I'd never sold anything before, I was just doing it for fun.  But I said, yeah ok that's cool, what do you want? 

So we drew up this little box. It turned into this massive thing that took a year to finish. I was building it in my bedroom, my wife was pregnant with my second child. She kept saying that if it wasn't done by the time he arrived, the thing would be going to the trash. She was angry, because it was in our bedroom. But when that was done, Charlie got that - it still works today. Really neat synth. It doesn't need any cables, but it has lots of patching through rotaries and stuff. He plays electric cello, does a lot with found and ambient sounds. He's also a really great photographer. He wanted to do a lot of different things. The Mini Arc (picks up half-finished pedal from the floor), it was one of my first pedals. It does pitch to CV kind of stuff, a tracker. It tracks very fast. He wanted that, he wanted it to be able to filter things... He wanted a lot. What I worked out for the tracking - someone saw it at his house one day and said oh we should get this guy to start a company! That guy was never involved, but one thing led to another, long story. Charlie came to me and said why don't you start a company? I was still mainly making music. A lot of music. Always been a musician. But what I was doing with studio work was never really going to go anywhere or be different. Working here and there. I worked hard and I like it, but it wasn't art. With this (picks up the mini arc) - there were three boxes (collects all the enclosures from various places around him) - they're huge, and very heavy. If you look at how big the board is, it's large... 

Everything with those is on purpose. The layouts are not grids. Everything about them is thought out. The colors are bright, that one glows in the dark. That took forever to get them to make it right. Saying it can't be done. These were everything I wanted to do, in a box. I did it. I figured, if I'm going to make a company, I'm going to have it be what I wanted to buy. I never had much money. If I was going to buy something, it would have to be perfect. 

Do you still use these first pedals in your music? 

Of Course! Certainly. I still own one of each. Two are in the shop, the first one works. It's funny, I get emails everyday from people who want to buy them. They've been sold out for years. I'm coming out with a tiny version of this one (points to the purple one). The Wow and Flutter is a eurorack module - right here (points at the module). But I'm bringing out 2-3 stompboxes this spring. One of them, the divine hammer - the prototype is right here (points at enclosure)... But anyways, I had these, and I'd been doing synth stuff, in bands... starting a company would just be easier with pedals. Euro [short for the modular synthesizer format name Eurorack] was not a household name like today. 

This one has CV out, this one has CV in, this one you can play a guitar into it. If you take the sync out, you can drive your VCO. It's pretty neat. So I sort of fell into having a company. I had to catch up really quickly. Look at the insides (of the first, big pedal, which he's holding). There's a lot of circuitry involved. 3-4 boards. Imagine doing that (the Audio Arc) if you've only been doing electronics for a year... You either do it or you fail. You catch up, you learn as you go along, as much as you can. 

By the time I got to the pedals, it was still a ton of work, but I knew where I was going. This one (the yellow pedal) is based on that (pointing at the audio arc). Hence the name. The pedal has the tracking circuit from the audio arc. 

How much of your circuits are your own designs? 

I don't make clones. 

So 100\% of your designs are yours? 

Analog electronics today are rarely entirely new. You're always dealing with the same blocks and fundamentals. When I design something, I try to make something new - that's really important to me. The Tracer city (one of the Snazzy FX products) is \emph{just} a filter box. It's filter with CV, etc. but I try to do it in my own way. I guess I try to think on a systems level, with things like that? Same with the Tidal Wave (another module). It might be that there isn't anything new about a filter, or an oscillator, but how you approach it, and how you structure it... and presenting, especially the interface? The musicality of something... I'm very interested in. I've been playing guitar for 29, 30 years. I've been making music my whole life. I was a musician, still am... but I don't approach things from an engineering standpoint, but I ask myself how is this going to function as an instrument, how is someone going to use this in their music? Early on, that meant how am I going to use it? With the audio arc, I wondered how Charlie was going to use it. Thinking of the end goal is important. Coming up with a panel first. Things like specs, or signal to noise ratios, little details like that? I don't want to sound bad, but I want it to be used to make music. 

Did you ever read the Buchla interviews where he discusses the difference between instruments and tools? 

No... it's a topic I talked about yesterday. People seem to say that Moogs are more musical because they have a keyboard. I'm not trying to start a debate. At NAAM, something I heard a lot, just with modulars in general, is that these are those mad scientist toys. Or people come up and say I bet you annoy the neighbours with these! Because it doesn't look like a keyboard, or a tuba, or a guitar... it has to make a racket. You can make beautiful music with these. You can also make horrible music. It's whatever you want. With the stompboxes, whatever I'm involved in, I want to know the heritage, and I have a lot of respect for the old stuff, to try and meet some of the older people, especially while I still can, that's really important. But those guys were looking forward. They weren't saying well the hammond organ  did this, so lets take this... they were really looking to how we're going to use these instruments. very open ended. It still can be very open ended, nothing has to be set in stone. I'm not really a digital guy, but the Ardcore did allow me to explore some other areas, with Darwin. I think it's neat how digital is utilized right now. I'm more an analog guy, and I'll stay in this direction, because there's so many areas to explore. 

In relation to that, you mostly have analog hardware?

Yes. Digital, especially the ardcore, when I saw what Darwing was doing with it, I thought it'd be great to bring this power to people. However, it still ultimately spits out analog. 

But sometimes the distinction between analog and digital is clear, like with a wavetable oscillator. Or even menus. Functions. For me, I have the most to offer in analog. That's why I got into this. I'm not knocking digital. Some of the stuff out there is great. But I got into this because I wanted more analog in my life. There's a real attraction because to me, it's more intuitive, I know more what's going on. Especially something like the chaotic attractors. It's an organic system. It's happening on the circuit board. It's not a simulation. I feel like there's an insect in the room, it's really neat to me. 

I like the magic of electronics. I know its not Magic, but I like the aura of it. When I was a kid I used to play with these radioshack kits, that had a bunch of springs. They'd come with a manual with a 100 projects that said connect this wire to this thing. It was electronic blocks, and you'd make a bird sound, or a motorcycle sound. As a kid, that blew my mind! Connect wires, and get something. I like that, still. Anyways. More of an analog person. 

Out of all the people selling digital synthesis products, you're the first one to try and incorporate the arduino in a commercial eurorack format. It makes sense to me that you and Darwin Grosse, who's part of Cycling 74, would be responsible for that. What's the place of digital as control and lo-fi synthesis tools? 

The Ardcore is a system. I learned C from Darwin's original sketchs. He'd made 15-20, well documented. I bought a book or two on C and Arduino or AVR. He said I could help him debug these, study the comments, copy and paste... a lot of the functions are already laid out. I spent a whole year on the ardcore. It was really exciting, it's a really great system. It makes me really happy when people write for it. People have done some amazing things. It was the right thing, it made sense, and I got excited. There was no drive for a digital product. It's open source - other music AVR systems were closed source. So I got excited and made it work. And the responses are good! It was a really neat project. 

It was also a real \emph{project}. Trying to do something that's new or exciting. 

What are the parallels between your roles as artist, engineer and designer? 

I've been making music for 30 years. Synths, for maybe 10? Snazzy FX started around 6 years ago. Not that long. The parallels are strong though. It's a lot like writing an album - as an album. These days, songs are more individual pieces. But when you thing about album, and you'd think about how to approach it as a whole... I used to be in this band called Building. Everything we did was not conceptual art, but it was really thought out, from the sounds to how the listener would take it in. I played guitar, bass, string and horn arrangements... most of the music. I don't have a classical background, but I had my weird way: write everything on guitar, then sit with the string players. I would write everything with a 4 track. That's the advanced technology we had when I was a freshman. That kind of thinking about things, pouring over every little detail, how your audience is going to be impacted by it, that's exactly how I think about design. 

I think of Snazzy FX as an art project. It's my art. These (points to his products) were a statement. The original website was very conceptual. I want everything to fit and work as a statement. We have a section online called allies - it's stuff we think is cool, and I think that's sort of how I approach the products too. The thing that makes me most excited, and that makes me know I'm doing my job, is when people write to me and say ``this product is making write music differently'' or ``I'm thinking about music differently because of this thing''. In that respect its a lot like writing music. The design aspect too is really intuitive. Spending months and months researching and looking into an area and then pouring that all out at the breadboard or the simulators. Writing the product. 

Do you use simulators? 

not so much. There's a couple realtime ones for the ipad that are fun when you're on the subway. The wavefolder - the Tidal Wave, my own design - came about in icircuit... it's a java app. I haven't messed around with SPICE too much. I don't put that much time in the simulators. They're still novelty to me. When I designed that wavefolder, I got really excited, printed the circuit from the simulator, said ``this is going to make design so much faster!''. But that's not what happened. I took that to the breadboard, and it worked like it was supposed to. But because it was on a screen, and I couldn't really interact with it, once on the breadboard so many other things happened and I spent almost a year on that thing. Became a lot more than a wavefolder. 

But I know a lot of people have used them. I'm still very hands on, breadboard kind of person though. (25:25) I like being at the bench. I always listen, I always have a speaker on. That's the problem with simulators, at least the ones I use. You can't listen. If you hear it, you're putting a sine wave through it and you hear how it changes, but you're not actually making music with it. 

What do you think of the distinction between musician and engineer or fabricator in cases where the design of an instrument is integral to the development of a musical piece? 

I kind of view it as a different medium. In high school I really like video, and then synthesizers... you find different mediums and then as you're presented more technology you try different things. Sculpture, painting, pen and ink... having these tools at my disposal is just another way to explore and create. I'm in a position where it ended up as a company, but in my mind I'm a company that makes sense to me and I believe in. I probably would be just as happy getting grants and doing research in an academic environment and doing stuff for art shows. It just sort of happened one way. 

You're making analog, real-time systems for composition of performances. That comes with a history and expectations, limitations... how aware of these are you? 

I guess the ardcore is a good example. You have limitations from the 8 bit DAC. How fast can we get this chip to go? I'm really happy how the routine for the DAC came out. Even for the drums, it sounds really good for 8 bit. But in general, analog - or pure analog - yes, the rails are +/-12V, no high voltages, but... there's a wide wide range of stuff to do there, and I don't feel too limited. Sometimes parts, especially through-hole parts, things get obsoleted all the time. In the past I've seen more than a handful of parts I've used are no longer available. I use SMT for manufacturing but I use through-hole for design and prototyping. It can be a little scary. Analog isn't going anywhere, it's always going to be needed to transfer the digital to the analog world... but if the part doesn't sell enough, they get rid of it. That's one limitation. In analog, you never know what's going to be taken off the market. When I started making synths I wasn't really set on a format, I just cut a piece of metal and sticking it in a rack. I think some of my old modules are +/- 15V. But anyway, for me modular is such a great format. It's so open and encouraging experimentation. My original thinking was that you could really put stuff out there that's different. Whether or not that's true is something you can debate, but I think it is. It also wasn't such a sure thing back then. There was MakeNoise and there was Harvestman and there was Bubblesound... some americans manufacturers. But now it's everyday, I just saw someone released 13 modules. It's crazy how many companies are jumping into it now. 

What do you think of Synthrotek selling everything as a kit? 

I actually stayed in a house with him at NAAM. Kits were one of the ways... well not the whole kit, but I used to get circuit boards from ken stone, and electromusic, ian fritz, people like that. 4ms sells kits too. I think kits are great for people who want to get involved. not everybody wants to learn how to go all the way, but they also want to be involved in soldering and making stuff. It's great. I'm actually thinking about getting some to build with my kids. I wouldn't want to be making kits though, I'm so disorganized - putting little parts in bags sounds like a nightmare. 

Ian seems like he has a thriving business, and he's doing a service to people... they seem to be reasonably priced, too. 

They are! I think that's one of their great things. What do you think about the fact they share all their schematics and designs? Is that something you would do if you didn't have to bag all the parts for kits? 

Yes and no, it depends on the project. Some of the things I've spent a year developing, no. I'm being honest. People don't put that in the equation. If you've spent a lot of time on these things, not opening a book and taking things straight out of it, you're developing things. I've thought about putting the miniarc out as a kit, or as something... I've wanted to post a few more schematics. I've put a few over the years. The problem for me is all my schematics are hand drawn, and even if the schematic can be useful to somebody, if its hand made, people will complain, and that made me think ``nevermind...''. So yes and no. I know Ian Fritz wasn't too happy at one point because he designed a new type of through zero filter, and shared the schematic, and somebody turned it around and made it into a product. Wasn't happy. Another Eurorack company started selling it, and luckily the community sort of policed that. People said hey this isn't cool, why are you doing this? But especially now, when Euro's getting so darn competitive, it actually feels competitive! It's amazing, it always had this real community feel to it, but a lot of bigger companies are jumping in, and a lot more of potentially people looking over each other's shoulders. 

but most of us in the community we share stuff all the time. We trade part sources, help with designs. Mark Verbos is one of my good friends, and David runs bubblesound, and he's staying here... there's a lot of helping. I don't know if that's going to change as the market evolves. NAAM was weird this year. 

Too many people? 

not even that. just a lot of big companies getting in. Dave Smith, Sequential Circuits, there's talk that more and more people are jumping in, and those people aren't going to be in it just for the love. They're going to be in to make money. You'd be amazed to see how many of us don't approach this with a money first attitude. I just want to pay my bills and make cool suff. My products are not designed to sell a million units, obviously... 

How have proprietary tools and designs affected your work? (35:49)

Probably not too much. The audio arc, for example, had an SSM40 filter chip in it. That's a filter chip you cannot get anymore. It's hard to find. Schematics aren't available... the patent is, but not the schematic. So that was case were going from the hobby level, one at a time, to manufacturing, you do have to make those choices. It hasn't been much of an issue for me, I usually try to go as discrete as possible. Maybe not always the transistor level, but I would never make a product that uses an all in one chip. Even if it means more parts, if you can do it with some that are going to be around for 20 years, you're much better off. 

But you know, I think this is something that the internet has changed for so many people. If you talk to engineers from the old days, they'll tell you information was scarce. But that was good, it made them come up with solution they couldn't have had otherwise. I think that when everything is available, we assume that to be the case, and then you find out ``oh they're not going to share the schematic, what am I going to do?'' that can actually be the best blessing possible, because you have to come with a different solution. Most of what you need to find, you will find. There's not really any specfic reason why you need to know exactly why I do something in a box. You can find most of the general stuff out there. I think that people starting electronics nowadays have it real good. It seems like more and more people are even starting out doing hobby stuff with SMT. and if you do that, you're in great shape because mot things are still available in that format. 

But the short answer is it didn't affect me. 

You talked about planned obsolescence, how does that work within your designs? 

Euro is made to last a long time. That's what's specific about this market, and stompboxes too. We're nto building this stuff the break. It's meant to last a long time. A lot of the newer laws in Europe about how something is supposed to be recycled (38:58) are interesting because we don't see these as things to just be thrown out. If you look at vintage synthesizers, right here, there are three that are really old. `79, `82, early 80s, late 70s, they all still work. With the exception of my sequential circuits, which does use oscillator chips, you could open them and repair them. The Yamaha, half of that thing is metal, it'll take a beating. I want my stuff to be life that. I want kids 20 years from now collecting it. 

ET: opening it up, fixing a part. 

DS: yes, exactly. And as I make these, that's something I've become more aware of. Making sure that parts are going to be around. And most of my associates who make stuff make it really well too. 

ET: Could SMT be viewed as a hinderance for people trying to open up your products in the future and fix or mod it? 

DS: it's a hindrance for me! I mean, yes and no. That's one of the thing about this planned obsolescence thing, large companies have made the assumption that since it's so small and tiny no one is going to fix it anyways. It'll just get thrown out. But I see SMT more and more, and even in kits. Turn the board over and all the parts are right there. 

Anything anyone has to do is send the thing to me. That's the first thing. Not that many of them come back for repairs, they don't break that often. When they do, they could take it to a friend who knew electronics to repair it. I think it's not more of an issue than it would be with... 

ET: Through hole components? 

DS: yeah... I mean, even if this was through hole, they wouldn't be able to know what was doing what... they'd still have to ask me. On the whole, there's a lot more people that are willing to find that stuff out. There's more kit people and nerds in Euro than there are in any other market place or industry. I sort of assume they'll figure it out. I still deal with repairs, even on these.

ET: How important have other people been in the development of modules? 

DS: The first thing is the forums. I mentioned electro-music. I wasn't so into muffwiggler back then, but it's great for people now. That sort of community, to learn about electronics, or even part sourcing, there's a forum called the mostly modular trade association, the mmta. It's for dealers and manufacturers to trade information. That wasn't there when I started, but it's a resource now. 

Back then, it was also Dave, who runs bubblesound, and Mark Verbos, who runs verbos sound, they were local. He heard about a company, then I would hear about it. Everybody was pretty forthcoming. Bill from WMD, he'd say ``oh you guys have extra of these, yeah we can sell you some of those''. So there was a good exchange. I never felt like anyone would refuse me anything. It was hard in the beginning though. Finding someone to do PCB stuffing, contract manufacturing... That's probably easier now. I had to learn the hard way on some stuff... 

There's more Euro centric manufacturers now. I don't mean people designing the boards, I mean people stuffing the boards, testing them. 

ET: you get your modules stuffed and tested? 

DS: I get the boards populated. I have kids, I don't want my house to be a factory. I'm also the type of person that feels like I don't need to have my hand in the manufacturing at all times. I'm much happier about design than I am about soldering 100 boards. And the community was really helpful in that process. There weren't as many dealers, but Sean from Analogue Haven was extremely helpful with a lot of people. The community has been exceptional, a lot of people helping each other out. 

ET: how is it today? do you teach newcomers how to get boards stuffed and finding distributors? 

DS: Just last summer, a new company was making the transition to euro, and they got my number, so gave them my sources and said  ``yeah, go here for metal, there for this, try this guy''. I'm not sure I would share that with larger companies. Control, the dealership? Is a perfect example. It's a hub. You're there to buy something, but you'll run into another designer, or someone who just bought your module, or they'll tell you ``someone was just asking about you'', it's great!

When I do a talk there, everyone brings beer or has beer, it's really nice. I love that I know my dealers on a first name basis. I've made a lot of good friends, and I want to encourage that as much as I can. I think things will change. But it's been great. The spirit I was treated with in the beginning with some of older guys like Ian Fritz and others from my generation - I want to keep that going. But I'm also really busy, so I do what I can. 

ET: Do you feel like there are multiple levels of community? I feel like there's the greater online platforms, people who speak the same language as you, then there's the friends and the people close to you...

DS: There might even be more than three levels. When I go to NAAM, or knobcon, or even CVfair, we have our own thing here every year. Those are people I see once or twice a year. Dan from 4ms, or Scott from harvestman... excited to see them, there's a bond there. And then there's the local community, people who I see a lot. My company attracts people like this... there's a level of education from people in arts and people in science that are reaching out to me and saying ``look, here's what I'm doing, your stuff is relevant''. You could call that people in the scholar community, or people in the crazy art community, but they're reaching out to me, sometimes for no other reason than establishing contact. That's really nice. It's not just where to get knobs, or how to fix that, it's about talking about ideas. 

ET: How was working with Darwin from C'74?

DS: It was great. We're still in touch a good amount. He interviewed me for his podcast. He was going to write a manual for the telephone game, because he noticed I was just sending text files to people. He said maybe I'll make you one. He gets one of every module, we still talk and I see him at NAAM sometimes, he's awesome. It was really fun working with him. I like this exchange of ideas, as I'm a bit of a hermit. I'm a family man, I work all the time, I don't have a lot of time to go out and socialize. When I do get to exchange ideas, it's important. Ideas are where the products come from. What takes the most time is working through the conceptual parts. That's why I get so many books. 

ET: how close do you feel to avant garde or experimental music? 

DS: Do I like weird avant garde music? Yes. Is that the question? 

ET: Rather, how close do you feel to the people in that milieu?

DS: I have mixed feelings about this. Some of the music I make is extremely poppy. Not sugarcoated, but it's songwriting. Structure, melodies, etc. But then I've also done a lot of experimental. But I don't tend more towards one or the other, I think that my initial work with modular encouraged me more to go in the experimental direction, because I like how free form and in the moment it could be. The people who do whacky stuff was a big base of people using this stuff and encouraging it, but I think that one of the problems there is it obscures the fact that other people can use too. 

That's why I try to champion both. Obviously my stuff is pretty weird. A lot of people make the mistake that you can't do anything else with a modular. That not true, I've used it live, the telephone game was developed working with a live drummer. Trying to do something that could improvised structure stuff with multiple melody lines that were related. I love experimental music, and I encourage it. The stuff I make so that it's powerful for all types of music. Euro's at a point right now where because there's so many manufacturers and dealers, it won't continue with just people doing experimental stuff. The most important thing right now to keep developping it is to tell everyone hat they can use it. It's like clay, you can use Eurorack for anything. People see wires everywhere and they think this is a mad scientist thing.

ET: You were mentioning Nic Collins. How close do you feel to circuit benders and reusing electronics? I think this really speaks to that openness in synthesizer making...

DS: I feel close to it in that that's where I got my start. There's multiple levels of circuit bending: some people don't want to know what's happening in their electronics. They don't want to know what's going on. I'm different. I want to learn and understand as much about electronics as I can. I love that stuff. At the same time, his book is a great example of trying to open this field up to anyone. I think that in the very beginning.  

DS: I also made stuff like Crackleboxes (picks it up). The battery might be dead (box turns on, makes sounds). This is in the middle. You've gotta solder it, look at a schematic, etc. but it approaches a lot of the circuit bent sound. For me, modular became so exciting because of generative stuff, because of the organic system of modules that can create music on its own. You can guide it, or it guides you, but there's an interaction going on. I'm not determining every event, point A to point B... I'm going to turn it on, and set something up, and something's going to come out. If you do it right, or if you do it wrong, it can keep going and going. Those unexpected possibilities, for a musician, that's gold. Having things you can improvise on, on be inspired by, or have it be going in the background, as part of your environment, that's what's exciting to me. I didn't get that so much with bending stuff. 

but bending did have the excitement of getting your hands dirty and learning a bit, and jumping into it. Some people are perfectly happy with that, that's great. 

I wanted to go as far as I could. In ten years, I hope the stuff I'm doing will be even more toward that weird goal of making these systems that are designed to do that weird stuff. That's not to say you can't use them for any other thing. The telephone game module, if you haven't spent time with it, I works great with the chaotic modules. that's my sequencer. You have some control over how it comes out, but it can do 4-5 melodic lines at once and works great with chaos. 

ET: Have you been doing more custom or comission works? 

DS: Yes, I have one right here! This is a new attractor I discovered, using an inductor. And I'm making something for someone up in Berkley, as well. A chaos thing. This first divine hammer is for a customer. So yes. I don't have much time, but I do do it. I've been getting these custom made wooden boxes, from a friend... I have some customers who seem like they buy everything I'll make. Sometimes I just have to pay rent. Right now, I'm just waiting for the Tidal Wave module to ship, until it does, there's no money coming in, and money needs to come in. So that's a chance to make something custom for someone. I'll contact a few people, and usually they'll say yeah, sure, how much is it? That's great, it gives me a chance to try things. This is one of my goals for the next year, I don't know if I'll have time for it, but I want to do smaller runs, maybe 20 instead of 200. I'll be able to do stuff that's more out there, and will appeal to a specific type of person but not everybody. 

If I can pull it off and get the manufacturing right, it'll be great. It's a lot more work. When I make something custom, I'm very perfectionist, it's never a rapid process. I feel bad - this custom divine hammer has had 4 circuit boards in it. I'll make it, say no this isn't right... email the guy and say sorry... it's going to take a while longer. 

The custom stuff is fun. There's way more ideas than time to get the stuff out. I have a lot of stacked up ideas, hopefully in the next couple of year this stuff will start coming out. 

ET: any last comments?

DS: I always tell people, don't be scare of my stuff! 

\section{email exchange transcript for John Bellona}

1 - your approach to hardware/software combinations seems really practical, but my sense is you still explore the capabilities of each framework or medium. Do you look up to any artist or engineer that might have done that particularly well as a template for your own work? 

When I first got into electronics, guitar pedals, digital music I loved engineer/producers. Dave Fridmann. Nigel Godrich. Tom Dowd.
  Later on, I was exposed to acousmatic composer Scott Wyatt, who applies a lot of engineering concepts to music that just make his music sounds really crisp and his trajectories are very clear. The sound is very important. This fact is essential for me to remember especially when having to consider gesture and technology in the same sphere. Its easy to undercut sound because you run out of time working on the other components, but sound is why I'm doing all the other things in the first place. So why undercut the reason I'm here?  While I am continually exposed to new works that I love (John Cage Four2 (SATB), David Behrman On The Other Ocean), I attempt to come back to the sound as helping to define the work. I will make decisions to support an idea or concept, but the resultant sound should always work with my ears. I hope this makes sense?

2 - one of the interesting challenges of teaching electronic music to me is the variety of interests and experiences within a given student body. How do you feel hardware and software facilitate that catering to individual personalities? Is one (hard or software) favored over the other at the academic level? 

Software is a pedagogical conundrum. Teaching concepts is essential to cutting across technologies, but then there is just the ability to work with software in general that requires the technical.  I've taught both ways, and with both, each cater.
Here are a two examples of success and failure (rambling)
 I've taught Photoshop as an in-depth technical course. (http://deecerecords.com/artd360/) I've seen previous instructors teach art concepts through the software, which didn't quite satisfy the entire student body (students from product design, architecture, painting, journalism, printmaking, sculpture...). I think it was important, for me, to ask why are students taking this course? And how is the course offered in the catalog?  By teaching the nitty gritty of how-to inside Photoshop, I witnessed non-technical students able to understand how to work with a digital technology and make work they were happy with, discovering solutions on their own. I also witnessed more tech-savvy student jump over the differences in Illustrator or other graphic/image software since they learned to understand what they were looking for. I was happy with the results in this, even though I was getting really into the how-to of a specific software.
	On the flip, I've taught Processing as an in-depth technical course (this listing is when I taught the lab that still included Flash (\url{http://deecerecords.com/artd252/#processing}), and I completely failed at it. While programming can be for everyone, the way it is taught will definitely cater to particular personalities. The lab I taught Processing in used to be Flash animation, but we slowly switched over to Processing only since the application of programming became more important than the industry need for Flash savvy designers (and Processing is free). For anyone who hasn't learned a programming language, and isn't into the idea of it, the challenge can be, at times, overwhelming. You don't need a lot of programming to do interesting work, but perhaps that was my fallacy in teaching the course that first time.  I thought people had to know array manipulation and Objects, but one doesn't.  This one should have been more exploratory and project-based.
	The software battles rage on...
	pure Data vs. Max/MSP.  
	Processing vs. Flash vs. openFramework vs. Arduino.  
	RTcmix vs. cSound vs SuperCollider.
	Logic vs. GarageBand.  ProTools vs. Logic.  GarageBand vs. Reaper.
	
	Do you have students buy a software? Or do you use open source?
	Do you teach how to use the software? Or do you ask larger questions that use the software to answer them?
	Do you use software that is for Mac or Windows? (I consider anyone using Linux as being independent and technical)

	Certain software hide concepts that are necessary to understanding both the analog and digital domains, and for this type of reason I stay away.  GarageBand is a huge no-no for me because the software doesn't allow students to understand how signal flow works.  Dynamics and Effects processing are applied in the same way in this software, but that's not the case in 95\% of other digital DAWs and 100\% of the analog world.  FX processing requires busses, and dynamics processing is placed in line. Is the software free? yes.  Is it already installed on every Mac and on every lab machine? yes.  Should we use it?  No... 
	People will also teach and use what they know. I've seen undergrads taking a RTcmix course, and I've heard gripes about learning an outdated software. 
	I think any software is fine, so long as you either address the needs of the students or are extremely clear about the course will be about (conceptual vs. technical, etc.)

3 - Do you feel like academia / institutionalized art is representative of the overall community (or even just your community) that operates between art and engineering? 

Do you feel like academia / institutionalized art is representative of the overall community (or even just your community) that operates between art and engineering? 
	I've found that collaboration is a necessity if we want to elevate our capacity to work with multiple disciplines (art, music, engineering, architecture, dance, to name a few) and to create exciting work. We almost no longer can do it all on our own because of the knowledge required.  If we can prepare students to ask questions, reach out to others, and establish collaborative relationships, everyone will benefit.  This type of learning also extends outside of the classroom into real-world problem solving, and helps situate the necessity of art in the eyes of an administration who may perceive music as only being an acoustic instrument one plays to become cultured in Classical/Art. (yes, capitals here) Electronic music has similarities with today's community because of the artistic and interdisciplinary process that is involved even if the the end results may sound much different than a listener's expectation from a piece. 

4 - what's your community like? What's the place of sharing at the local / international level there? 

I've really enjoyed the Kyma community, which is a small niche in the larger world of electronic music. I enjoy the community because it's comprised of academics, sound designers (film and video games), musicians, composers, video artists, and more. Everyone is very humble and open to exploration, collaboration, and dialogue.  While I completely understand that the hardware locks out many practitioners, the community contains many individuals with whom I want to have artistic conversations with.

With regard to academia, I see my work and process paralleling that of art and digital music departments than traditional composition. This probably has to do more with the ideas of exploration, collaboration, questioning, and openness. I've also learned quite a bit from working with Harmonic Laboratory, the art collective I am part of. Working between disciplines informs how I approach a project, the connections I see and want to explore, the type of vocabulary I use, and the work I create. Collaborating with Harmonic Lab has allowed a work to take shape rather than to oppressively mold the container.

\section{email exchange transcript for Martin Howse}

Could you start by describing a little bit your approach to
installation / performance hardware? How did it come to be? What's the
place of physical items in your work today?

I think that my approach comes from an obsession with the material
basis of digital and communications technology - an awareness that
these are energetic processes which may be blackboxed, abstracted or
plain invisible but which have a strange, dynamic materiality which
cna be played with and investigated through hardware. So the approach
is very much a revealing, either through re-working materials towards
technology (for example, performances using earth as an active,
electrochemical, biologic material), or dissecting and almost
dissolving (in chemical sense) digital technology (in workshops), or
devising software which examines its own material conditions (for
example, Island2 installation). This approach perhaps came to be from
early experiences with DIY computers in the 1980s, building bits and
pieces, and programming in a very direct way (instruction by
instruction, looking up operational codes in tables and writing these
by hand before they would be input on a simple keypad). I have always
been obsessed with this peculier place where code operates in the
world, this strange, hidden place where something happens which can
then become or is physical and audible. So physical items are very
much present in my work today, self-made electronics and material
combinations.


Is your interdisciplinary practice (art, technology, teaching) means
to achieving a unique goal, or different ways to express thoughts and
interests?

These are different ways to express my interests; I don't think it
would be possible to enter into these conceptual concerns within one
practice or discipline - for example I began working mostly within
video and conceptual art and realized that technology was an
important material concern for me; teaching as in workshops is a way
of generating new ideas for myself and others.


What level of complexity do you look for in your electronic instruments? 

I hardly ever use electronic instruments other than those I have
built myself and some of these I have had huge problems in trying to
reduce the complexity - this has been the hardest work, how to map a
vast mind-set of connections and processes to a simple
interface. Working with materials presents a simpler interface.


how do you approach the use of new devices in your setup? 

I would say that new devices are suggested by the setup - the
materials and concerns force new ways of thinking or doing things -
for example, I started working with modulating sound in light very
physically and from here I've been looking at crystal modulations of
light, and devices which could bend or twist light.


how have proprietary designs, tools, and planned obsolescence affected your work? 

Proprietary designs and tools have made things harder on a practical
level. People share interesting things created using proprietary
tools, under proprietary licenses and this clamps down knowledge and
approached to certain problems. It feels stupid to limit aesthetic
potentials and equally force people to re-invent. On the other hand,
most proprietary work is not so interesting aesthetically, just
through the mindset subscribed to. Planned obsolescence forces me to
upgrade and waste time.


What have the responses to the detektor and dark interpreter devices been? 

The detektor and dark interpreter are quite different devices, but
the response to both has been interesting - a few people don't really
understand the thinking behind them or how they worlk, but most have
been very appreciative. I like the idea of these potentially mas
produced devices as art works, as spreading some ideas and ways of
working like a virus. I think they work in this way rather than as
practical instruments.


they are both hardware/software combinations. is that a practical decision? 

The detektor is purely hardware, so no software as it more or less
directly translated electromagnetics into audio across two wideband
frequency ranges. The Dark Interpreter is both as it is really close
to this exploration of the material base and tangibility of code or
software - I wanted the user to literally put their fingers into the
code, to run the code over their skin.


how important are other practitioners in the development of that work? 

The detektor is more based on others work, practically and also nin
development with Shintaro Miyazaki. There's also a whole community of
artists and tinkerers working in that area so it is pretty important
to look and work with others. For the Dark Interpreter this is pretty
much a personal work and struggle!


how has teaching affected your creative interests? 

A lot of the phenomena I work with, electromagnetics and say
radioactivity have grown out of teaching and workshops; again,
generating and sharing ideas, growing interests.


in your experience, who does open source / affordable hardware empower the most in the arts? 

I'm not so sure about empowering (particularly within the arts) as I'm
uncertain hardware or software platforms really do this; rather the
general approach is important. I think there are not so many artists
empowered by open sources.


Is there any recent technical development you're particularly excited about? 

I'm attracted to what could be achieved in a DIY sense within quantum
computing. I'm more interested in what becomes accessible from within
technical developments - I need to be able to make these
technologies, to play with them.


what is your current professional community? Do you feel close to it? 

I guess my community is within new media art, crossing into sound art,
and the noise or glitch scene. So it's a bit of a spread-spectrum
community. I feel close to other artists working in a very direct way
with technology (say Peter Flemming in Canada, Bengt Sjolen, Ralf
Baecker, a long list of others), and synth makers like Peter Blasser,
although I wouldn't call that a community as is very diverse and
spread out.

How close do you feel to avant-garde or experimental music traditions? 

I feel closer to avant-garde literature than avant-garde or
experimental music. I don't feel so rooted in a tradition of
experimental music although I enjoy listening to work from within say
musique concrete tradition; it's not something I relate to so easily
outside a live or living context.

How has your awareness of those traditions influenced your hardware or
text-based work?

Literature has a stronger influence on both sets of work. I like to
think of how literature can influence and change hardware - how this
could become a new way of working with what seems very engineered or
given as to how it should be used. That's the challenge, for example
with the Dark Interpreter and other synths how to say insert the
influence of Edgar Allen Poe directly into the software and hardware.

\section{email exchange transcript for Jessica Rylan}

could you start with a brief description of your hardware and music backgrounds, what you are doing today and the important steps in getting to that point? 
what is the current place of hardware design in your practice? 
what parallels do you draw between technical processes and other disciplines or hobbies? 
In what way does scientific research trickle down to influencing your electronics and the music you make with them? Would you say there are parallels between the processes? 
Are you familiar with Hannah Perner Wilson’s kit of no parts? As part of her Master’s Thesis at MIT, she devised this set of objects that could be built using some homemade elements - conductive ink to turn seashells into speakers, conductive thread to make electronics in wearables, etc. Would you consider using these methods for electronic music? 
The personal synth seems to be a good example of what can happen when the artist/engineer personas merge: unique \"machinstruments\". You've described it as offering unexpected possibilities - a device with which you have an emotional relationship. To what extent do you feel like other electronic musician make a setup their own? 
What would electronic music be life if every one of them had their version of the personal synth? 
When designing audio hardware, how do you approach challenges and limitations?
How have proprietary designs / tools and planned obsolescence affected your practice?
You've questioned the scientific approach to sound in favor of embracing a more chaotic, unpredictable method. How do you reconcile that with the limitations of analog electronics, and how does your current music setup reflect that? 
How important have other people been in the technical development of your practice? 
You've described your interest in electronics as coming from a very personal place, from your grandfather and popular electronics found in your old house. How has that personal connection evolved and influenced your design work? Has anyone taken the place of those sources of inspiration? 
What is your current professional community like ? 
To what extent do you identify with a tradition of use, re-use, misuse or subversion of technology for the arts? 
Do you feel compelled to help people with the same design questions you had when you started? 
How has musical training influenced your instruments and musical practice? 
Do you perform with anything other than your own instruments? 
How fluid is the transition from circuit design work to music?  

\section{email exchange transcript for Bonnie Jones}

Bonnie Jones Interview Questions 3/15/15

Could you start by describing a little bit your delay \/ mic setup for performance \/ composition? How did it come to be, and how is it important in your work today? 

My current set up varies but usually I have the following possibilities: 
3-4 delay pedals fed input to output and played by touching the back of the pedal with an instrument cable plugged into my mixing desk. 
Samples bank accessed from a computer using a freeware theater cueing program 
Telephone pick up microphones used for electromagnetic sounds from computer or mixer
Sometimes an assortment of antique or regular microphones for different “voices” 
Transducers “exciters” for sending sounds into stationary objects that I usually find in the space – cardboard boxes, metal or plastic bowls etc. 
Two different types of contact microphones one is a regular cheap piezo and one is a higher quality piezo that is embedded in a block of wood to create an amplified surface. 
Old Radio Shack (RIP) cassette tape recorder
Shortwave radio – used for radio sampling and also for playing the digital delay pedals through. 
The delay pedal is still one of my primary performance instruments – meaning if I had to play a gig with just one element of my set up – it would almost always be at least one pedal. I could say that it’s the one part of my set up that feels closest to a traditional instrument like a sax, or a violin or something. Though the other elements in a way create a larger modular instrument – the pedals do have a singularity and for the most part allow me a complete range of musical expression in any given music situation. 

What parallels do you draw between technical processes (like assembling that setup) and other disciplines or hobbies? 

I like to say that my set up came about like discovering a language written on a cave wall. As a musician I always approached playing as a way to understand the basic structure of that language as well as the process I can use to learn how to communicate with that language. I suppose the word that could be used is an intuitive approach to understanding the technical aspects of my instruments and their musical possibility. 
In that way my instrument sometimes feels like a poem. My first creative discipline was writing, so I draw a lot of parallels there both in the instrument and also in my compositional ideas. I’ve always been really interested in what happens when the words speak for themselves in a poem, when they do things that you can’t foresee, when one word placed next to another word become something that you can’t determine. My goal then was to recognize that thing and then try to understand that thing. With my musical set up it’s similar to that, I put different elements in tension with each other and they end up revealing to me a bit of their language.

What level of complexity and indeterminacy do you look for in your electronic instruments? 

I like when an instrument is indeterminate but there is also a desire to be able to play something – so yes while the pedals have a certain level of indeterminacy – playing them for over a decade has really allowed me to understand many of the ways to produce specific sounds. As for complexity – I use tools for the sounds they make vs the nature of their construction or code. I like getting to know an electronic instrument, and I appreciate when that instrument has some surprises or enables me to create and discover sounds that I wouldn’t expect – but I wouldn’t care if something was complex if I didn’t like the way it sounded. 


How do you approach the use of new devices in your setup? 

I just throw them in the mix and see how they play. Sometimes I will introduce a new device, or object (rocks, bells, lights, etc.) without thinking about how I will use it in the concert, but including it on the table almost as a way for it to instruct me on how it might be used in the performance. I like having things on the table that aren’t used – I like bringing outside objects to see what they might do when I’m in the middle of set, how I might decide to bring them into the music.

How have proprietary designs, tools, and planned obsolescence affected your work? 

It’s harder for me to get pedals with the kinds of circuit boards that make for good bending – design of these boards has really minimized the elements leaving much of the board sort of blank. 

I usually work with freeware/shareware software and they are usually unsophisticated so I haven’t yet run into the problem where my software doesn’t do what I need for performance or requires any expensive upgrades etc. For performance I try to stay away from anything that might crash or misfire or require proprietary software. I had an experience in Oslo where right before a concert my computer stopped working. The entire piece was based on having these samples I had carefully composed to be used alongside live electronics. Because I was using all freeware and easily downloadable software, I was able to just switch to another computer and set it up for the performance. It wasn’t perfect of course but it was easier than other situations where the work is more dependent on expensive software etc. 

How has teaching audio electronics affected your creative interests?

I take the approach where teaching and working with my nonprofit TECHNE is part of the continuum of my entire creative practice. It’s been important and freeing to unbox the areas of my life that would seem disparate and try to get around the cause/effect/influence categorization. This isn’t exactly what you’re asking but I do think these days it’s important to democratize aspects of what a creative life actually looks like. That said, I spend a lot more time these days putting my creative process into shareable language so that I can transfer that knowledge to students. It’s tricky to explain how to do intuitive electronic music exploration. I never had to quantify it before and now I approach my practice with a different layer of explication always in the background. This has made me more interested in writing about music. 

In your experience, who does open source / affordable hardware empower the most in the arts?

Easy access to free tools and cheap hardware plays a key role in bringing in new ideas and new perspectives from less visible artists. I love a lot of the art that is being created today, but sometimes I feel like it lacks imagination – it seems to come from a specific perspective/position and I am invested in finding ways to open that up.  
I still see DIY culture as pretty critical. Even in its co-opted and packaged state, being able to make shit with whatever is available to you is pure improvisation. I appreciate that and seek that out. Sure, as we get older and our art develops we want our vision to expand – and that requires more support – but we know that support gets doled out in strategic ways and often with strings visible and invisible. I still believe that the most radical shifts happen when artists without the benefit of institutional, financial, or technological supports, just make things happen.  

Is there any recent technical development you're particularly excited about? 

I’m intrigued by ambisonics though I know very little about it aside from the lecture I went to in Stockholm at EMS. 

What is your current professional community? Do you feel close to it? 
I work with a lot of different writers, artists and musicians in the US, Europe, Asia, and elsewhere and it’s amazing to be connected to so many different minds. Recently I’ve been drawn to working with activists and social /racial justice groups – I feel like that’s closing an important loop for my own selfhood and art. 

Do you identify with hacking / bending / re-use / misuse of technology for the arts? 

Definitely – the technological arms race in the arts is tricky. Artist as R\&D for technology corporations is a real thing. Subversion is still the place where those other voices can be heard. 

How close do you feel to avant-garde or experimental music traditions?

A tricky question because even though I didn’t study music in undergrad, my creative education outside of academia is still driven by an awareness of and direct influence from those historical canons – specifically the electronic musicians of the 50/60s, John Cage etal. My creative trajectory from an outsider’s perspective would appear to be very linked to those musicians and artists, however I begin to feel more alienated from this history.The question I have these days is what does it mean to be in dialogue with these artists? What’s the relevance of these questions for someone like me – and by that I mean a women of color? And in a more critical way are the questions these artists were asking so specific to a person’s race, privilege, and class that they will naturally exclude so many other vital questions. I’m a 37 year old artist who has been working with electronic music for decades and am still interested in finding historical role models who are asking questions that seem to relate to who I am, where I am, and my experience of the world. If we believe that we are not influenced by our artistic education we are lying to ourselves, if we believe our artistic education is the truth about value, relevancy and validity in art we are lying to ourselves. 

How has your awareness of those traditions influenced your hardware or text-based work? 

These things are important to me at this point – they may or may not be traced to historical art/philosophical antecedents “discovered” by the artists I’ve read about in art history books. The object has a life – the object pushes you and you push it. You can make things that you do not understand at all and maybe later you figure it out and maybe not Abstraction and improvisation is about hiding and revealing the self at the same time – because of that it is also about resistance 
I care about communication but there are so many ways to communicate and so many languages in which to do that

I am deeply skeptical and suspicious of what is visible. 





\section{email exchange transcript for Louise and Ben Hinz of Dwarfcraft and Devi Ever FX}

what are your musical interests and training? 

Louise was trained in classical and jazz in high school and college, playing
the upright bass. I'm self taught, which goes a long way to explain my “style.” 
We're both sort of “bass heads” interest-wise, which is why we really focus on all 
our gear sounding good with low-end instruments. It's a very important part of a 
band for us. In my case an otherwise great band will be written off if the bass isn’t 
there. But we aren’t narrow in our musical interests.

do you see any parallels between designing a circuit and playing guitar? 

Not really. One is a task with an end goal and a required outcome. Playing 
music is freedom, love, and life.

both of the companies you're involved with are popular in the pedal market. What made you go from basement experimenters to a business, and how did that influence your circuit design work? 

I was in need of new sounds, and I didn't have money to buy any pedals, 
so I began experimenting with modifying really cheap gear, but soon after I 
learned some basic building blocks of audio electronics and started down my 
own path.  Some friends online heard the terrible demos I made and I sold a few 
that way.  Analogue Haven and 9Volt (from Japan), where the first dealers to buy 
our stuff. 

Some of my early work came from using components incorrectly to get  weird sounds that were not available elsewhere, which is still really important to 
us. We decided a few years ago that if we couldn’t do something new and 
interesting that we loved, we wouldn’t do it at all. 

the Pitch Grinder is your first commercially available digital product
(unless I missed something in the Devi website / history). How was
that foray into software / hardware combinations, and what prompted
you to do that? 

Digital is the future. It's also the now. Most of the ideas I currently have 
could only be realized digitally. I think there are tons of great analog circuits, and 
I will use them forever, but for me I'm much more interested in pursuing digital 
audio processing. We got started on the Pitchgrinder when I was introduced to 
Bob Lowe, an engineer here in Eau Claire. I threw a boatload of ideas at him, 
and we kind of sussed out what was doable from there.

how traditionalist is your customer base? 

I don't think, on the whole, our customers are very traditional. I'm sure we 
sell some of our more “normal” stuff to more “normal” people here and there, but 
most of our customers are on the fringe. When we asked for song submissions a
while back there was a ton of stoner/doom tracks, some noise, and some really 
trashy, noisy pop stuff.  

Does that conflict with your musical interests? 

Not really. We are pretty eclectic at the shop. Sometimes all I want is Tom 
Petty, some days it’s Roni Size, some days it's Monolake. The older I get, the 
less genre specific I get. I try not to rule anything out before hearing it. One of our 
henchman has us listen to the new releases every day. Like, all of them. Usually 
we just talk shit on everything, but it got me up to speed with FKA Twigs, which is 
not categorized in a genre I usually look into. 

Dan Snazelle was mentioning that Eurorack has become more popular than
when he started Snazzy FX. How did you come to start selling modules,
and was it a significant shift from pedals? 

I have wanted to do modules forever, I was just kind of waiting to make 
some time to learn how! I really love that instrument/system, so I'm just glad to be 
a little part of the Machine. 

There is a pretty exclusive attitude, and some really picky motherfuckers in 
that scene, much more than the rock type circles I tend to run in. So I'm making 
an “Intro to modular synthesis” video series, to make other folks' entry easier 
than mine.  Lots of it is beautifully complex, but I would much rather the scene be 
more inclusive and welcoming.

modules also put you one step closer to experimental / avant garde /
contemporary classical music. Any opinions? 

I don’t agree that any one instrumentation gets you closer to 
experimentation or avant garde.  I think that is a cop out.  Any instrument, 
especially one as open ended as the modular, does not put you in a genre or 
make you better at one thing than another. That is just humans being lazy and 
categorizing shit. Frankly, I'm better at that avant garde stuff when I use guitar 
gear.

how do you approach limitations in hardware design? 

Ugh, good question. Usually I ask for EVERYTHING. Then I see how many 
“Nos” and “Maybes” I get back from our engineers (myself included) and usually I
try to figure out at least one of the “Nos” and often times we can cram in a couple 
“maybes” too. Better to go for it all and whittle it down than start small and realize 
what you could have done far better after the fucking thing is in stores. The same 
with recording, actually. “Can I put another drum track on there?” “Hell yeah, I 
already did 12 guitars, we'll whittle it down later.”

have proprietary tools / designs and planned obsolescence influenced
your work? 

Well, now that we can see some of our through hole components going out 
of production I'm pushing SMD builds from here on out, for the most part. We 
also have to figure out replacement parts for a few things coming up and redoing 
the boards for them. Not fun! Other than that, we just have to keep replacing our 
phones and computers, I guess.  

We really aren’t into the old, vintage parts mojo BS, so most of that misses 
us completely.

Conversely, have you shared any of your designs / tricks / magic? 

Not really, for the most part I just apply fairly common knowledge in
unconventional ways, so I wouldn't really be blowing anyone's hair back. There 
isn’t any magic here, just hard work and some weird brains working together.  
Some of our schematics are online, but to my knowledge, none of them 
are correct. 

how important have other people been in the technical aspect of your
work? 

Very important now. Early on, I did all of the designs myself and just paid 
someone else to lay out the PCB for us. The things we're working on at the 
moment are collaborative, but mostly they stem from my ideas, and I guide the 
design process. It's far better to hire someone who has the skills I don't, rather 
than try to master EVERYTHING, and end up losing my shit in the process. We 
have one full time “techinical” Henchman in the workshop, and Bob Lowe works 
in his own shop, on his own time. 

do you have any opinions or thoughts on your professional community? 

Pretty much everybody I've met in the industry has been top shelf. Once 
you're in it, I feel like if you haven't got too bad of a mental disorder you can 
identify with everybody else in the trenches. 

have you helped beginners with design question like Devi did for you? 

I don't recall going to Devi for design questions until I was trying to build 
her stuff, which can be really confusing the first couple times. We asked her a 
LOT of business questions, though.  She helped with finding dealers, pricing, 
finding the best prices on LEDs, boring stuff like that. 

I don't recall really giving out any information on that subject either, other 
than “Oh yeah, that Pitchgrinder runs on a big PIC chip. You should try that!” Just 
some general shop talk here and there. We also recommend websites and a few 
books for people to research on their own. We don’t really have time to train 
people in how to build and design. I also kind of hate when people ask questions 
that are easily searchable on the great wide internet.

are there any philosophies or beliefs that seem important to you as
you do this work?  

“Pedal and Steer” and “Don't Scream, Do Something.”
“Pedal and Steer” comes from when we were teaching our kids to ride a
bike. But it really applies to any task. (Pedal) put in the work. The
raw energy. (Steer) Keep your eyes open and direct that energy.
“Don't scream, Do something” also comes from raising the kids. When
there's a problem, kids can seem to shut down and cry about it, and in
their own way, plenty of adults do that, too. But usually, you need to
shut the fuck up, pedal and steer. Of course those are the easiest and
hardest things in the world, depending on the day.
