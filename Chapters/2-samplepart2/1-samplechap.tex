% ----------------------------------------------------------------------------------------
% CHAPTER TITLE
% ----------------------------------------------------------------------------------------
\chapter{Interviews: highlights and analyses}\label{interviews}
\lhead{\chaptertitlename\ \thechapter. \emph{Interviews}}
% ----------------------------------------------------------------------------------------

These interviews wished to give previous sections additional context. Through their work, these artists and engineers have offered their vision of this fragmented and open field. They do not necessarily represent the space of possbilities for electronic music hardware practices: rather, they illustrate that this fragmentation and openness ultimately creates a robust, self-sustaining and challenging space for creation. 

The process undertaken here wished to address common questions while allowing each practitioner to develop the specificities of their work. After describing these overall commonalities and methods, the specifics for each artist and some highlights from their response are presented. 

\section{Interviews: methodology and highlights}

In the process of preparing these questions, five loose themes around which to organize questions were chosen. Those were: 

\begin{itemize}
	\item the current place of hardware in their work 
	\item their approaches in dealing with technical limitations
	\item their perception of the professional community they might feel part of
	\item the importance of an ethos in their design work
	\item the extent to which they consciously engaged with the questions ofexperimental or avant-garde music 
	\end{itemize}

These points were then adapted to fit the preliminary research undertaken for each interviewee. They were complemented when necessary by questions regarding each person's background or specific experience. Four exchanges took place over email, which was less flexible but offered more time for the responders. The other four took place in person (Nicolas Collins, Sunny Nam, Dan Snazelle) or over the phone (Tristan Shone). All of those interactions happened between november 2014 and march 2015. 

The goal was to get an understanding of their current relations to electronic music hardware, how they developed that approach, and where they see it going next. This section details highlights and analyses derived from this body of statements. 

\emph{Please refer to appendix A for full transcripts. Page numbers correspond to the location of a specific statement within that appendix. }

\subsection{Tristan Shone: Author and Punisher}

Shone's background as a metal musician, mechanical engineer and sculptor offered a chance to discuss the possibilities of interdisciplinary work within a compositional context with which I was familiar. Questions focused on his arduino-based hardware, the influence of his profession on his art, and his practical approach to design and open source. 

\paragraph{highlights}

When considering Shone's professional history, he truly appears to be in the middle of multiple disciplines. He has degrees in mechanical engineering and sculpture, and makes a living from his music and various design and manufacture employments. He is close to academia: his current employment is in a medical research facility at UCSD, while one of his previous positions involved prototyping for MIT multimedia artist Chris Csikszentmihályi. 

He openly acknowledges the practicality and aspirations behind this 

\paragraph{analysis}



\subsection{Louise \& Ben Hinz, Devi Ever FX and Dwarfcraft Devices}
\paragraph{highlights}
\paragraph{analysis}

\subsection{Nic Collins}
\paragraph{highlights}
\paragraph{analysis}

\subsection{Bonnie Jones, Techne}
\paragraph{highlights}
\paragraph{analysis}

\subsection{Jessica Rylan, Flower Electronics}
\paragraph{highlights}
\paragraph{analysis}

\subsection{Martin Howse}
\paragraph{highlights}
\paragraph{analysis}

\subsection{Dan Snazelle, Sanzzy FX}
\paragraph{highlights}
\paragraph{analysis}

\subsection{Sangwook Sunny Nam}

\paragraph{highlights}
\paragraph{analysis}

\section{Greater context}

\section{trends?}

% ---------------
% To incorporate in this chapter
\begin{unsortedStuff}	
\section*{(TO INCORPORATE)}
	\begin{itemize}
		\item 
	\end{itemize}
\end{unsortedStuff}
		
%Blank page to add written thoughts
\begin{optBlankSpace}
	\newpage
	\mbox{}
\end{optBlankSpace}

