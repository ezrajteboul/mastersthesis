% ----------------------------------------------------------------------------------------
% CHAPTER TITLE
% ----------------------------------------------------------------------------------------
\chapter{Interviews: highlights and analyses}\label{interviews}
\lhead{\chaptertitlename\ \thechapter. \emph{Interviews}}
% ----------------------------------------------------------------------------------------
This section will further discuss design concepts with some practitioners. By presenting an equally disparate collection of opinions and philosophies, we can better understand the varied motivations, backgrounds and interests that maintain the overwhelmingly interdisciplinary and scattered nature of electric instrument design. 

Through their work, these artists / engineers have offered their vision of this fragmented and open field. They do not necessarily represent the entire space of possbilities for electronic music hardware. Rather, they illustrate that this fragmentation and openness ultimately creates a robust, self-sustaining and multi-faceted space for creation. What that implies will be discussed in detail below.

The process undertaken here addresses common questions while allowing each practitioner to develop points specific to their work. After describing these overall commonalities and methods, the specifics for each artist and some highlights from their response are presented. 

\section{Interviews: methodology and highlights}

In the process of preparing these questions, five themes around which to organize questions were chosen. Those were: 

\begin{itemize}
	\item current place of hardware in their work 
	\item dealing with technical limitations
	\item their perception of the professional community they might feel part of
	\item the importance of an ethos in their design work
	\item engagement with the questions of experimental or avant-garde music 
	\end{itemize}

These points were then adapted to fit the preliminary research undertaken for each interviewee. They were complemented when necessary by questions regarding each person's background or specific experience. Four exchanges took place over email, which was less flexible but offered more time for the responders. The other four took place in person (Nicolas Collins, Sunny Nam, Dan Snazelle) or over the phone (Tristan Shone). All of those interactions happened between november 2014 and march 2015. 

The goal was to get an understanding of their current relations to electronic music hardware, how they developed that approach, and where they see it going next. This section details highlights and analyses derived from this body of statements. 

\emph{Please refer to appendix A for full transcripts.}

\subsection{Tristan Shone: Author and Punisher}

Shone's background as a self-taught musician, mechanical engineer and sculptor offered a chance to discuss the possibilities of interdisciplinary work in a compositional context with which I was familiar. Questions focused on his arduino-based hardware, the influence of his profession on his art, and his  approach to design and open source. 

\begin{quote}
	
	``I'm just trying to make music. I have no purist aspect, other than playing everything live. Anything I have to do for that is ok. If I have to get funding from the North Korean dictatorship, I'd do it.''
	
	\end{quote}
	
Shone's pragmatic approach to hardware design results in some of the more over-the-top designs recently developed. Given his unique position as a professional CAD-CAM specialist and mechanical engineer, he appears as particularly qualified to be making that type of aesthetic statement, but he is also increasingly working to make sure this visually striking work gets in the way of the result, which is the music and which he hopes to bring attentio to. 

This explains two recent trends that discussing these topics have elucidated. First, that he is building smaller, lighter controllers for performances. Admittedly, this is to match airline luggage requirements - but it is in line with his push for less visually overwhelming devices. Second, it has motivated collaborations with visual artist Will Michaelson. This appears as reassurance that he people in the diy community are still concerned with the final sonic product of their processes. 

Shone is an example of someone who has extensively benefited from both open and commercial hardware or software efforts. His devices rely heavily on the HIDuino firmware, developed by Dimitri Diakopoulos and released with an MIT license, but his sampling and synthesis engines are mostly contained within the Ableton Live software. 

He is also aware and respectful of other practitioners in the field that constitute what Nic Collins describes as hackers, but does not identify with them (or any other field of electronic music hardware, for that matter). In many ways, he considers his electronic work to be fairly simplistic and sraightforward, adaptations of other previous work to his particular objectives. He is open to the idea of sharing concepts and ideas (as demonstrated by his Make magazine contribution), but just like with his performance, the music is the bottom line of all these processes. 

Technical pragmatism does not however mean lack of values. Shone is still concerned with authenticity - here, personalization isn't solely motivated by increased performance, reliability and efficiency in compositional work, but also by a desire for instruments that truly reflect the nature of the materials that they are built with, where each material decision is justified by a performative advantage or compromise. 

To conclude, he seems fairly uncommited to a particular community, trend, or practice, whether in music or design. Through this relative independence and proficiency in fabrication and composition, Shone is an answer to Tudor and Collins' concern with being both a musician and an engineer. Few of the other interviewees seem to blend both activities as closely and naturally. This interdisciplinary ease is presented as necessary, natural, and ultimately only in service of the art:

\begin{quote}
	
	``I definitely think I'm more a musician and an artist. There's something exact about what I do, but what I like the most is the accidents and the free form nature of what I'm doing. I use the engineering to achieve the goals I want for my music.''
	
	\end{quote}

%\paragraph{highlights}
%\paragraph{analysis}

%\subsection{Louise \& Ben Hinz, Devi Ever FX and Dwarfcraft Devices}

%If Shone comes from an academic and professional manufacturing perspective, the Hinz are mostly self taught. 

%\paragraph{highlights}
%\paragraph{analysis}

\subsection{Nic Collins}

``There’s so much interest in what’s called ``silence studies'' in the last ten years. This is cyclical. There’s a Cage / Rauschenberg moment in the 60’s, then it came back with a vengeance. It took so long for Cage’s ideas... not to be accepted, but rather internalized. For example, people from many aesthetics could view silence as a positive element, rather than as the absence of something. It hit something, at the turn of the millennium, when all these people realized you could carve things out of all those negative spaces.''

Collins appreciation of cycles of development and rediscovery in experimental music give this discussion a unique perspective. Just like Cage's compositional ideals took multiple iterations of recylcling before achieving their status, Collins presents Tudor and microcontrollers as interesting parallels, symptom of a particular ``zeitgeist''. Tudor wasn't academically discussed as a composer until after his death, while microcontrollers made an apparition in the 80's, then a covert rise in the hacking world as what powered the speak and spells or Casios of the 90's, before finally powering the arduino revolution of the late 2000's.

In Collins' mind, these are expressions of the public's interest. If hardware hacking was a response to a certain ``digital hangover'', the interest in arduinos was a response to the hard work and cheap accessibility the system encapsulated. 

Collins' recent work appears as a collection of parallel tracks, aware of all these cycles and their current popularities, yet still focused on the sound professor's roots and concerns. These could be resumed succintly and incompletely to teaching and modern minimalism - indeed, the simplification allows us to address something that seems paramount to Collins in discussion: that in the long term, his teaching has inspired more projects and ideas than most other sources. 

%\begin{quote}
	
	%``There’s something pretty dreary about concerts at Circuit Bending festivals. It often seems like the music might be an afterthought. There’s nothing wrong with being a luthier. There are people whose tradition is building great instruments. It can be Stradivarius, it can be Trimpin, it can be the engineers that are behind the cracklebox, it can be Bob Moog, but they’re not necessarily the people to whom you want to listen to records by.''
	
	%\end{quote}
	
%The divide between expert musician and fabricator seems to be a rule that few practitioners (like Tristan Shone, Bonnie Jones or Tristan Perich) seem to be able to overcome. 

%\paragraph{highlights}
%\paragraph{analysis}

%\subsection{Bonnie Jones, Techne}
%\paragraph{highlights}
%\paragraph{analysis}

%\subsection{Jessica Rylan, Flower Electronics}
%\paragraph{highlights}
%\paragraph{analysis}

%\subsection{Martin Howse}

% \paragraph{highlights}
% \paragraph{analysis}
%
% \subsection{Dan Snazelle, Snazzy FX}
% \paragraph{highlights}
% \paragraph{analysis}
%
% \subsection{Sangwook Sunny Nam}
%
% \paragraph{highlights}
% \paragraph{analysis}
%
% \section{Greater context}
%
% \section{trends?}

% ---------------
% To incorporate in this chapter
\begin{unsortedStuff}	
\section*{(TO INCORPORATE)}
	\begin{itemize}
		\item 
	\end{itemize}
\end{unsortedStuff}
		
%Blank page to add written thoughts
\begin{optBlankSpace}
	\newpage
	\mbox{}
\end{optBlankSpace}

