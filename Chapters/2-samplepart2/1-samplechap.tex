% ----------------------------------------------------------------------------------------
% CHAPTER TITLE
% ----------------------------------------------------------------------------------------
\chapter{Interviews: highlights and analyses}\label{interviews}
\lhead{\chaptertitlename\ \thechapter. \emph{Interviews}}
% ----------------------------------------------------------------------------------------
This section will further discuss design concepts with related practitioners. By presenting an equally disparate collection of opinions and philosophies, we can better understand the varied motivations, backgrounds and interests that maintain the overwhelmingly interdisciplinary and scattered nature of electric instrument design. 

Through their work, these artists / engineers have offered their vision of this fragmented and open field. They do not necessarily represent the entire space of possbilities for electronic music hardware. Rather, they illustrate that this fragmentation and openness ultimately creates a robust, self-sustaining and multi-faceted space for creation. What that implies will be discussed in detail below.

The process undertaken here addresses common questions while allowing each practitioner to develop points specific to their work. After describing these overall commonalities and methods, the specifics for each artist and some highlights from their response are presented. 

\section{Interviews: methodology and highlights}

In the process of preparing these questions, five themes around which to organize questions were chosen. Those were: 

\begin{itemize}
	\item current place of hardware in their work 
	\item dealing with technical limitations
	\item their perception of the professional community they might feel part of
	\item the importance of an ethos in their design work
	\item engagement with the questions of experimental or avant-garde music 
	\end{itemize}

These points were then adapted to fit the preliminary research undertaken for each interviewee. They were complemented when necessary by questions regarding each person's background or specific experience. Four exchanges took place over email, which was less flexible but offered more time for the responders. The other four took place in person (Nicolas Collins, Sunny Nam, Dan Snazelle) or over the phone (Tristan Shone). All of those interactions happened between november 2014 and march 2015. 

The goal was to get an understanding of their current relations to electronic music hardware, how they developed that approach, and where they see it going next. This section details highlights and analyses derived from this body of statements. 

\emph{Please refer to appendix A for full transcripts.}

\subsection{Tristan Shone: Author and Punisher}

Shone's background as a self-taught musician, mechanical engineer and sculptor offered a chance to discuss the possibilities of interdisciplinary work in a compositional context with which I was familiar. Questions focused on his arduino-based hardware, the influence of his profession on his art, and his  approach to design and open source. 

\begin{quote}
	
	``I'm just trying to make music. I have no purist aspect, other than playing everything live. Anything I have to do for that is ok. If I have to get funding from the North Korean dictatorship, I'd do it.''
	
	\end{quote}
	
Shone's pragmatic approach to hardware design results in some of the more over-the-top designs recently developed. Given his unique position as a professional CAD-CAM specialist and mechanical engineer, he appears as particularly qualified to be making that type of aesthetic statement, but he is also increasingly working to make sure this visually striking work gets in the way of the result, which is the music and which he hopes to bring attentio to. 

This explains two recent trends that discussing these topics have elucidated. First, that he is building smaller, lighter controllers for performances. Admittedly, this is to match airline luggage requirements - but it is in line with his push for less visually overwhelming devices. Second, it has motivated collaborations with visual artist Will Michaelson. This appears as reassurance that he people in the diy community are still concerned with the final sonic product of their processes. 

Shone is an example of someone who has extensively benefited from both open and commercial hardware or software efforts. His devices rely heavily on the HIDuino firmware, developed by Dimitri Diakopoulos and released with an MIT license, but his sampling and synthesis engines are mostly contained within the Ableton Live software. 

He is also aware and respectful of other practitioners in the field that constitute what Nic Collins describes as hackers, but does not identify with them (or any other field of electronic music hardware, for that matter). In many ways, he considers his electronic work to be fairly simplistic and sraightforward, adaptations of other previous work to his particular objectives. He is open to the idea of sharing concepts and ideas (as demonstrated by his Make magazine contribution), but just like with his performance, the music is the bottom line of all these processes. 

Technical pragmatism does not however mean lack of values. Shone is still concerned with authenticity - here, personalization isn't solely motivated by increased performance, reliability and efficiency in compositional work, but also by a desire for instruments that truly reflect the nature of the materials that they are built with, where each material decision is justified by a performative advantage or compromise. 

To conclude, he seems fairly uncommited to a particular community, trend, or practice, whether in music or design. Through this relative independence and proficiency in fabrication and composition, Shone is an answer to Tudor and Collins' concern with being both a musician and an engineer. Few of the other interviewees seem to blend both activities as closely and naturally. This interdisciplinary ease is presented as necessary, natural, and ultimately only in service of the art:

\begin{quote}
	
	``I definitely think I'm more a musician and an artist. There's something exact about what I do, but what I like the most is the accidents and the free form nature of what I'm doing. I use the engineering to achieve the goals I want for my music.''
	
	\end{quote}

%\paragraph{highlights}
%\paragraph{analysis}

%\subsection{Louise \& Ben Hinz, Devi Ever FX and Dwarfcraft Devices}

%If Shone comes from an academic and professional manufacturing perspective, the Hinz are mostly self taught. 

%\paragraph{highlights}
%\paragraph{analysis}

\subsection{Nic Collins}

``There’s so much interest in what’s called ``silence studies'' in the last ten years. This is cyclical. There’s a Cage / Rauschenberg moment in the 60’s, then it came back with a vengeance. It took so long for Cage’s ideas... not to be accepted, but rather internalized. For example, people from many aesthetics could view silence as a positive element, rather than as the absence of something. It hit something, at the turn of the millennium, when all these people realized you could carve things out of all those negative spaces.''

Collins appreciation of cycles of development and rediscovery in experimental music give this discussion a unique perspective. Just like Cage's compositional ideals took multiple iterations of recylcling before achieving their status, Collins presents Tudor and microcontrollers as interesting parallels, symptom of a particular ``zeitgeist''. Tudor wasn't academically discussed as a composer until after his death, while microcontrollers made an apparition in the 80's, then a covert rise in the hacking world as what powered the speak and spells or Casios of the 90's, before finally powering the arduino revolution of the late 2000's.

In Collins' mind, these are expressions of the public's interest. If hardware hacking was a response to a certain ``digital hangover'', the interest in arduinos was a response to the hard work and cheap accessibility the system encapsulated. 

Collins' recent work appears as a collection of parallel tracks, aware of all these cycles and their current popularities, yet still focused on the sound professor's roots and concerns. 

\begin{quote}
	``I’ve carried parallel practices in hardware and software for years and years and years. They’ve always worked together but what I would say is at the moment, it’s the chaotic aspects, the instability of circuits that are coming to a full forward in the stuff I’m doing.''
	
	\end{quote}
	
These chaotic aspects, if not a trend, seem to be widespread amongst artists working between circuit designe and engineering - beyond Jessica Rylan, Dan Snazelle and Collins quoted in this thesis, one simply needs to refer to scholarship on japanese noise music or circuit bending to appreciate the far reaches of intuitive approaches in music and technology. 

And Collins, as both someone who ``wrote the book'' and has been teaching all over the world for the past three decades appears as both aware of his role in the development of these practices and their continued developments. Citing Korean movements of audio circuit minimalisms or George Lewis depending on the context, the breadth of \emph{Handmade Electronic Music} seems natural after talking to its author. This is not only in the sense that the book seems to be a logical output of its source, but also that it filled a certain need and interest within artistic and academic communities. 

In no way is this self-congratulatory - Collins still is aware of some shortcomings and limitations in the methodologies he's documented. 

\begin{quote}
	
	``There’s something pretty dreary about concerts at Circuit Bending festivals. It often seems like the music might be an afterthought. There’s nothing wrong with being a luthier. There are people whose tradition is building great instruments. It can be Stradivarius, it can be Trimpin, it can be the engineers that are behind the cracklebox, it can be Bob Moog, but they’re not necessarily the people to whom you want to listen to records by.''
	
	\end{quote}
	
The divide between expert musician and fabricator seems to be a rule that few practitioners (like Tristan Shone, Bonnie Jones or Tristan Perich) seem to be able to overcome. Collins was one of the first interviewees in this process - these opening remarks set the tone for the other conversations. 

%\paragraph{highlights}
%\paragraph{analysis}

\subsection{Bonnie Jones, Techne}

Bonnie Jones was of interest to this project because she has performed extensively using a set of live-bended digital delays and an assorted set of complementary systems (some of which quite Tudor-like in their indeterminacy the respectful use of the sonic qualities). Furthermore, her experience founding, directing and teaching for Techne (an electronic music hardware summer program for girls) placed her in a privileged position to discuss making and composing as it relates to gender and education. 

Let us focus first on Jones' primary instrument, the delay pedal. As she details, an element that is often used to make up a shortcoming of electric instruments - their temporal flatness, the lack of space-related delays - is turned into an instrument by the charged process of turning it around and opening it, exposing the circuit board and making it as much of tactile instrument as a ``sax, or a violin''. This appears as important - live circuit bending can be a valid and respected instance of instrumentation. 

Relating back to Tudor's vision of inderterminacy, Jones' approach to unpredictable behaviors is one that is explicitly rooted in practice and expertise. ``I appreciate when that instrument has surprises or enables me to create sounds that I wouldn't expect'', she details, but all efforts ultimately need to be effective: ``I wouldn't care if something was complex if I didn't like the way it sounded.'' This practical approach reflects that of Shone's, or Snazelle. Introducing context-specific elements or free software tools in her performance allows her to respond to a particular prompt, and possibly to engage more closely with the audience. This emphasis on the responsibility of the author to cater to a public seems specific to a specific set of performative traditions, and yet Jones concludes her interview with the following line: ``I am deeply skeptical and suspicious of what is visible.'' 

Considering the extent to which electronic music is based on nested black boxes and obscured processes, this distrust in what is visible seems to not leave much for the artist to work with. Jones discusses the contradictory nature of making and improvisation, which attempt to both reveal ideas and obscure the self. Open hardware, in this context, seems almost futile. If everyone needs to doubt what is presented before them, individuals might develop a more critical and accurate view of their tools, but at the price of time-taking distrust. 

%\paragraph{highlights}
%\paragraph{analysis}

\subsection{Jessica Rylan, Flower Electronics}

Jessica Rylan, who's published chaotic electronics papers under the name Jessica Piper, was included here for her apparent relation to chaotic / semi autonomous systems for composition. In her discussion with Tara Rodgers, she also expressed a disappointment in standard paradigms for synthesis, and in the biased communities present in her field. She's currently pursuing a nano-optics PhD at Stanford, meaning that her audio hardware design and performances have slowed to a standstill - however, getting a chance to have her elaborate on some of those topics was a good opportunity for clarification. 

In her replies, self-limitations such as not using an oscilloscope / multimeter for a year, or picking the circuit based on how its schematic looks rather than how it sounds seems to border on performance art rather than engineering. In that sense, her scientific development seems particularly important: periods of self-teaching and experimentation followed by intense involvement in academic environments and a period as a employee of Don Buchla's business. 

She addresses the issue developed by Collins of the engineer which is not the best player of its own designs, and once again, the bottom line appears to be pragmatism: ``I found design to be a really exciting world for discovering sounds I hadn't heard before and didn't know existed, as well as a way to realize sounds I wanted to hear but couldn't find.''

This, once more, brings us back to the concepts of musical chaos, which Rylan presents as inherently easier to implement in the analog domain. 

Another point that resonates strongly is the relationship between her academic experience and her musical experiments: ``the approach to circuit design taught in engineering programs is strictly at odds with the kind of music that the personal synth allowed for.'' As Collins described in our discussion, Tudor was uncomfortable with electronics until Mumma gave up on teaching him vacuum tube circuit design and moved to solid state. The lesson here seems to be that succesfully combining both practices is based mostly on a fragile balance between intuition and curiosity. The type of dreaming that Rylan expresses a few lines later - daydreaming of ``having our own fab and making our own transistors'', appears as typical of the grandiose ideas that often fuel simpler projects, as well as what we can ultimately hope for from developments in open design and fabrication. 

%\paragraph{highlights}
%\paragraph{analysis}

\subsection{Martin Howse}

Getting a chance to hear back from Martin Howse was particularly instructive because he is one of few artists who seem to strike that elusive balance between system design and compelling uses of technology in a musical context. 

Howse's designs seem to utilize a combination of digital systems and organic materials, rooted in ``an obsession with the material at the basis of digital and communications technology.'' For context, it'll be interesting to remember that what drove the Barrons in their self-decaying circuit work was a fascination for cybernetics. 

``the approach is very much a revealing, either through re-working materials towards technology (for example, performances using earth as an active, electrochem- ical, biologic material), or dissecting and almost dissolving (in chemical sense) digital technology (in workshops), or devising software which examines its own material con- ditions (for example, Island2 installation).''

The unprompted parallel between teaching, making and installation art including music is here very clear, confirming that the work displayed in his \emph{dark interpreters} designs are the result of an intentional, conscious process. The goal is to present a tangible surface for machine languages: ``I wanted the user to literally put their fingers into the code, to run the code over their skin'' In this process, the importance of teaching as a method for inspiration and self-learning parallels the ideas discussed with Collins, emphasizing the unconventional sense of community one can derive from those ideas if they wish to do so. 

A number of Howse's designs are available online, and were developed using open source software. Interestingly enough, circuit design is a domain in which open source circuit design software is harder to learn than the closed source but free standard, EagleCAD. A committed curiosity for Howse's work therefore involves some amount of inherent learning, indirectly sustainin FOSS alongside open electronic music hardware. 

He shares an interesting characteristic with Bonnie Jones, as both have experience in the field of literature. For Howse, this reflects in the code he writes through comments and name references - another example of the myriad of interdisciplinary opportunities within open electronic music instrument design. This connection to peers and audiences seems to be mediated mostly through his work, as the community aspects of the technology or the performances don't seem to project further for Howse than his workshops and some specific collaborators. 

% \paragraph{highlights}
% \paragraph{analysis}
%
\subsection{Dan Snazelle, Snazzy FX}

Dan Snazelle was interviewed because of his work with the Ardcore synthesizer module, which effectively brings a tradition of additive / substractive electronics with the multipurpose paradigms of more recent microprocessor-based synthesis. 

As research revealed that this project was a collaboration between Snazelle and Darwin Grosse, the set of questions I prepared covered the common core described at the beginning of this chapter, but also attempted to get more information concerning this collaboration, hybrid systems in electronic music. It also inquired about the development of a small community around the Ardcore's collaborative and open code base. 

\paragraph{the guitar pedal as gateway technology}

The first element worth noting was that Snazelle's initial commercial line marketed through Snazzy FX consisted of three guitar pedals. This isn't due to a particular interest in the format. Rather, at that time (mid 2000's), the boutique synthesizer module market was a small one, and guitar pedals would be easier to sell without needing significant design changes. 

The connection between diy, guitar electronics, and more standalong synthesis hardware is important. Pedals offer a first, simple practical experience for many musicians, especially those coming from a popular music context rather than an experimental, avant-garde or contemporary classical one. The relatively low complexity means that the high barrier set by engineering described by Jessica Rylan can appear less insurmountable, and succesful professional carreers can be built around those efforts. 

Furthermore, pedals' simplicity often beat that of even the simplest digital circuits if the maker in question has any issues with programming. Snazelle is only one of many practitioners who feel like analog operates on a more human and approachable scale, which often results in an intuitive connection to analog-centric systems (although eurorack has become increasingly digital). 

\paragraph{making things to make music to make things to}

Although Dan's musical output seems to be less well advertized than his products, he presents his purely creative output as constant throughout his various carreers, important to his well-being, varied in its inspirations and aspirations, and most importantly complementary to his hardware practice. 

The circumstances led us to describe his compositional work less than his physical products, but simply being in his office and seeing his collection of instruments is proof of his care for sonic results. The parallels between both processes are obvious to him: 

(on making modules, pedals, and audio hardware) ``It's a lot like writing an album - as an album (...). I think of Snazzy FX as an art project. It's my art. These (pointing to his products, prototypes, setup) were a statement. (...) the design aspect is really intuitive.''

This approach to the materiality of sound making appears as a bridge between commercial but small scale businesses and installation work. From the circuits to the enclosure design, Snazelle is mostly concerned with the interaction between his products and their users, hoping to make his clients ``think about making music differently''. 

Snazelle adds on his designs: ``I wanted to go as far as I could. In ten years, I hope the stuff I'm doin will be even more toward that weird goal of making these systems that are designed to do that weird stuff.''

The vagueness of \emph{weird stuff} was both confusing and fitting - Snazelle, with his chaotic modules and the ardcore, is not the most prolific eurorack manufacturer but one of the most adventurous ones. He isn't selling his version of classic modules, but trying to develop the items he wishes were available. Stepping back, he resumes an overall vision as such: 

``I'm not determining every event, point A to point B... I'm going to turn it on, and set something up, and something's going to come out. If you do it right, or if you do it wrong, it can keep going and going. Those unexpected possibilities, for a musician, that's gold.''

Agency and control are once more essential to this discussion of custom and original musical hardware. As these case studies and interviews contextualize each other, we can see that the spectrum of practices they represent offer a nebulous array of answers, which is 

\paragraph{on community, business and peers}

When discussing the importance of his predecessors in the field, Dan speaks of everyone with respect and appreciation. It is clear that his particular business is built on a tradition of sharing information through privileged channels, which usually start online and develop at conferences or meets as each actor becomes more dedicated to this scene. 

As one of the slightly more senior interviewees in this process, Snazelle stated that the last few years in his business had seen an explosion of manufacturers, costumers and technological developments, which menace to change the overall friendliness of the participants. As larger companies come in on the eurorack market, he describes the competition as being based on popularity rather than originality, and expresses concern as to the overall quality of products marketed for large sales potential. 

He also acknowledges that this increased user-base and the monetary weight that comes with it has allowed the entrepreneur and self-taught circuit designer to access many more resources than were available to him in his beginnings. 

% \paragraph{highlights}
% \paragraph{analysis}
%
\subsection{Sangwook Sunny Nam & Joshua Florian}

Sunny Nam, as a mastering engineer, 

%
% \paragraph{highlights}
% \paragraph{analysis}
%
% \section{Greater context}
%
% \section{trends?}

% ---------------
% To incorporate in this chapter
\begin{unsortedStuff}	
\section*{(TO INCORPORATE)}
	\begin{itemize}
		\item 
	\end{itemize}
\end{unsortedStuff}
		
%Blank page to add written thoughts
\begin{optBlankSpace}
	\newpage
	\mbox{}
\end{optBlankSpace}

