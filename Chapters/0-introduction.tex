% ----------------------------------------------------------------------------------------
% INTRODUCTION
% ----------------------------------------------------------------------------------------
\chapter{Introduction}\label{introduction}
\lhead{\chaptertitlename\ \thechapter. \emph{Introduction}}
% ----------------------------------------------------------------------------------------



\section{a custom aesthetic}

Music and electronics are \"comfortably interrelated" \cite{kettelkamp1984}. The space within which electronics are conceptualized and developed is defined by the technical means available to the designer and manufacturer. Advances in hardware specifications allow for increasingly sophisticated implementation of analog and digital signal processing techniques, as well as more robust, elegant or inspiring human/machine interfaces.  
	
Audio devices, as a subset of consumer electronics, are bound by those technical limitations. Both corresponding markets see product updates every time a significant subset of those limitations evolve. Beyond those common boundaries, however, audio electronics design and a generalist engineering practice exhibit different approaches to innovation and renewal. 

In the 1960's, transistors rendered the dangerous and inefficient vacuum tube obsolete in most devices except for very high frequency circuits and audio amplifiers \cite{barbour1998}. The latter case is an early example of inefficient systems being maintained for aesthetic considerations: vacuum tubes produced specific nonlinearities, internalized as a cultural bias that has perpetuated the relevance of an otherwise antiquated technology \cite{barbour1998}. That favoritism is based in subjectivity - some music practitioners prefer solid state amplifiers for an arbitrary subset of qualities the technology offers \cite{hamm1973}. Different musical backgrounds and experiences inform the technical choices made in the design of each musician's specific setup. Specifically, they influence which limitations, materialized as imperfections, are acceptable in a personal musical system. Vacuum tubes are only one instance of this phenomena amongst many: discrete transistors, analog synthesizers, low-resolution digital audio, etc.  

The task of designing electronic music instruments can not be reduced to a set of performance requirements for a circuit, algorithms, or both. Viewing instruments as optimal machines does not do justice to their potential as catalysts of poetic experiences - interesting results arise when designers explicitly address them as post-optimal objects \cite{dunne2005}. Successful instruments must inspire their users, helping them to transform abstractions and intuitions into a concrete sonic output, with limitations often proving inspirational \cite{evens2005,rovan2009}. Don Buchla, in designing his \"music boxes\", Don Buchla describes himself as a builder of instruments rather than of machines - he is consciously addressing a niche market \cite{pinch2001}. Nic Collins brings this one step further by coining the term silicon luthier, the craftsperson of electronic music \cite{collins2008}. By intentionally limiting a user base, Buchla and Collins both suggest that electronic music instruments can benefit from a small-scale, custom approach in both the design and manufacturing processes. 

This benefit is compounded by the constantly evolving nature of electronic hardware and the digital software it often hosts. Technological advances are not correlated to changes in aesthetics of music: compelling uses of a new instrument are as important as the design of the instrument itself \cite{braun2000}. Designing an instrument for a specific person or use increases its chances of successfully empowering a creative process by addressing the issues inherent to ubiquitous performance and composition systems today \cite{armstrong2006,haslett2005}. 

That particular, personal, small scale, inspiringly imperfect vision of electronic music instrument design materialized in various forms through the 20th century: inventors, do-it-yourself (diy), hacking, circuit bending. Today, this set of historical practices joins back with trends in general hardware design, with movements such as the open source and maker movements ~\cite{mellis2014,perner2011}. Practitioners meet in FabLabs as much as in their basement shops ~\cite{mellis2011}. Both appear as symptomatic of a similar interest across user-bases: a desire take advantage of emerging technologies to create personal, more inspiring and efficient experiences for themselves and their audiences ~\cite{hermans2014}. The primary goal of this thesis is to contextualize and analyze the work of designers who fit within this vision of audio hardware, aware of precedents but challenging tradition and expectations through inspiring if imperfect uses of electronics.

\section{Motivation}

This work exists because of a deep curiosity for the instruments that makes a music electronic, their creators, and their place in today's art world. It hopes to foster future interdisciplinary research in the field of electronic music hardware and encourage open, experimental devices that blur the line between composition and design. 

Diy music instrument communities are based on sharing designs, advice and results. Successful resources document projects from start to end, with more than enough information to tackle any eventual mistake. These attributes find clear parallels in open source hardware design practices, as defined by the open source hardware association (OSHWA, w2015). By documenting and contextualizing relevant projects, one can contribute back to both communities. Furthermore, presenting this material in an academic context helps link these practitioners to some of the driving forces behind those emerging technologies, as those still require strong corporate or academic backing. 

This ethnography of design practices in musical technologies should also illustrate an aspect of the interdisciplinary nature of music technology, allowing further comparisons and connections with other designers, trends and fields. By rooting contemporary electronic music practice to the hardware that enables it, a stronger bond to the developing field of sound art is favored ~\cite{cluett2013}. A more explicit connection is drawn to the people who make audio electronics, whether they are boutique designers or factory workers ~\cite{rylan2015}.  Issues arising from a direct transferal of design practices from an engineering based in market economies and cult of performance can be better identified, addressed, and solved for the further development of inspiring interfaces ~\cite{ghazala2004,christensen2005,Feldman2007,silver2009,perner2011,hertz2012,riis2013,jackson2014}.

\section{Scope and structure}

This work wishes to contextualize and analyze a selection of audio electronic systems as products of a technological, cultural and social environment. By acknowledging the products, prototypes, tools, and people involved in this process, a more accurate description of these ecosystems of invention can be achieved \cite{vinck2003}. This, in turn, allows for a clearer view of the possibilities and futures for the field of devices for musical expression. 
	
	For this clear description to emerge, this document will introduce a selective history of formative practices paralleling the history of modern electronics \cite{holmes2002}. Starting with the isolated forefathers of the field and their \"spirit of invention" \cite{dunn2001}, useful references through to Nic Collins' Handmade Electronic Music \cite{collins2006} will allow us to discuss the concepts of component design, system design and interaction design in a musical context. In this discussion, do-it-yourself, hacking and circuit bending are important, both in terms of general design and in how they apply to musical practices. We will then discuss the current incarnation of this spirit of invention and experimentation, in the greater context of making and sharing cultures. This world of open innovation \cite{christensen2005}, and, more specifically, open hardware, will allow us to tackle the necessity and opportunity to present one facet of electronic instrument design as that of an ecosystem for invention. In this context, the growth and variety of the ecosystem will appear as almost as burgeoning as its real world analog. 

	Linking this back to the field of experimental music and physicality, the paper concludes with a set of experiments based on those analyses of existing projects and provides documentation for those who might wish to develop ideas presented here.

\section{Retrospective}

Hacking is pervasive \cite{paradiso2008}. Open source hardware practices, which share a deep link to hardware hacking \cite{williams2012}, are the bridge between longstanding methods in electronic music instrument designs and innovations in the field of accessible invention and fabrication. Through the work undertaken in this project and the writing that came out of it, it was most important to convey that making instruments could be as deeply personal and creative as composition, with endless inspiring parallels, an activity no longer reserved to academics or professionals. The goal isn't to create perfect products, rather, it is to produce something you are happy to use for musical purposes, regardless of its hypothetical shortcomings. In an age where the White House organizes its on maker faire and half of the documented projects are artistic (White House, w2014), this work will not be alone.
	

