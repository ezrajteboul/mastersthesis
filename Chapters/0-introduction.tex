% ----------------------------------------------------------------------------------------
% INTRODUCTION
% ----------------------------------------------------------------------------------------
\chapter{Introduction}\label{introduction}
\lhead{\chaptertitlename\ \thechapter. \emph{Introduction}}
% ----------------------------------------------------------------------------------------
The present document is concerned with current practices of professional electronic music instrument makers, here referred to as \textit{silicon luthiers} \citep{collins2008}. Specifically, it offers a technically literate analysis of modern day hardware in which approaches to circuits and interfaces not following engineering methodologies are intentionally undertaken to offer different and inspiring tools for composition and performance.   

This section introduces historical prompts that justify an analysis of musical technology  through Dunne's concept of post-optimality \citep{dunne2005}. It then details how this point of view can theoretically address some current issues arising in the design fabrication of electronic music instruments. This discussion is finally used to frame the scope, structure and motivations for this thesis. 

\newpage

\section{Toward a Post-Optimal Aesthetic} 

\begin{quote}
	... When's there not a black and white keyboard you get into the knobs and the wires and the interconnections and the timbres, and you get involved in many other aspects of the music, and it’s a far more experimental way? It’s appealing to fewer people but it’s more exciting.
\end{quote}

Donald Buchla, in \citep{pinch2001}

Buchla, in designing his ``music boxes'', describes himself as a builder of instruments rather than machines. The decision not to include a familiar keyboard-format controller in his modular synthesis system was influenced by an appreciation Cage's and Tudor's innovative performances, to whom Buchla sold his first products. He made this choice fully aware of the consequences this might have on his business and popularity \citep[p.44]{pinch2001}. 

Nicolas Collins, student of Tudor, contextualizes the consequences of this type of design philosophy in contemporary electronic music instrument design. Through his book \emph{Handmade Electronic Music}, he encourages users to experiment with electronic hardware in order to be familiar with musical tools not necessarily on a technical basis, but rather, on an experiential basis \citep{collins2006}. By offering an experimental approach to electronic music devices, Collins offers a path to becoming a craftsperson of electronic music which does not require a professional engineering background \citep{collins2006,collins2008}. 

In effect, the approaches taken by Buchla and Collins suggest that electronic instruments do not benefit from being reduced to a set of performance requirements for a circuit, algorithms, or both. In their own ways, both practitioners legitimize the claim that electronic music and its musicians can benefit from small-scale, personal approaches in the design and manufacture of instruments. Viewing those devices as machines which should be optimized for the same metrics as other engineered products (efficiency, reliability, solvability etc.) does not do justice to their unique roles as catalysts of poetic experiences: interesting results arise when designers explicitly address electronic devices as post-optimal objects \citep{dunne2005}.  

Dunne's concept of post-optimal object is best resumed here by the following quote: 

\begin{quote}
	If user-friendliness characterizes the relationship between people and the optimal electronic object, then user-unfriendliness, a form of gentle provocation, could characterize the post-optimal object. 
\end{quote}

\citep[p.xviii]{dunne2005}

The idea that a physical item can better assist musical functions when it presents some forms of limitations finds precedents in theoretical discussions of electronic music instrument design \citep{evens2005,rovan2009}. Engineers implicitly acknowledge this separation between what is theoretically preferable and what is used in practice by offering products based on inefficient, dangerous or unreliable technologies such as vacuum tubes amplifiers or synthesizers \citep{barbour1998,hamm1973}. 

In explicitly addressing non-standard approaches to instrument design, it is useful to consider Perner-Wilson's kit-of-no-parts framework. In engineering, the kit of parts describes ``an approach to designing discrete components that function as modular parts within a coherent system'', where these ``parts have been optimized for speed, efficiency, and repeatability of assembly'' . The kit-of-no-parts approach then encourages users to think of items as the product of a personal craft, where understanding a material down to its components can foster expertise through ``skilled use of tools, intimate knowledge of materials'' which in turn allow for ``more diverse and intelligible results.''\citep{perner2011}

In discussing the post-optimal aspects of some electronic instruments, this work breaks down designs into three levels, all or some of which can exhibit post-optimal characteristics: the circuit level, the system level, and the interface level.

In effect, post-optimality is already implemented by individuals in the field when they rely on antiquated technologies to inspire new music \citep{pelousek2014}, or when they fabricate hardware that enables personal approaches to electronic music composition and performance \citep{haslett2005,armstrong2006}.  

That personal, inspiringly imperfect vision of electronic music instrument design materialized in various forms through the 20th and 21st century: invention, do-it-yourself (DIY), hacking, circuit bending. Today, this set of historical practices joins back with trends in general hardware design, with movements such as the open source and maker movements ~\citep{mellis2014,perner2011}. Practitioners meet in FabLabs, which replace the garage and basements of previous generations ~\citep{mellis2011}. Both appear as symptomatic of a similar interest across user-bases: a desire take advantage of emerging technologies to create personal, more inspiring experiences for themselves and their audiences ~\citep{hermans2014}. 

The primary goal of this thesis is to contextualize and analyze the work of designers who fit within this vision of audio hardware, aware of precedents but challenging tradition and expectations through inspiring if imperfect uses of electronics.

\section{Motivation}

This work exists because of a deep curiosity for the instruments that makes a music electronic, their creators, and their place in today's art world. It hopes to foster future interdisciplinary research in the field of electronic music hardware and encourage open, experimental devices that blur the line between composition and design. 

DIY music instrument communities are based on sharing designs, advice and results. Successful resources document projects from start to end, with more than enough information to tackle any eventual mistake. These attributes find clear parallels in open source hardware design practices, as defined by the open source hardware association \citep{oshwa2015}. By documenting and contextualizing relevant projects, one can contribute back to both communities. Furthermore, presenting this material in an academic context helps link these practitioners to some of the driving forces behind those emerging technologies. Some of the devices that have encouraged those communities still require strong corporate or academic backing (Arduino, Raspberry Pi). 

This survey of design practices in musical technologies should also illustrate an aspect of the interdisciplinary nature of music technology, allowing further comparisons and connections with other designers, trends and fields. By rooting contemporary electronic music practice to the hardware that enables it, a stronger bond to the developing field of sound art is favored \citep{cluett2013}. A more explicit connection is drawn to the people who make audio electronics, whether they are boutique designers or factory workers \citep{rylan2015}.  Issues arising from a direct application of engineering methods based in market economies and cult of performance can be better identified, addressed, and solved for the further development of inspiring interfaces\citep{ghazala2004,christensen2005,Feldman2007,silver2009,perner2011,hertz2012,riis2013,jackson2014}.

\section{Scope and Structure}

This work wishes to contextualize and analyze a selection of audio electronic systems as products of a technological, cultural and social environment. By acknowledging the products, prototypes, tools, people and environments involved in this process, a more accurate description of these ecosystems of invention can be achieved \citep{vinck2003}. This, in turn, allows for a clearer view of the possibilities and futures for the field of devices for musical expression. 
	
For this clear description to emerge, this document will introduce a cursory history of formative practices paralleling the development of modern electronics \citep{holmes2002}. Starting with the isolated forefathers of the field and their ``spirit of invention'' \citep{dunn2001}, useful references through to Collins' \textit{Handmade Electronic Music} will allow a presentation of concepts of component design, system design and interaction design in a musical context. In this discussion, do-it-yourself, hacking and circuit bending are important movements both in terms of general design and musical practices. This thesis then presents the current incarnation of this spirit of invention and experimentation, in the greater context of making and musical cultures. That presentation consists of an analysis of specific projects from the ten years, further contextualized by interviews with their authors. Open innovation made and make electronic instrument design an ecosystem of invention, and this ecosystem is thriving. 

	
	
 


	

