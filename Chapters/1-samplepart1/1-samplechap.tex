% ----------------------------------------------------------------------------------------
% CHAPTER TITLE
% ----------------------------------------------------------------------------------------
\chapter{Experimental electronic systems in the musical arts}\label{background}
\lhead{\chaptertitlename\ \thechapter. \emph{Background}}
% ----------------------------------------------------------------------------------------

If experimental music is unaware of its result until after the facts, transferring the concept over to engineering could appear as counterintuitive. Great effort goes into making commercial electronics consistent, predictable and uniform. The same largely holds for electronic instruments - \"one-offs\" with cryptic functions of interfaces drastically reduce the overall adoption of the device, and therefore, their commercial potential \cite[p.5]{haslett2005}. In this context, the gap between the determinism of product design and the desire for originality of musical performances is bridged by the musician. \"Finding your sound \" becomes as much about finding an original way to use an instrument as finding an original instrument.  

Making, modifying or otherwise altering electronics is one option for that musician to both better understand the tools at their disposal and make the most of them. Numerous historical precedents for personalized instruments will be presented in the upcoming chapter, while contemporary instances will be developped in the next. All of them were developed out of a concern for originality, costs, or simple curiosity. Throughout those examples, we'll see how electronic instrument design practices have developed from the example set by the electron pioneers, explored the different levels for experimentation within that space, and how those modern practices converge with contemporary digital technologies, towards an increasingly personalized musical field of expression \cite{hermans2014}.

\section{an organology of electronic instruments?}  

In understanding the development of electronic music instruments, it appears as important to consider at first each of those terms separately. If the development of electronics defined what could be done by artistically-inclined tinkerers, pre-existing instruments and musical traditions shaped what those tinkerers implemented. The influences of acoustic music traditions are still clearly visible on electronic music today, and one could consider the latter as both a continuation and response to the former. 

The fabrication of acoustic instruments is a craft. Organology, formalized in 1914 by the Hombostel-Sachs system \cite{von1961}, was an attempt at classifying musical instruments of the time. By considering the mechanical differences between each instruments, the system separated music-making devices into four categories: 

\begin{itemize}
	\item idiophones, or self-supporting vibration systems
	\item membranophones, with a vibrating membrane
	\item chordophones, using vibrating strings
	\item aerophones, relying on vibrating volumes of air
\end{itemize}

The materials and their geometry clearly define a mechanism, and that mechanism defines the instrument. This is a purely physical parameter space, which takes advantage of the complexities of the real world to develop sophisticated systems. 

What separates acoustic and electronic instruments is also what limits organological classification. Indeed, it is difficult to expand to electronic instruments because the physical phenomena behind oscillations (sound) in these instruments is more difficult to categorize. In attempts to modernize organology, electronic instruments are simply designated as electrophones \cite{hugill2012}. As Hugill notes, this system does not give justice to categories such as extended acoustic instruments, gestural interfaces, and infra-instruments \cite{bowers2005}. It appears difficult to design a robust classification system without shifting from a focus on mechanical engineering to an approach referencing electrical and mechanical systems equally \cite{haslett2005}. 

	Electronic music redefined the concept of what an instrument could be and how it could be conceptualized. This discussion focuses on the expanded skillset that comes with the introduction of electro-mechanical hybrids in the matter of creating sound, and exposes some of its consequences. 

\section{circuit design. innovation at multiple levels: component, system, interaction}

The design parameters of electronic instruments is a newer space. It is defined by the electronic components available, which has two consequences for every electronic instrument maker since the rise of publicly available electronics in the nineteen-twenties:

\begin{itemize}
	\item electronic instrument and their pre-manufactured electronics offer an additional degree of separation from the organic world. 
	\item design plans go from being a representation of the final product to an schematic abstraction. The two offer different liberties in the fabrication process.
\end{itemize}
	
The question of experimentalism in electronic music hardware implies the following: to what extent has the standardization of resistors, capacitors and other basic electric building blocks standardized the space of possibilities for electronic music instruments? 

The design of those instruments includes elements of electrical engineering and product design, both of which can be informed by some musical tradition or system. The result is, at least, a schematic. A complete schematic provides a concise set of instructions that allow anyone to materialize that design by assembling parts. 

To materialize that design, a schematic will turn into a bill of materials (BOM) describing which parts to order, a method for connecting those parts, and a design for how to house the electronics and make the eventual controls available to the user through a specific physical layout. 

This process is often done with computer-assisted design (CAD) tools, which turn any of the above documents into digital files. The electronics themselves can also be augmented by digital signal processing tools, which transfer some of the work from hardware design to software writing. 

Design is complemented by fabrication, which will take the information described above and turn it into a physical item. There again, computer-assisted machining (CAM) tools have recently become commonplace. 

The only difference between audio hardware and every other electronic device lies in the choice of circuit topology, specific part values, and/or algorithms it implements - here, specific applications arise mostly because of the designer’s intent. The outstanding majority of parts in electronic music hardware find applications in non-musical related tasks. The laptop is the embodiment of a general-purpose device that can be configured to fulfill a variety of tasks including audio. In doing so, it almost entirely reduces electronic instrument design to software design. 

Software design offers a rich tradition of experimentalism for audio. Max Mathews acts as the forefather of a long line of extensively-studied composers who either programmed their own instruments, or worked closely with engineers to do so. Those efforts effectively redefined the world of electronic music, and as we'll see excluding digital from a discussion of contemporary audio hardware in music would be as toxic as it would be difficult. Open source originates in software- this thesis wishes to expose the possibilities of interdisciplinary design approaches. 

	By looking at the role of experimentalism in electronic music hardware, one can get a sense of how hardware can complement the speed of innovation that characterizes the digital domain. Improving physical music devices can take three forms:  
	
\begin{itemize}
	
\item component innovation

\item system innovation 

\item interaction design  

\end{itemize}

Historically, electrical engineers have been concerned with those first two points, while the latter is more open to interpretations from various sources. The New Interfaces for Musical Expression (NIME) conference has developed out of a desire to unite efforts in that field. 

Looking at the development of those three types of innovation in electronic music and electronics gives us a better sense of where the field is today. 

\subsection{electronic music as invention}

	Before 1915 and the beginning of commercially available vacuum tubes, electronic music relied on a spirit of adventure and experimentalism close to the later works of Tudor, Kuivila, Collins and Ghazala. Dunn's pioneers developed system because they invented or developed interesting forms for the basic components of an electronic circuit (the resistor, the capacitor, the inductor): 
	
\paragraph{C.G. Page}

	C.G. Page “toyed” with horseshoe magnets, spools of copper wire and batteries - in 1837, he would publish one of the first reports of electronic sound, which he called galvanic music without being able to explain it \cite{page1837}. 
	
\paragraph{Elisha Gray}

	Elisha Gray, in 1874, was the first electronic musician to tour his home country. His invention, the musical telegraph, was invented after he realised he could control the pitch of the hum produced by a vibrating metal strip attached to his bathtub while running experiments with his nephew. After touring once with it, he ignored its musical potential and failed when he attempted to use a modified version of the device as an early iteration of what would then become the multiplexer \cite{holmes2002}. 
	
\paragraph{William Dudell}

	William Duddell’s 1899 “singing arc” offered pitch control for the audible hum of carbon-arc light bulbs. Here again, the musical application was coincidental - he was originally trying to get rid of the unwanted buzzing sound \cite{nasmyth1908,holmes2002}. 

\paragraph{Thaddeus Cahill}

	Thaddeus Cahill’s Telharmonium, patented in 1896, was the first successful massive undertaking of electronic music hardware. In its full form, the two-hundred ton instrument was a sophisticated electro-mechanical polyphonic instrument, developed and assembled for the sole purpose of entertainment. With Cahill rises the persona of musical inventor as businessman, effectively becoming one of handmade electronic music’s earliest father figure \cite{holmes2002}. 

	This short list describes only some of the inventors who were much comfortable with raw materials and physical experimentation than most of today’s electro-acoustic or digital music community. Interestingly enough, none of the signal generation methods described above (electromagnetic oscillation, RLC-circuit oscillation, spark-gap oscillation, electromechanical oscillation) are common in the following generations of electronic music hardware experiments. 

	With electronics still in their infancy - no design tradition other than what could be borrowed from acoustic instruments, no formal structures other than the companies entrepreneurs like Cahill would try to build -  experimentation is the only option. This is the origin of electronic music, where the promise of the new medium was enough for people to blindly experiment. 

\section{electronic music as an institution}

In the period following the first world war, the popularization of radio went with the development of hobby electronics. It was more cost-effective to buy a kit and build a radio yourself rather than purchase the completed product. People were not a afraid to open things up (Collins, \ref{AppendixA}). In 1922, a Freshman “masterpiece” radio cost \$17 as a kit, while a completed set cost \$60. This corresponds to \$240 v. \$850 in 2014 (radioblvd.com, w2014; data.bls.gov, w2014). 

This was facilitated in the U.S.A. by companies like Radio Shack and Allied Electronics. Both companies, amongst many, sell the parts - resistors, inductors, capacitors, transformers, tubes - and the tools necessary to assemble a variety of consumer audio and radio electronics. 

Radio Shack’s first catalog (w2014) from 1939, contained 80\% kits, parts and tools and 20\% completed products. The increasing availability of parts and tools solidified after the second world war, with catalogs like Radio Shack’s growing from 1939’s 72 pages to 110 in 1946. Heathkit, a major electronics kit focused on high fidelity audio and radio equipment, was founded in 1947. Popular Electronics #1, which compiled articles on those kits and the various technological developments available to the patient home-builder, was published in 1954. 

This resulted in a generation of engineers and musicians growing up with electronics, who were both were eager to take advantage of the relatively positive post-war atmosphere \cite{holmes2002}. 

Electrical engineering was at that point one of the main research areas in the western world. In that context, the final edition of the Radiotron Designer's Handbook \cite{langford1953} dedicated a third of its 1500 pages to audio electronics. Schematic, tools, methods and designs are optimized, thoroughly investigated and standardized. Institutions serve as the breeding ground for most renowned artistic applications of technology, resulting in the emergence of the French (RTF), German (WDR), and Italian (RAI) experimental music studios, all housed by public radio stations. In the U.S., RCA and Columbia University composers collaborated to develop the RCA synthesizer, closer to the telharmonium in scale than anything in the first half of the 20th century. 

\paragraph{Hugh Le Caine}

In Canada, Hugh Le Caine’s Electronic Sackbut and its concept of voltage-control proved to be an inspiration for the Moog, and a foreshadow of most electronic music systems. His work was developed through a contract with the Canadian National Research Center, he was then employed by University of Toronto and McGill (Montreal) to develop equipment for their studios \cite{holmes2002}. 

Regardless of the motivations of their manufacturers, this first wave of kits and tools empowered the masses. It also placed the 1950’s western techno-musical avant-garde in corporations and institutions. Most of the technology used in those musical systems was a byproduct of the war research effort, it seems logical for the first wave of artists and designers to be involved with those organizations. 

Two relatively well documented exceptions exist in the U.S:

\paragraph{Raymond Scott}

Raymond Scott was a self-taught hobbyist with the additional motivation, funds and time provided by his successful career as a composer of electronic music for commercials. Having independently developed one of the earliest multi-track tape recorders (7 to 14 tracks), amongst a number of other inventive projects, he would also ultimately receive visits from an impressed Robert Moog. However, he would fail to turn his technological expertise into a successful business \cite{holmes2002}. 

\paragraph{Louis and Bébé Barron}

The second exception was Louis and Bébé Barron, a couple who would provide soundtracks for films such as 1956’s Forbidden Planet, were innovative in their implementations that they would describe as “they were alive”, an interesting parallel with Scott’s Electronium. In the tradition of Duddell and Gray, the circuits that Louis built were accidently overdriven and eventually failed. In recording that decaying process, the sounds created by circuits inspired by the work of Norbert Wiener would be used as the basis for their composition, which they assembled in a way similar to Schaeffer’s concrète process \cite{dunbar2010}. 

Similarly, John Cage’s experiments throughout the 50’s would incite his collaborator David Tudor to abandon the piano that brought them together in favor of an experimental approach to music through homemade electronics (Holzaepfel, 1993; Collins, 2004). Unlike our two previous examples, however, Tudor and Cage would have strong affiliations with academia and indirectly, industry. 

\section{electronic music as a living instrument}

\subsection{Tudor and Cage: electronics speak for themselves}

	Variations II is a 1960 piece by John Cage, “the greatest degree of abstraction of a compositional and notational model that Cage developed over the period from 1958 to 1961.” It was extensively studied, then performed in 1961 by David Tudor, for whom Cage effectively wrote the piece. The extent to which the work was internalized by Tudor has led to some to consider him as the piece’s co-composer \cite{pritchett2004}. It included the use of a complex system of signal processing electronics, designed to implement some of Cage’s ideals of compositional indeterminacy. “You could only hope to influence the instrument”, said Tudor in describing his use of the device \cite{nakai2014}. 

Tudor’s debuts are not independent of public or private groups. Fluorescent Sound is commissioned in 1964 by Robert Rauschenberg and sees its sole performance in Stockholm’s Moderna Museet. Bandoneon!, his second piece, premiered in 1966’s Nine Evenings of Theatre and Engineering. Instigated by Rauschenberg and facilitated by Bell Laboratories, this event gave artists from the american avant-garde a chance to collaborate with some of the world’s most productive engineers in New York’s Armory building \cite{kuivila2004}. The collaboration would prove instrumental in fostering the experiments in art and technology (EAT) series. 
	
Reminiscent of Barron’s living circuits, Tudor described his second piece as “composing itself out of its own composite instrumental nature.”  \cite{tudor,kuivila2004}. Tudor, through various collaborations, develops an affinity for circuits which ultimately make some of the compositional decisions themselves. 

Cage had been using “found” electronics in performance since the late 30’s, with pieces such as the Imaginary Landscapes series or 1959’s Water Walk. Reich’s 1965 It’s Gonna Rain proved that process-based composition rooted in solely in technology held musical value. Tudor, with his implementation of Variations II, Bandoneon!, and the Rainforest series, legitimizes a compositional approach based explicitly on materiality. “The objects should teach you what it wants to hear”, he states after performing Rainforest IV (1973). Cage will echo the statement in 1987 with the following response: “the components, the circuitry is the music, and it comes alive when it is performed” \cite{nakai2014}. 

Tudor was in a unique position of artistic experience and legitimacy that enabled his experiments to become regarded as groundbreaking and foundational (Collins, 2004). Through collaborations with Gordon Mumma, David Behrman, Hugh Le Caine, John Fulleman, and John Driscoll, and thorough personal investigation, he gathered enough experience to masterfully implement one of the first documented uses of chaotic electronics in music \cite{kuivila2004}. Just like Cage, he would not be attached to the idea of considering himself a composer \cite{kuivila1998}. Tudor is the model for multiple generations of hobbyists, independent scholars and experimenters who situate themselves somewhere between self-taught artistic systems designer and musician. 

A common thread in Tudor’s work was the spotlighting of electronics customarily hidden during performances, \"composing inside electronics\" \ref{appendixA}. In trying to take advantage of the sculptural aspects of the instruments through sonification by using contact microphones or transducers, his work also developed a notion of sound art which challenges the distinction between installation and performance \cite{driscoll2004}. His approach was different from Lucier’s: where the latter focuses each piece on a specific physical phenomenon, Tudor’s work was rarely concerned with minimalist systems. \cite{collins2004,driscoll2004}. 

Another manifestation of his originality was his coercion of general-purpose abstraction borrowed from electrical engineering, then adapted for the performance of his pieces and equipment \cite{kuivila2004}. This repurposed set of abstract notations would blur the line between schematic and score, just like Cardew and Earle Brown proposed the use of abstract art as basis for musical performance. In doing so, his performances became more and more personal - Tudor’s oeuvre is “practiced, not preserved” \cite{kuivila1998}. This “musical practice based on constant modification and innovation” \cite{driscoll2004} is a direct foreshadowing of the methods which define online do-it-yourself communities active today. 

If Tudor defined a practice of experimental electronic music systems that complements Cage’s sonic uncertainty, it is largely thanks to semiconductors (Collins, in\ref{AppendixA}. Transistors and integrated circuits offered the functionality of vacuum tubes without the latter’s size, weight, price and high voltage hazards. These became commercially available and well documented as he started using custom electronics \cite{collins2004}, with Texas Instruments selling their first silicon transistor in 1947 (texas instruments, w2014) and integrated circuits (ICs) becoming available in the 1960’s.  

The easily replaceable, cheaper components would make prototyping musical electronics more accessible, also allowing Tudor to easily enroll help for performances. Starting with 1973’s performance of Rainforest IV, his group of collaborators would solidify as the Composers Inside Electronics group (C.I.E., w2014). The group, reformed for Tudor’s death in 1996 is still active today. Their approach is resumed by Tudor in 1976: 

\begin{quote}
						
Electronic components and circuitry, observed as individual and unique rather than as servomechanisms, more and more reveal their personalities, directly related to the particular musician involved with them. The deeper this process of observation, the more the components seem to require and suggest their own musical ideas, arriving at that point of discovery, always incredible, where music is revealed from \‘inside,\’ rather than from \‘outside.\’ 
						
\citep{tudor1976,nakai2014}

\end{quote}

Tudor’s innovative methods and C.I.E.’s ever growing cast have guaranteed the relevance of their work in scholarship of musical hardware leading up to the current decade \cite{collins2004,collins2006,collins2008,collins2010,nakai2014,driscoll2004,kuivila2004}. Through the learning, supplying and sharing tools offered online, Tudor’s ideals of experimentation and collaboration have come to be more relevant and accessible than ever.

\subsection{Brian Eno, Frippertronics and unintentional realizations}

\subsection{prototyping electronics for the art}

\subsection{A performative tradition of experimentation}

\section{Hacking} 

	The essence of Tudor’s successors’ would be academically captured by 
former C.I.E. member Nic Collins. His Handmade Electronic Music book was first published in 2006, presenting an extensive amount of information on homemade electronic instruments with insight from years of experience, references and sources. By completing this project, Collins not only proved that blending academic, commercial and hobbyists attitudes could be successful in all three of those areas, but also links decades of practices in the do-it-yourself electronic music world to the “maker” movement, which started to coalesce around a similar time with publications such as “Make” magazine.

	The values of this book - concepts of imagining, prototyping, assembling from your home - find a strong precedent in pre-industrial craftsmen and the world of small production \cite{collins2006,ghazala2005,kuivila2004}. 

Although Collins’ introduction hints at the advantages of this approach in the context of electronic music, the advantages of a craft approach to instrument design and fabrication can be further resumed in the potential for personalization, transparency, and skill-transfer \cite{perner2011}. Conversely, the fascination for Collins’ hacking or Ghazala’s more clearly defined circuit-bending can be explained as a desire to fill a need unsatisfied by commercial products \cite{dunne2005}, a process that has shown its ability to serve as the source for entire sub-genres of the musical arts \cite{dunne2005,kelly2009,novak2013}. 
\begin{quote}
“The circuit— whether built from scratch, a customized commercial device, or store-bought and scrutinized to death— became the score.”
\citep{collins2004}
\end{quote}

	Discussing Tudor and Mumma and Kahn, Collins describes the origin of his interest in music hacking, which references the origin of experimental electronics as a legitimate ground for musical composition: 

\begin{quote}
“I learned from Tudor and Mumma that you did not have to have an engineering degree to build transistorized music circuits. David Tudor’s amazing music was based partly on circuits he did not even understand. He liked the sounds they made, and that was enough.” 
- David Berhman \cite[p.ix]{collins2006}  
\end{quote}

\subsection{the art of hardware hacking}

A look at the structure and content of the work reveals six parts: starting, listening, touching, building, looking, and finishing”. Those consist of between two and eight chapters, for a total of 30. Each covers a specific theme such as “tape heads: playing credit cards” or “a little power amplifier: cheap and simple”, revolving around a few schematics, diagrams, guidelines and suggestions for implementations. Historical background is added when appropriate (“David Tudor and Rainforest”, p40; “Circuit Bending”, p91…). 

There are more pictures than schematics, and what should be striking to anyone already familiar with building electronics is that there rarely are any values or names for parts in schematics - another consequence of Tudor irreverently defacing the key document of electrical engineering. Devising 264 pages of electronics for music involving solely circuits simple enough to describe in a few lines of text appears as a feat in itself. The self-deprecating “keep it stupid” attitude of the introduction rapidly turns into rules number 1 and 2 of hardware hacking, “Fear not!” and “don’t take anything apart that plugs into the wall” (p225). In other words, an electromechanical understanding of electronic music is empowering, and you should experiment as long as you do not risk hurting yourself. This hints at another trope of the open hardware movement, mentioned in the introduction: simple, homemade hardware facilitates sharing and learning, thereby encouraging its practice.

\begin{quote}

“Finally Tim-Berner’s Lee birthed the World Wide Web and a hundred Fuzztones flowered.” 
	
\cite[p211]{collins2006}

\end{quote}

This book can almost serve as an entirely self-contained introduction to electronics in music (falling just short of a soldering iron and a speak \& spell). Collins however made sure that it also is aware of the resources that can compliment it through a list of printed hyperlinks that is fairly exhaustive for its time. In that sense, “Handmade…” also relates to more general current practices in open hardware design. Indeed, it is precisely the open sharing platforms allowed by the world wide web that have fostered the communities which collectively comprise the maker movement. The appendices also reflect loosely broad categories of resources important to both practices: previous documented work. Appendix B appears in retrospect as a paper instructables or hackaday, if every project was hosted on a different homepage. Resources for buying parts and tools are listed in appendix C, where the essentials have not changed and already included allelectronics, jameco, and radioshack. Finally, inspiration: appendix E describes the tracks on CD sold with the book and appendix D lists the rules of hardware hacking and the avant-garde. 
	
	Before discussing the influence of Collins’ book on following publications in related books and articles, it seems important to mention the availability of the book in various digital forms. The original draft for the book, a compilation of class notes, is freely available for download off of the author’s website (w2014). The first hit for “handmade electronic music pdf” on most search engines will give a pdf of the 2006 edition of the book of dubious legality, but extremely easy to find nonetheless. 

	By tolerating or passively encouraging open access to resources, Collins contributes directly to the community he has helped shape. More than writing the book on hardware hacking for non-engineers, he’s an essential force in making open hardware design the self-sustaining cycle it aspires to be through the maker movement. By publishing this through a large company while in a professional academic and musical position he also lends the weight of a more widely recognizable figure to a movement and methodology that challenges the usefulness of those very institutions.

\subsection{impact:}  

\begin{quote}

“Collins’s Handmade Electronic Music details many ways in which human interaction can be built into music and sound making devices.”

\cite{mills2010}

\end{quote}

	Measuring the influence of this book through simple academic metrics, such as the number of times it is cited in publications following it \cite{harzing2008} yields the following results. “Handmade…” seems to have been referenced in 88 publications (google scholar). For comparison, Road’s “Computer Music Tutorial”, published 10 years earlier, returns 1267 citations, the “Art of Electronics” (a standard circuit design text from 1989) returns 3640, and Gharzala’s “Circuit Bending” from 2005 returns 45. 

	When it is cited, “Handmade…” is rarely commented upon directly. It is referred to in surveys of contemporary sound art, music, and music technology practices \cite{kelly2011,mills2010,pigott2011,rodgers2010}, and as an inspiration in the development of a specific musical controller \cite{ariza2007,hoadley2010,murphy2010,riis2013,valle2011}.
	
	Overall, the academic impact of the work is fairly confined to the field it wished to solidify (music hardware hacking). Its varied content originated from a set of lecture notes, which have since found their place in other college level classes at other institutions. 

On a community level (“do-it-with-others”, diwo, or “do-it-together”, dit), Collins’ impact is difficult to measure objectively. Highly frequented music hardware hacking forums such as Experimentalist Anonymous, diystompboxes, freestompboxes.org, and electro-music.com all contain mention and praises for the accessibility and simplicity of “Handmade…”, but results usually number between 5 and 50. As a reference, searching for Ray Wilson’s popular “music from outer space” do-it-yourself synthesizer website returns between two to three times more results. Collins’ website numbers the quantity of workshops he’s given based on the book since 2004 to “dozens”, meaning that a significant portion of the sharing could still be happening in-person.  

This appears as indicative of the last, and arguably most important point about homemade electronic music instrument design as part of more general arts & crafts \/ diy \/ maker movements. They are fueled by personal engagement and, more importantly, passion. In those circumstances, theorizing, documenting and commercializing is not the priority of the majority but rather the duty and privilege of the few who turn their passions into a viable working position. In doing so, they are given the choice of where they wish to place themselves between their aspirations and that of their community. 

\section{re-commercializing diy: the maker movement}  

The latest iteration of homemade electronics culture, the maker movement, is described as a third industrial revolution (The Economist, w2014). Here, Fab Labs, modelled after MIT’s NSF-funded center for bits and atoms, allow the public to benefit from some of the efforts of academia to give fabrication a place in everyone’s lives \cite{padfield2014,blikstein2013}. Through this democratization of invention \cite{blikstein2013}, diy culture expands its reach from sunday inventors and engineers with some free time to a wider audience which could include more and more artists. 

	Often, compelling uses of rapid prototyping (3D printers, cnc mills and other computer assisted manufacturing techniques) are artistic. D-Shape’s large-scale concrete 3D printer is advertised using a sculpture by Andrea Mongante’s Shiro Studio (Shiro Studio w2014). Makerbot’s frontpage for the replicator displays a picture of the device with a completed red plastic bunny on the extrusion platform - this use falls in the artistic range rather than the utilitarian. 

In June 2014, the White House held a maker faire, with president Obama declaring June 18th “National Day of Making” (White House, w2014). Of the 20 projects displayed on the event’s website, 9 were artistic (including 2 instruments: a violin and a banana-synthesizer), and most displayed some level of aesthetic concern. If the third industrial revolution is a democratization of invention, is that the place the arts (and specifically, sound devices) will have? Does a modern version of 9 Evenings exist, and if not, what would it look like? 

On the sidelines of the maker movement, the technology has changed (shifting to digital or mixed-signal systems) along with its means of delivery. Radio Shack and its ilk have lost most of their importance, with online suppliers like Jameco, Digikey and Mouser providing large parts catalogs. Specialized marketplaces like sparkfun and seeedstudio occupy more hobbyist-oriented markets. To compare, by 1970, a Radioshack catalog offered one or two pages , or around .5\% of the catalog for kits, while thirty pages (20\%) are dedicated to parts. In 2002, the last year the company offered catalogs, it contained 25 out of 450 pages of parts, and no kits. The company is currently closing 1100 of its 7000 locations. While this is in no way due solely to the maker movement, Radioshack’s slow death or reconversion does mark the end of an era. 

The rise of accessible embedded systems (arduino, maple boards, raspberry pi, beagleboards…) and availability of relatively free music software (pure data, chuck, csound, supercollider…) leaves music hardware often reduced to the role of place holder for more malleable software. This comes with its own set of design principles (user experience / user interface design), resulting in versatile controllers meant to be used with a computer (monome, linnstrument). With wearable circuits comes the ability to turn most of anything into a controller using the plethora of sensors available to today’s tinkerer. 

Regardless of objective efficiency and versatility, interest in analog sound generation and old-fashioned interfaces is commonplace in electronic music \cite{collins2006}. The range of possibilities for musical expression has become significantly more fragmented. Each approach, whether it is controllerism, hacking or analog traditionalism is being pushed to their extremes. In exchange for this splintering, online resources provide better documentation than ever before.

The main academic platform for music hardware after 2002 is NIME, which shares all proceedings freely. Amateurs find more informal information on a collection of public or semi-private forums (diyaudio, electro-music forum), databases from some type of digital format of designs (hylander), and repositories ran by benevolent individuals that sometimes also try to run small businesses (music from outer space, ken stone).

Most general hardware hacking websites (hackaday, instructables, arduino projects site) contain a significant number of audio hardware projects (hackaday audio, instructables audio, arduino audio forum). It could be said the tools and knowledge necessary for audio hardware designs are more accessible to anyone with an internet connection and the time to teach themselves electronics and programming. In addition, academia is heavily advertising the potential of open source design and manufacturing tools. However, there is very little organized measure or monitoring of the reach of these technologies. Do they truly beneficiate learners and beginners, or do they simply empower some artists and scientists already working on similar projects? Do they only reach people in higher education or with a higher education, or do they attract mostly people already interested in open access technologies? 

The following sections cover two methods of inquiry for these questions. First, a series of interviews with various point of views in the art and technology world allow us to get some relevant snapshots of current practices. Secondly, responses to those projects and adaptations are devised and documented. An analysis of the process and the efforts undertaken to share those designed follows. 

% ---------------
% To incorporate in this chapter
\begin{unsortedStuff}	
\section*{(TO INCORPORATE)}
	\begin{itemize}
		\item 
	\end{itemize}
\end{unsortedStuff}
		
%Blank page to add written thoughts
\begin{optBlankSpace}
	\newpage
	\mbox{}
\end{optBlankSpace}

