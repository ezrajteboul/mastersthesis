% ----------------------------------------------------------------------------------------
% Conclusion
% ----------------------------------------------------------------------------------------
\chapter{Conclusion}\label{conclusion}
\lhead{\chaptertitlename\ \thechapter. \emph{Conclusion}}
% ----------------------------------------------------------------------------------------
 
\section{Summary}

The previous chapters have made explicit the prevalence and promise of post-optimal objects within the practices of contemporary silicon luthiers. Furthermore, it post-optimality could be thought of in some cases as a latent characteristic of any musical technology, which could be exposed through hacking, bending or repurposing. 

A clear progression from financial and scientific insecurity in electronic music research is visible, as today practitioners with varied levels of qualification make a living as musicians, business owners, teachers and artists. 

A number of those identify a long-lasting interest in the promises of those unpredictable, post-optimal chaotic systems, possibly as a counterpart to the surgical and consistent responses of highly engineered audio tools. The social aspect of the practice is also non negligible, with a number of connections being made through Collins' publication and practice, other workshops, histories, and archived material. 

Perhaps most usefully, this thesis illustrates that incomplete or very informal records of past and current hardware work can be studied to derive functional approximations and complete analyses of electronic music devices. Snazelle's claim that designers have access to much more than he did even just a decade ago appears to be accurate. 

No common motivation was clearly identified in trying to understand why exactly people still tackle hardware design issues when much of these systems could be simulated entirely in software. Paraphrasing Collins, there simply is a tendency for a fringe of the musical population to appreciate the materiality of the tools that allow make their art possible, and the variety of specific motivations give each device a better chance to be unique and innovative. 

In 1950, Lester Paul's radio program explained his peculiar guitar playing sounds with a device, the \emph{Les Paulverizer}. Paul, as a noted enthusiast for innovative uses of technology in music, described this machine as one which could enable him to use ``one guitar, and make it sound like six''. The Les Paulverizer, in reality, was a farce. Its result was fabricated through the use of multiple tape machines and playback speed manipulations ahead of time, then synchronized with Paul's movements on stage. 

Living with the lie of the Paulverizer, its author eventually developed a system that could implement some of its fabled capacities in real time. The \emph{real} Les Paulverizer was in effect a foreshadowing of controllers: knobs and switches made available to the musician to send controls offstage and trigger various effects\citep{kane2014}. 

Brian Kane makes clear that Paul's style and reputation relied largely on technological innovations such as the Les Paulverizer. Arguably, Paul was successful in playing with this subterfuge, letting it run until it led to popular success and ultimately motivated very real technical innovation.

 All their efforts and devices were also quite real, but the example of the Les Paulverizer reminds designers that documentation is much harder when your invention isn't real, but also ultimately that the result is all that matters. 

\section{On the Importance of Open Design Practices}

Hacking is pervasive \citep{paradiso2008}. Open source hardware practices, which share a deep link to hardware hacking \citep{williams2012}, are the bridge between longstanding methods in electronic music instrument designs and innovations in the field of accessible invention and fabrication. Through the work undertaken in this project and the writing that came out of it, it was most important to convey that making instruments could be as deeply personal and creative as composition, with endless inspiring parallels, an activity no longer reserved to academics or professionals. The goal isn't to create perfect products, rather, it is to produce something you are happy to use for musical purposes, regardless of its hypothetical shortcomings. In an age where the White House organizes its on \textit{Maker Faire} and half of the documented projects are artistic (White House, w2014), this work will not be alone.

\section{A note on diversity in audio electronics and this thesis}

This project hoped to address some cultural and social aspects of open hardware design in a musical context: it appears as important to acknowledge the implicit biases present in this process. Most of the people listed here are from the northeastern United States, and arguably all of them have been operating within a western art world. Three of the ten interviewees are women.

There are therefore clear biases. Future work should include a more thorough effort to seek out and include minorities. 

\section{Perspectives}

As the manufacture of circuit boards, surface mount electronics and increasingly powerful microcontrollers become available to the public, a number of new possibilities become available to designers of electronic music instruments. These engineering developments always offer an opportunity to think about how these tools are meant to be use in a musical context, but also a chance to think of how they can be repurposed as part of post-optimal and musical object, system or activity. This thesis suggests that explicitly acknowledging the creative potential of unpredictability through craft in electronic music instrument and identifying latent opportunities for post-optimal approaches in product design holds both scientific and compositional promise. As experiments develop and are materialized in an ever-greater variety of devices, further scholarship on this topic is necessary for the documentation and continued development of electronic music.   


