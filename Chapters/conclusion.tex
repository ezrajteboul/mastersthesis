% ----------------------------------------------------------------------------------------
% Conclusion
% ----------------------------------------------------------------------------------------
\chapter{Future: Conclusion}\label{conclusion}
\lhead{\chaptertitlename\ \thechapter. \emph{Conclusion}}
% ----------------------------------------------------------------------------------------
 
\section{Summary}

In 1950, Lester Paul's radio program explained his peculiar guitar playing sounds with a device, the \emph{Les Paulverizer}. Paul, as a noted enthusiast for innovative uses of technology in music, described this machine as one which could enable him to use ``one guitar, and make it sound like six''. The Les Paulverizer, in reality, was a farce. Its result was fabricated through the clever use of multiple tape machines and playback speed manipulations ahead of time, then synchronized with Paul's movements on stage. 

Living with the lie of the Paulverizer, its author eventually developed a system that could implement some of its fabled capacities in real time. The \emph{real} Les Paulverizer was in effect a foreshadowing of controllerism, knobs and switches made available to the musician to send controls offstage and trigger various effects\citep{kane2014}. 

Kane makes clear that Paul's reputation relied largely on technological innovations such as the Les Paulverizer. Arguably, Paul was succesful in playing with this subterfuge: obfuscation led to commercial and popular success, as well as creative uses of technology. 

Perner-Wilson's third point, \emph{transparency}, has not been much discussed in this work, but is very relevant here: by encouraging everyone to work out the implementation details of a device for themselves, instruments become more transparent.  

\section{Tutorials}


\section{afterword}


Freely shared and discussed technical information in the context of audio electronics was important before the existence of ``open source'' and as such, the tools that have allowed its development have largely been embraced by the musical electronics community. We've seen this embodied in various projects and explicitly expressed by their authors. Arguing the impact of open source, even in its modern sense, could  appear as an redundant task: entire genres are based on experimental, largely unregu



 This debate seems close to ``analog versus digital'', ``tubes versus ICs'' and ``tool versus instrument''. 

The more interesting question has therefore always been to understand the complex technical, cultural and social interplays that characterize the development of new electronic musical instruments. Overall, the ecosystem of invention within which developpers operate offers more background, history and documentation than ever before. Along with that, innovations are presented at breakneck speeds. 

Considering how long the works of Cage, Tudor and Buchla have been capitalized upon, it seems daunting to consider how long it will take for breakthroughs of that caliber to emerge, if ever. However, just like Tudor's work as a composer and circuit designer took the event of his death to really come to a forefront, documentation and responses while these practitioners are working seems crucial. If anything, the only purpose this document wishes to serve is to perpetuate the cycle of information sharing. 

This project wished to be aware of the social component of invention and composition. It is important to acknowledge the implicit biases present in this process: most of the people listed here are from the United States, and arguably all of them have been operating within a western art world. Three of the ten interviewees are women, but that is largely not 